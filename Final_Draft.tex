\documentclass[12pt]{amsart}

\usepackage[colorlinks=true, allcolors=blue]{hyperref}
\usepackage{tabularx}
\usepackage[utf8]{inputenc}
\usepackage{graphicx}
\usepackage{geometry}

\def\thesistitle{The Probability that the GCD of Products of Ideals in a Number Ring is B-smooth}
\def\name{Adrian Vazquez}
\newif \ifshort

%% Committee
%%%%%%%%%%%%%%%% Make appropriate substitutions here %%%%%%%%%%%%
\newcommand{\thesisadvisor}{Dr. Brian Sittinger}
\newcommand{\firstfaculty}{Dr. Ivona Grzegorczyk}
%\newcommand{\secondfaculty}{Dr. Jorge Garc\'{i}a}
\newcommand{\univefaculty}{Dr. Jill Leafstedt}
%%%%%%%%%%%%%%%% ~Make appropriate substitutions here %%%%%%%%%%%%

\newcommand{\committee}{\vspace*{1.25in} 
\begin{tabular}{ll}
   \multicolumn{2}{c}{\hspace*{2.9cm} APPROVED FOR THE MATHEMATICS PROGRAM}\\[10mm]
   %% Advisor
   \multicolumn{2}{c}{\hspace*{1.65cm}\rule{4.5in}{.01in}}\\[-4mm]
   \hspace*{3cm}\thesisadvisor, Thesis Advisor \hspace*{0cm}&  Date\\[4mm]
   %% Committee 1
   \multicolumn{2}{c}{\hspace*{1.65cm}\rule{4.5in}{.01in}}\\[-4mm]
   \hspace*{3cm}\firstfaculty, Thesis Committee \hspace*{0cm}&  Date\\[4mm]
   \multicolumn{2}{c}{\hspace*{1.65cm}\rule{4.5in}{.01in}}\\[-4mm]
   %% Committee 2
   %\hspace*{3cm}\secondfaculty, Thesis Committee  &  Date\\[20mm]
   %\multicolumn{2}{c}{\hspace*{6.15cm} APPROVED FOR THE UNIVERSITY}\\[6mm]
   %\multicolumn{2}{c}{\hspace*{1.9cm}\rule{4.5in}{.01in}}\\[-4mm]
    \hspace*{3cm}\univefaculty, AVP Extended University \hspace*{0cm}&  Date\\
 \end{tabular}}
%% ~Comittee

%% Included Preamble
\usepackage{amssymb,latexsym,amsmath,amsthm,amsfonts,graphics}
\newcommand\Month{%
\ifcase\number\month\relax%
\or January\or February\or March\or April\or May\or June%
\or July\or August\or September\or October\or November\or December\fi}
\newcommand{\R}{\mathbb{R}}
\newcommand{\C}{\mathbb{C}}
\newcommand{\N}{\mathbb{N}}
\newcommand{\Z}{\mathbb{Z}}
\newcommand{\Q}{\mathbb{Q}}
\newcommand{\F}{\mathbb{F}}
\newcommand{\OO}{\mathcal{O}}
\newcommand{\aaa}{\mathfrak{a}}
\newcommand{\ppp}{\mathfrak{p}}
\newtheorem{theorem}{Theorem}[subsection]
\newtheorem{proposition}[theorem]{Proposition}
\newtheorem{lemma}[theorem]{Lemma}
\newtheorem{example}[theorem]{Example}
\newtheorem{definition}[theorem]{Definition}
\newtheorem{algorithm}[theorem]{Algorithm}
\newtheorem{conj}{Conjecture}
\newtheorem{corollary}[theorem]{Corollary}
\theoremstyle{definition}
\newtheorem*{remark}{Remark}
%% thesis margins
\setlength{\headheight}{0in}
\setlength{\voffset}{.25in}
\setlength{\oddsidemargin}{0.0in}
\setlength{\evensidemargin}{0.0in}
\setlength{\textheight}{8.5in}
\setlength{\textwidth}{6.5in}
\setlength{\footskip}{0.55in}
\linespread{2.0}
%% !Included Preamble

%% Author Preamble
\usepackage{todonotes}
\newcommand{\f}[1]{\mathfrak{#1}}
\usepackage[capitalize]{cleveref}
\usepackage{bm}
%\usepackage[style=numeric, backend=biber]{biblatex}
%\addbibresource{thesis.bib}
%% ~Author Preamble

\begin{document}
\newpage
This page is here to make the page numbers come out correctly.

Do not print this page.
\newpage

\pagestyle{empty} \pagenumbering{roman}

\vspace*{.45in}

\begin{center}
\LARGE\it \thesistitle
\end{center}

\vskip 0.8in

\begin{center}
A Thesis Presented to \\
The Faculty of the Mathematics Program \\
California State University Channel Islands
\end{center}

\vskip .75in

\begin{center}
In (Partial) Fulfillment \\
of the Requirements for the Degree \\
Masters of Science
\end{center}

\vskip .75in

\begin{center}
by
\end{center}
\begin{center}
\name
\end{center}

\begin{center}
 \Month, \number\year
\end{center}
\newpage

%%%%%%%%%%%%%%%%%%%%%%%%%%%%%%%%%%%%%%%%%%%%%%%%%%%%%%%%%%%%%%%%%%%%%%%%%%%%%%%%
%% copyright notice
%% this is optional

% \newpage

% \setlength{\topmargin}{2cm}
% %\setlength{\textheight}{25.5cm}

% \vspace*{6in}

% \begin{center}

% \copyright{} \number\year\\
% \name\\
% ALL RIGHTS RESERVED


% \end{center}
%%%%%%%%%%%%%%%%%%%%%%%%%%%%%%%%%%%%%%%%%%%%%%%%%%%%%%%%%%%%%%%%%%%%%%%%%%%%%%%%
%% approval page

\newpage
 \setlength{\paperheight}{13in}
 \setlength{\topmargin}{1.25cm}

{\center\emph{Signature page for the Masters in Mathematics Thesis of \name}}

\committee
\newpage

%%%%%%%%%%%%%%%%%%%%%%%%%%%%%%%%%%%%%%%%%%%%%%%%%%%%%%%%%%%%%%%%%%%%%%%%%%%%%%%%
% A dedication is optional. Edit this section to make it your own or delete it.
 %\newpage
 %\setlength{\topmargin}{-0.84375in}
 %\setlength{\paperheight}{11.5in}
\newgeometry{top=0.1cm}
\pagestyle{plain}
\section*{Acknowledgements}
    First and foremost, I would like to express my sincerest gratitude to my thesis advisor Dr. Brian Sittinger, whose patience, guidance, and expertise were critical to my success in my independent studies courses and this thesis. I would also like to thank Dr. Ivona Grzegorczyk for her guidance and advice in academic matters and for serving in my thesis committee.
    
    I give a very special thanks to my former colleague Chris Luk, Chief Engineer at Raytheon Intelligence and Space. Chris' enthusiastic, scholarly, and cultured personality kept my motivation and spirits high during the long hours at the Point Mugu Engineering Center and encouraged me pursue a master's degree. I also give a special thanks to my supervisor Cecelia Feit, Chief Engineer Charles Yi, and Thomas To for their glowing letters of recommendation. I express my deepest gratitude to Gilbert Gutierrez and Hoang Tran for lending me their support and expertise during the execution of the AN/ALR-67 Task Orders.
    
    From the California State University Northridge mathematics department, I am deeply indebted to Dr. Jason Lo who revealed to me the beauty of algebra, and to Dr. Majid Mojirsheibani for revealing the beauty of probability theory.
     
    I thank professors George Cracuin, Debbie Wong, Emil Sargsyan, and Agnes Marsubian from the Los Angeles Mission College mathematics department, whose lectures and high standards fostered the growth, discipline, and mathematical foundation necessary for my success in university and my career.
     
    Finally, I give a very special thanks to A. D. Aleksandrov, A. N. Kolmogorov, M. A. Lavrent’ev for authoring \textit{Mathematics: Its Content, Methods and Meaning}, and the anonymous person who recommended that I read it. This book is the primary reason that I decided to study mathematics.
\restoregeometry
\newpage

%\pagestyle{plain}

%{\bf Abstract}
%\bigskip
\section*{Abstract}
As another extension of a result of Benkoski that the probability of $r$ positive integers being relatively $k$-prime equals $\frac{1}{\zeta(rk)}$, Cheon and Kim in 2016 derived the probability that the $k$-greatest common divisor ($k$-GCD) of products of positive integers is $B$-smooth. We first provide a more concise proof to this result. Then, we generalize this result to any given ring of algebraic integers.

\newpage
\tableofcontents
\newpage
%% List of Figures
\newpage
\pagenumbering{arabic}

%% Thesis Content
\section{Overview}

\subsection{Introduction}

In 1737, Euler derived a closed form of the sum of the reciprocals square of rational integers $\zeta(2)$, which was later generalized as a complex function in 1859 by Riemann. In 1849, Dirichlet \cite{Dirichlet} proved that the probability of two random integers are relatively prime is $\frac{1}{\zeta(2)}$, where $\zeta$ denotes the Riemann zeta function
$$\zeta(s) = \sum_{n=1}^{\infty} \frac{1}{n^s} = \prod_{p \; \text{prime}} \Big(1 - \frac{1}{p^s}\Big)^{-1}.$$
Subsequently in 1900, Lehmer \cite{Lehmer} extended Dirichlet's result by proving that the probability that $k$ integers are relatively prime is $\frac{1}{\zeta(k)}$. This was subsequently proved again by Nymann in 1970 by using Inclusion-Exclusion principle \cite{Nymann}. In 1976, Benkoski \cite{Benkoski} provided a proof showing that the probability that $k$ integers are relatively $r$-prime is $\frac{1}{\zeta(rk)}$. In 2010, Sittinger \cite{Sittinger} provided a more concise proof of Benkoski's theorem using methods developed by Nymann and extended the result to the algebraic integers. More recently, Cheon and Kim provided a proof for the probability that the GCD and $k$-GCD of products of positive integers is $B$-smooth \cite{Cheon}.

In this paper we provide more concise versions of Cheon and Kim's results by applying counting arguments. We then use our method to generalize their results to the algebraic integers. 

\subsection{\texorpdfstring{$B$}{\mathbb{B}}-smooth Numbers in Cryptography}
Given sufficiently large integers, there is currently no known efficient non-quantum prime factorization algorithm.
The difficulty of the prime factorization problem is leveraged by cryptographers to build cryptographic schemes that are difficult to crack using even the current most efficient algorithm, the General Number Field Sieve, running on hundreds of specialized computers (one can imagine the near impossibility of randomly guessing the factorization). 

With the rise of cybercrime, espionage, and the world's reliance on computers and digital information, it has become more critical to ensure that secure cryptography schemes exist in order to protect data privacy and integrity. To this end, factorization algorithms are constantly evolving in order to attack and prove the effectiveness of cryptography schemes that employ the difficulty of the prime factorization problem (some of these prime factorization algorithms can be applied on discrete logarithm problem based cryptography).

\noindent\textbf{Example:} RSA-130 is the following integer with 130 digits:
$$18070820886874048059516561644059055662781025167694013491701270214500566625402440\\$$
\vspace{-.7 in}
$$48387341127590812303371781887966563182013214880557.$$
\noindent It can be shown that RSA-130 has prime factorization $pq$, where
$$p = 39685999459597454290161126162883786067576449112810064832555157243,$$
\vspace{-.7 in}
$$q = 45534498646735972188403686897274408864356301263205069600999044599.$$
The history of factorization algorithms is rich and deep, but for the sake of brevity, we provide a condensed history of factorization algorithms and describe how smooth numbers and their error estimates play a central role in their discovery.

%** Trial division
The most na\"ive approach to factorization is called trial division (sometimes called direct search factorization) which was described by Fibonacci in his book \textit{Liber Abaci} \cite{Mollin}. In this algorithm, an integer $N$ is systematically divided by all integers in $[2, \sqrt{N}]$ until we find the factorization. While this algorithm is guaranteed to produce the correct result every time, it would be completely impractical for factoring large integers in a reasonable amount of time. Since then, mathematicians have devised algorithms which exploit the properties of integers in order to reduce the amount of computations required. In turn, this reduces the time need to find the factorization.

%**** Fermat's Factoring Method 
On April 7 1643, Fermat responded to a letter from Mersenne asking for a method to factor the integer $100895598169=898423\cdot 112303$ within a day \cite{Shiu}. Fermat discovered a method of factoring integers by utilizing the observation that any odd integer $n$ can be written as a difference of squares $N=a^2-b^2=(a+b)(a-b)$ where $a,b$ are non trivial. Fermat's Factorization Method works by selecting values for $a\geq\sqrt{N}$ until $\sqrt{a^2-N}$ is an integer, yielding $b$. However since
$\sqrt{N}=317640$, this means Fermat would have needed to take 187723 square roots. Consequently it is possible that Fermat must have known some of the optimizations that can be applied to his method.

%**** Dixon's Random Squares Factoring (1981)

In 1981, Dixon discovered an algorithm based on solving the congruence of squares problem: Find integers $x,y$ such that $x^2\equiv y^2 \bmod N$ \cite{Dixon}. In essence, instead of searching for the square of an integer as Fermat did, Dixon instead searches for integers that have small prime factors. 

Dixon's algorithm takes as an input a value $B$ (for which there exists an optimum value), a list of random integers in $[1,N]$, and the composite integer we wish to factor $N$ \cite{Dixon}. Using $B$, we build what is called a factor base, that is, all primes that are $B$-smooth. Next, we search for all positive integers $x$ such that $x^2 \bmod N$ is $B$-smooth, meaning its factors are in the factor base. For the sake of brevity, after a sufficient number of such values for $x$ are found, Dixon finds a set that satisfies the congruence of squares through the use of linear algebra, where for every pair $(x,y)$ there is a 50\% chance that $\gcd(N, x+y)$ factors $N$. With a result from Pomerance's Multiplicative Independence for Random Integers, Dixon rigorously proves the expected runtime of his algorithm \cite{Granville}.

%**** Quadratic Number Sieve
The quadratic sieve discovered by Pomerance is an optimization on Dixon's algorithm. Dixon's algorithm casts a wide net when attempting to search for $B$-smooth numbers by random sampling from a list of integers; however the quadratic number sieve optimizes this stage of Dixon's algorithm. As Pomerance describes, when attempting to recognize $B$-smooth numbers, there are fewer than $B$ primes up to $B$ \cite{Pomerance1}. This means that the number of trial divisions is at most $\max(\log_2{n}, \pi(B))$, where $\pi$ is the prime counting function. Pomerance describes that this search for suitable $B$-smooth numbers can be optimized by utilizing a method similar to the Sieve of Eratosthenes, we can identify a $B$-smooth number in $\log\log B$ steps on average.

In 1983, the quadratic sieve held the record for the largest factored number, namely one that had 71 digits and took 9.5 hours to factor \cite{Pomerance2}. In 1994, this algorithm factored RSA-129 (having 129 digits, and used 600 computers and 600 internet volunteers), however, Pomerance would later go on to describe that in April 1996, Pollard's general number sieve successfully factored RSA-130 (130 digits) in 15\% of the time that the quadratic number sieve would have taken \cite{Pomerance0}. 

The general number sieve, in broad terms, requires the selection of two polynomials $F(x,y), \, G(x,y)$ with which we search for coprime pairs of integers $a,b$ such that the integers $F(a,b)$ and $G(a,b)$ are both $B$-smooth \cite{Bai}. Then the probability of finding such pairs (which depends on the choice of polynomials) is inversely proportional to the runtime of the algorithm. In other words, the higher this probability is, the faster the algorithm runs.

We have now seen that finding efficient ways of identifying $B$-smooth numbers is critical to reducing the amount of operations in factoring algorithms. This results in reducing the computation time and resources, and improving the chances of solving the prime factorization problem that is at the core of many encryption schemes. 

Smooth numbers also have applications, including: cracking weak cryptography schemes, creating strong hashing functions, primality testing, and error correcting functions \cite{Naccache}.

\section{Probability that the GCD of Products of Positive Integers is  \texorpdfstring{$B$}{\mathbb{B}}-smooth}
We now provide concise proofs for the probability that the GCD and $k$-GCD are $B$-smooth by using counting arguments. Appendix A contains detailed versions of Cheon and Kim's original proofs using the inclusion-exclusion principle.

\begin{definition} Let $B$ be a fixed positive integer. A positive integer $x$ is \textbf{\bm{$B$}-smooth (\bm{$B$}-friable)} if $x$ has no prime divisor greater than $B$. 
\end{definition}

\noindent \textbf{Example:} Let $B = 20$. Then $57$ is $20$-smooth, because $57 = 2 \cdot 19$ and both 2 and 19 are less than 20. However $46$ is not $20$-smooth, because $46= 2 \cdot 23$ and 23 is a prime number greater than 20.

\vspace{.1 in}

For the remainder of this section, we find the probability that the greatest common divisor (GCD) of $m$ products of $n$ positive integers (all randomly chosen in a uniform and independent manner) is $B$-smooth.

\vspace{.1 in}

To this end, we fix a positive integer $N$. Then, we randomly choose uniformly and independently distributed  integers $r_{ij} \in [1, N]$ for each integer $1 \leq i \leq m$ and $1 \leq j \leq n$ where $m \geq 2$ and $n \geq 1$. Our goal is to compute the probability that an ordered $m$-tuple $(r_{ij})$ satisfies the following condition:
$$\gcd\Big(\prod_{j=1}^nr_{1j}, ..., \prod_{j=1}^n r_{mj}\Big)$$
is relatively prime to the first $l$ primes greater than $B$. 

Before stating the main theorem for this section, we establish the following useful lemma that we use in the ensuing derivation below.

\begin{lemma} If $n$ and $k$ are positive integers, then
	$$\frac{1}{n} \Big\lfloor \frac{n}{k} \Big\rfloor = \frac{1}{k} + O\Big(\frac{1}{n}\Big).$$
\end{lemma}

\begin{proof}
	Since we know by definition of the floor function that $0 \leq x - \lfloor x \rfloor < 1$ for any real number $x$, we can write
	$$0 \leq \frac{n}{k} - \Big\lfloor \frac{n}{k} \Big\rfloor < 1.$$
	Then, rearranging this inequality and dividing all three parts by $n$ yields
	$$0 \leq \frac{1}{k} - \frac{1}{n}\Big\lfloor \frac{n}{k}\Big\rfloor < \frac{1}{n}.$$
	The claim now directly follows.
\end{proof}

Now, we are ready to state and prove the main theorem of this section.

\begin{theorem}
	Fix positive integers $B$ and $N$, and let $p_1, p_2, ...$ be the primes greater than $B$ in increasing order. The probability of the GCD of products of random positive integers is not divisible by the first $l$ primes greater than $B$ is given by \label{probability-random-first-l}
	$$\prod_{i=1}^{l} \Big[1 - \Big(1 - \Big(1-\frac{1}{p_i}\Big)^n\Big)^m\Big].$$
\end{theorem}

\begin{proof}
	We first find the probability that 
	$$\gcd\Big(\prod_{j=1}^nr_{1j}, ..., \prod_{j=1}^n r_{mj}\Big)$$
	is coprime to the first $l$ primes greater than $B$.  To this end, note that for any prime $p$, there are $(N - \lfloor\frac{N}{p}\rfloor)^n$ products of $n$ positive integers each of which is at most $N$ that are not divisible by $p$. Hence, there are $N^n - (N - \lfloor\frac{N}{p}\rfloor)^n$ products of $n$ positive integers each of which is at most $N$ that are divisible by $p$. Therefore, there are 
	$$N^{mn} - \Big(N^n - \Big(N - \Big\lfloor\frac{N}{p}\Big\rfloor\Big)^n\Big)^m$$
	$m$-tuples of products of $n$ positive integers each of which is at most $N$ that are not divisible by $p$.
	
	Hence, the probability that an ordered $m$-tuple of products of $n$ positive integers each of which is at most $N$ are not divisible by $p$ is equal to
	$$\frac{N^{mn} - \Big(N^n - \Big(N - \Big\lfloor\frac{N}{p}\Big\rfloor\Big)^n\Big)^m}{N^{mn}} = 1 - \Big(1 - \Big(1 - \frac{1}{N} \Big\lfloor\frac{N}{p}\Big\rfloor\Big)^n\Big)^m.$$
	
	Then, since divisibility by finitely many primes yields independent events, the probability that the GCD of $m$ products of random positive integers at most $N$ are not divisible by the first $l$ primes greater than $B$ is equal to
	$$\prod_{i=1}^l \Big[1 - \Big(1 - \Big(1 - \frac{1}{N}\Big\lfloor\frac{N}{p_i}\Big\rfloor\Big)^n\Big)^m\Big].$$
	
	Since we ultimately want to let $N \to \infty$, we now use the above Lemma to estimate the error from removing the floor function from our probability statement. This gives us
	\begin{align*}& \prod_{i=1}^l \Big[1 - \Big(1 - \Big(1 - \Big(\frac{1}{p_i} + O\Big(\frac{1}{N}\Big)\Big) \Big)^n\Big)^m\Big]\\
		&= \prod_{i=1}^l \Big[1 - \Big(1 - \Big(1 - \frac{1}{p_i}  \Big)^n\Big)^m\Big] + O\Big(\frac{1}{N^{nm}}\Big).\end{align*}
	
	\noindent Finally letting $N \to \infty$ now gives us the desired result.
\end{proof}

We have found the probability that the GCD of products of random integers is not divisible by the first \(l\) primes greater than \(B\). We now use this result to find the probability that the GCD of products of random integers is not divisible by \textit{all} primes greater than $B$, thereby giving the probability that this GCD is $B$-smooth.

\begin{theorem} Suppose that each positive integer $r_{ij}$ is chosen uniformly and independently from $\{1, 2, ..., N\}$. Then, the probability that $\gcd(\prod_{j=1}^n r_{1j}, ... , \prod_{j=1}^n r_{mj})$ is $B$-smooth converges (as $N \to \infty$) to
	$$\prod_{p>B} \Big[1 - \Big(1 - \Big(1 - \frac{1}{p}\Big)^n \Big)^m \Big].$$
\end{theorem}

\begin{proof}
	Letting $P(l, N)$ denote the probability in the previous theorem (and for convenience setting $P(0, N) = 1$, we define $g_N(l) = P(l-1, n) - P(l, n)$. Then, we see that $g_N(l)$ is precisely the probability that $\gcd(\prod_{j=1}^n r_{1j}, ... , \prod_{j=1}^n r_{mj})$ is coprime to $p_1, ..., p_{l-1}$ and divisible by $p_l$ for uniformly and independently chosen $r_{ij}$'s from $\{1, 2, ..., N\}$. In particular, note that $g_N(l)$ is non-negative. We claim that we can move the limit to infinity past the summation sign:
	$$\lim_{N \to \infty} \sum_{k=1}^{\infty} g_N(k) = \sum_{k=1}^{\infty} \lim_{N \to \infty} g_N(k).$$
	In order to accomplish this, we invoke the Dominated Convergence Theorem (for series). To this end, we need to show that $g_N(l)$ is bounded above by $g(l) = \frac{n^m}{p_l^m}$ and $\sum_{l=1}^{\infty} g(l)$ converges.
	
	We start by showing that $g_N(l)$ is bounded. First of all, observe that
	$$g_N(l) \leq \text{Pr}\Big(p_l \Big| \gcd\Big(\prod_{j=1}^n r_{1j}, ... , \prod_{j=1}^n r_{mj}\Big)\Big).$$
	Computing the numerator to the probability on the last line, we find that
	\begin{align*} 
		\Big|\{(r_{ij}) : p_l \mid \prod_{j=1}^n r_{1j}\Big|^m &= \frac{(N^n - |\{(r_{1j}) : p_l \nmid \prod_{j=1}^n r_{1j}\}|)^m}{N^{mn}}\\
		&= \frac{(N^n - |\{r_{11} : p_l \nmid r_{11}\}|^n)^m}{N^{mn}}. \end{align*}
	Therefore, it follows that
	\begin{align*} 
		g_N(l) &\leq \frac{(N^n - |\{r_{11} : p_l \nmid r_{11}\}|^n)^m}{N^{mn}}\\
		&= \Big[1 - \Big(1 - \frac{1}{N} \Big\lfloor \frac{N}{p_l}\Big\rfloor\Big)^n\Big]^m\\
		&\leq \Big[1 - \Big(1 - \frac{1}{p_l}\Big)^n\Big]^m\\
		&\leq \frac{n^m}{p_l^m},\end{align*}
	where the last inequality directly follows from Bernoulli's inequality. Moreover $\sum_{l=1}^{\infty} g(l)$ converges, because we can bound this series from above with a convergent $p$-series (noting that $m \geq 2$):
	$$\sum_{l=1}^{\infty} g(l) = n^m \sum_{l=1}^{\infty} \frac{1}{p_l^m} < n^m \sum_{j=1}^{\infty} \frac{1}{j^m}.$$
	
	Having satisfied the hypotheses of the Dominated Convergence Theorem, we first observe that since $\sum_{k=1}^l g_N(k)$ is a telescoping sum, we obtain
	$$\sum_{k=1}^l g_N(k) = \sum_{k=1}^l (P(k-1, n) - P(k, n)) = 1 - P(l, N),$$ 
	and thus
	$$\lim_{N \to \infty} \sum_{k=1}^{\infty} g_N(k) = 1 - \prod_{p > B} \Big[1 - \Big(1 - \Big(1 - \frac{1}{p_i}  \Big)^n\Big)^m\Big].$$
	
	\noindent Then since $\displaystyle \sum_{k=1}^{\infty} \lim_{N \to \infty} g_N(k)$ represents the complement of the probability we wanted to compute, the Dominated Convergence Theorem yields the desired assertion.
\end{proof}

Having derived the probability that the GCD of products of randomly chosen positive integers is $B$-smooth, we now find more convenient bounds for this product representation.

\begin{theorem} The probability that $\gcd(\prod_{j=1}^n r_{1j}, ..., \prod_{j=1}^{n}r_{mj})$ is $B$-smooth is bounded above by 
	$$\frac{1}{\zeta(m)}\prod_{p\leq B}\Big(1-\frac{1}{p^m}\Big)^{-1},$$
	and is bounded below by 
	$$\prod_{B<p\leq\hat{n}} \Big[1 - \Big(1 - \Big(1 - \frac{1}{p}\Big)^n \Big)^m\Big] \cdot \prod_{\hat{n}<p\leq\hat{r}} \Big(1 - \Big(\frac{n}{p}\Big)^m\Big) \cdot \frac{1}{\zeta(s)},$$
	where $\hat{n}=\max\{n,B\}$, $\hat{r}=\max\{\hat{n}, \lfloor n^{\frac{m}{m-1}}+1\rfloor\}$, and $s = m(1 - \log_{\hat{r}}{n}) > 1$.
\end{theorem} 

\noindent \textbf{Remark:} It is understood that for the lower bound, the first finite product is equal to 1 if $B = \hat{n}$, and the second finite product is equal to 1 if $\hat{n}=\hat{r}$.

\begin{proof}
	We start by deriving the upper bound. Since $(1 - \frac{1}{p})^n$ decreases as a function of $n$ due to $1 - \frac{1}{p} \in (0, 1)$, we obtain
	$$\prod_{B<p\leq\hat{n}} \Big[1 - \Big(1 - \Big(1 - \frac{1}{p}\Big)^n \Big)^m \Big] \leq \prod_{p>B} \Big(1 - \frac{1}{p^m}\Big) = \frac{1}{\zeta(m)} \prod_{p\leq B} \Big(1 - \frac{1}{p^m}\Big)^{-1},$$
	where the rightmost equality uses the infinite product representation of the Riemann Zeta function.
	
	\vspace{.1 in}
	
	Next, we derive the lower bound for our probabilistic expression. We start by splitting the product at $\hat{n}$. Then we apply Bernoulli's inequality in the form $(1 - x)^n \geq 1 - nx$ where $x \in [0, 1]$ and $n \geq 1$, where we take $x = \frac{1}{p}$ where $p > \hat{n}$ is a prime number. This yields 
	$$\prod_{B<p\leq\hat{n}} \Big[1 - \Big(1 - \Big(1 - \frac{1}{p}\Big)^n \Big)^m \Big] \geq \prod_{B<p\leq \hat{n}}\Big[1 - \Big(1 - \Big(1 - \frac{1}{p}\Big)^n \Big)^m \Big] \cdot \prod_{p>\hat{n}}\Big(1 - \Big(\frac{n}{p}\Big)^m\Big).$$
	
	\vspace{.1 in}
	
	It remains to find a lower bound for $\prod_{p>\hat{n}} (1 - (\frac{n}{p})^m)$. From using the definition of $s$, we see that $p \geq \hat{r}$ is equivalent to $(\frac{n}{p})^m \leq \frac{1}{p^s}$. Using this fact in conjunction to the infinite product representation for the Riemann zeta function, we obtain 
	\begin{align*} \prod_{p>\hat{n}}\Big(1 - \Big(\frac{n}{p}\Big)^m\Big) &= \prod_{\hat{n}< p \leq \hat{r}}\Big(1 - \Big(\frac{n}{p}\Big)^m\Big) \cdot \prod_{p>\hat{r}}\Big(1 - \Big(\frac{n}{p}\Big)^m\Big)\\ &\geq  \prod_{\hat{n}< p \leq \hat{r}}\Big(1 - \Big(\frac{n}{p}\Big)^m\Big) \cdot \prod_{p>\hat{r}}\Big(1 - \frac{1}{p^s}\Big)\\ &\geq  \prod_{\hat{n}< p \leq \hat{r}}\Big(1 - \Big(\frac{n}{p}\Big)^m\Big) \cdot \frac{1}{\zeta(s)}. \end{align*}
	The claimed lower bound now directly follows. 
	
	\vspace{.1 in}
	
	It remains to show that $s > 1$. To this end, observe that since $r=\lfloor n^{\frac{m}{m-1}}+1\rfloor$ and $\hat{r} = \max\{n, B, r\}$, we have $\hat{r} \geq r > n^{\frac{m}{m-1}}$. Hence, we conclude that $s = m(1 - \log_{\hat{r}}{n}) > 1$ as required.
\end{proof}

\begin{theorem} The probability of the GCD of products of random integers is not divisible by the first $l$ primes greater than $B$ is given by 
	$$\lim_{N\to\infty} \frac{T(l,N)}{N^{nm}} = \prod_{i=1}^{l} \Big(1 - \Big[1 - \Big(1-\frac{1}{p^i}\Big)^n\Big]^m\Big).$$
\end{theorem}

\begin{proof}
	By the work done in the section, we see that $\displaystyle \frac{T(l,N)}{N^{nm}}$ is equal to 
	\begin{align*}
		&\frac{1}{N^{nm}}\sum_{P\subset X_l}{(-1)^{|P|}}\Big[\sum_{Q\subset P}(-1)^{|Q|}\Big(\sum_{R\subset Q}(-1)^{|R|}\Big\lfloor\frac{N}{\prod_{p\in R} p} \Big\rfloor\Big)^n\Big]^m\\
		&= \sum_{P\subset X_l}{(-1)^{|P|}} \Big[\sum_{Q\subset P}(-1)^{|Q|} \Big(\sum_{R\subset Q}(-1)^{|R|} \frac{1}{\prod_{p\in R} p}\Big)^n\Big]^m+O\Big(\frac{1}{N^{nm}}\Big).
	\end{align*}
	
	\noindent Next, factoring the principal term above yields
	\begin{align*}
		&\sum_{P\subset X_l}{(-1)^{|P|}} \Big[\sum_{Q\subset P}(-1)^{|Q|} \Big(\sum_{R\subset Q}(-1)^{|R|}\frac{1}{\prod_{p\in R} p}\Big)^n \Big]^m + O\Big(\frac{1}{N^{nm}}\Big)\\ &= \prod_{i=1}^{l} \Big(1 - \Big[1 - \Big(1 - \frac{1}{p_i}\Big)^n\Big]^m \Big) + O\Big(\frac{1}{N^{mn}}\Big). 
	\end{align*}
	
	\noindent Finally, taking the limit as $N \to \infty$ yields the desired conclusion.
\end{proof}

%%%%%%%%%%%%%%%%%%%%%%
\section{Probability that the \texorpdfstring{$k$}{\mathbb{k}}-GCD of Products of Positive Integers is  \texorpdfstring{$B$}{\mathbb{B}}-smooth}
In this section we extend the previous results to the $k$-GCD, and in a similar fashion, begin with a more concise proof.

\begin{definition} Fix a positive integer $k$. The \textbf{$k$-GCD} of $n$ nonzero integers $x_1, ..., x_n$, written $\gcd_k(x_1,x_2, ...,x_n)$, is the largest integer whose $k^{\text{th}}$ power divides each of $x_1, x_2, ..., x_n$.
\end{definition}	

\noindent Observe that when $k=1$, the $k$-GCD reduces to the classic GCD.

\begin{definition} When $\gcd_k(x_1,x_2,\mathellipsis,x_n)=1$ we say that  $x_1,x_2,\mathellipsis,x_n$ are \textbf{relatively $k$-prime}.
\end{definition}

\noindent \textbf{Example:}
As $k$ varies, it is possible that the $k$-GCD of a set of nonzero integers can change. 

\noindent For instance, although $\gcd(2^7, 2^6 \cdot 3, 2^8 \cdot 5) = 64$, we see that $\gcd_2(2^7, 2^6 \cdot 3, 2^8 \cdot 5)=8$, while $\gcd_3(2^7, 2^6 \cdot 3, 2^8 \cdot 5)=4$ and $\gcd_6(2^7, 2^6 \cdot 3, 2^8 \cdot 5) = 2$. 

\noindent However since $\gcd_8(2^7, 2^6 \cdot 3, 2^8 \cdot 5) = 1$, we find that $2^7, 2^6 \cdot 3, 2^8 \cdot 5$ are relatively $8$-prime.

As in the previous proofs, we fix a positive integer $N$. Then, we randomly, uniformly, and independently choose integers $r_{ij} \in [1, N]$ for each integer $1 \leq i \leq m$ and $1 \leq j \leq n$ where $m \geq 2$ and $n \geq 1$. We want to find the probability that an ordered $m$-tuple $(r_{ij})$ such that 
$$\gcd{}_k\Big(\prod_{j=1}^nr_{1j}, ..., \prod_{j=1}^n r_{mj}\Big)$$
is coprime to the first $l$ primes greater than $B$. 

Before stating the main theorem for this section, we establish the following lemmas that we use in the ensuing derivation below. We first need the following terminology from elementary number theory.

\begin{definition}
	Let $a, n$ be integers and $p$ be a prime number. We say that $p$ \textbf{exactly divides} $n$, written as $p^a \mid \mid n$, if $p^a \mid n$ but $p^{a+1} \nmid n$. 
\end{definition}

\begin{lemma}
	Fix positive integers $k$ and $N$. Let $Q_{p^k}(N)$ the number of ordered $n$-tuples of positive integers in which each entry is at most $N$ such that $p^k$ does not divide the product of these entries. Then, we have
	$$Q_{p^k}(N) = \sum_{a_1+...+a_n < k} \Big[\prod_{j=1}^n \Big(\Big\lfloor \frac{N}{p^{a_j}}\Big\rfloor - \Big\lfloor \frac{N}{p^{a_j + 1}}\Big\rfloor\Big)\Big].$$
\end{lemma}

\begin{proof}
	Since the number of positive integers at most $N$ that are divisible by $p^j$ (for some prime $p$ and positive integer $j$) equals $\lfloor\frac{N}{p^j}\rfloor$, it follows that the number of positive integers at most $N$ where $p^j$ \textit{exactly} divides the positive integer is equal to
	$$\Big\lfloor\frac{N}{p^j}\Big\rfloor - \Big\lfloor\frac{N}{p^{j+1}}\Big\rfloor.$$
	
	\noindent Next, the number of ordered $n$-tuples $(m_1, ..., m_n)$ of positive integers in which each entry is at most $N$ satisfying the conditions $p^{a_j} \mid\mid m_j$ for each $j = 1, ..., n$ is given by the quantity
	$$\prod_{j=1}^n \Big(\Big\lfloor\frac{N}{p^{a_j}}\Big\rfloor - \Big\lfloor\frac{N}{p^{a_j+1}}\Big\rfloor\Big).$$
	
	\noindent Therefore, we conclude that $Q_{p^k}(N)$ is given as follows:
	$$Q_{p^k}(N) = \sum_{a_1+...+a_n < k} \Big[\prod_{j=1}^n \Big(\Big\lfloor \frac{N}{p^{a_j}}\Big\rfloor - \Big\lfloor \frac{N}{p^{a_j + 1}}\Big\rfloor\Big)\Big].$$
\end{proof}

We now give a less cumbersome representation for $Q_{p^k}(N)$ by estimating the error from removing the floor functions from its expression.

\begin{lemma}
	Using the same notation as in the previous lemma, we have the following estimate:
	$$Q_{p^k}(N) = \Big(N - \frac{N}{p}\Big)^n \Big(1 + \frac{{}_nH_1}{p} + ... + \frac{{}_nH_{k-1}}{p^{k-1}}\Big) + O(1).$$
\end{lemma}

\begin{proof}
	Applying the basic estimate $\lfloor x \rfloor  = x + O(1)$ to the result from the previous lemma, we obtain
	\begin{align*} Q_{p^k}(N) &= \sum_{a_1+...+a_n < k} \Big[\prod_{j=1}^n \Big( \frac{N}{p^{a_j}} - \frac{N}{p^{a_j + 1}} + O(1)\Big)\Big]\\ &= \Big(N - \frac{N}{p}\Big)^n \sum_{a_1+...+a_n < k} \frac{1}{p^{a_1 + ... + a_n}} + O(1). \end{align*} 
	\noindent Next for each non-negative integer less than $k$, there are ${}_nH_j = \binom{n-1+j}{j}$ solutions in non-negative integers to the equation $a_1+...+a_n = j$. Hence we conclude that
	$$Q_{p^k}(N) = \Big(N - \frac{N}{p}\Big)^n \Big(1 + \frac{{}_nH_1}{p} + ... + \frac{{}_nH_{k-1}}{p^{k-1}}\Big) + O(1),$$
	as required.
\end{proof}

Now, we are ready to state and prove the main theorem of this section.

\begin{theorem}
	Fix positive integers $B$ and $N$, and let $p_1, p_2, ...$ be the primes greater than $B$ in increasing order. The probability of the $k$-GCD of products of random integers is not divisible by the first $l$ primes greater than $B$ is given by 
	$$\prod_{i=1}^{l} \Big[1 - \Big(1 - \Big(1-\frac{1}{p_i}\Big)^n \Big(1 + \frac{{}_nH_1}{p} + ... + \frac{{}_nH_{k-1}}{p^{k-1}}\Big)\Big)^m\Big].$$
\end{theorem}

\begin{proof}
	Letting $Q_{p^k}(N)$ denote the number of $n$ positive integers each of which is at most $N$ that are not divisible by $p^k$ (as in the previous two lemmas), there are $N^n - Q_{p^k}(N)$ products of $n$ positive integers each of which is at most $N$ that are divisible by $p^k$. Therefore, there are 
	$$N^{mn} - (N^n - Q_{p^k}(N))^m$$
	ordered $m$-tuples of products of $n$ positive integers each of which is at most $N$ that are not divisible by $p^k$.
	Hence, the probability that an ordered $m$-tuples of products of $n$ positive integers each of which is at most $N$ that are not divisible by $p^k$ is equal to
	$$\frac{N^{mn} - (N^n - Q_{p^k}(N))^m}{N^{mn}} = 1 - \Big(1 - \frac{Q_{p^k}(N)}{N^n}\Big)^m.$$
	Now, we use the result of the previous lemma to rewrite the latter probability as follows:
	$$1 - \Big[1 - \Big(1 - \frac{1}{p}\Big)^n \Big(1 + \frac{{}_nH_1}{p} + ... + \frac{{}_nH_{k-1}}{p^{k-1}}\Big)\Big]^m + O\Big(\frac{1}{N^{mn}}\Big).$$
	Then, since divisibility by finitely many primes give independent events, we deduce that the probability that an ordered $m$-tuples of products of $n$ positive integers each of which is at most $N$ that are not divisible by $p_1^k$, ..., $p_l^k$ is equal to
	\begin{align*} &\prod_{i=1}^l \Big[1 - \Big(1 -\Big(1-\frac{1}{p_i}\Big)^n\Big(1+\frac{{}_nH_1}{p_i}+\mathellipsis+\frac{{}_nH_{k-1}}{p_i^{k-1}}\Big)\Big)^m +O\Big(\frac{1}{N^{mn}}\Big) \Big]\\
		&=\prod_{i=1}^l \Big[1 - \Big(1 -\Big(1-\frac{1}{p_i}\Big)^n\Big(1+\frac{{}_nH_1}{p_i}+\mathellipsis+\frac{{}_nH_{k-1}}{p_i^{k-1}}\Big)\Big)^m\Big] + O\Big(\frac{1}{N^{mn}}\Big). \end{align*}
	
	\noindent Letting $N \to \infty$ now gives us the desired result.
\end{proof}

We have found the probability that the $k$-GCD of products of random integers is not divisible by the first \(l\) primes greater than \(B\). We now use this result to find the probability that the $k$-GCD of products of random integers is not divisible by \textit{all} primes greater than $B$, thereby giving the probability that this $k$-GCD is $B$-smooth.

\begin{theorem} Suppose that each $r_{ij}$ is chosen uniformly and independently from $\{1, 2, ..., N\}$. Then, the probability that $\gcd_k(\prod_{j=1}^n r_{1j}, ... , \prod_{j=1}^n r_{mj})$ is $B$-smooth converges (as $N \to \infty$) to
	$$\prod_{p>B} \Big[1 - \Big(1 - \Big(1-\frac{1}{p_i}\Big)^n \Big(1 + \frac{{}_nH_1}{p} + ... + \frac{{}_nH_{k-1}}{p^{k-1}}\Big)\Big)^m\Big].$$
\end{theorem}

\begin{proof}
	Letting $P(l, N)$ denote the probability in the previous theorem (and for convenience setting $P(0, N) = 1$, we define $g_N(l) = P(l-1, n) - P(l, n)$. Then, we see that $g_N(l)$ is precisely the probability that $k\text{-}\gcd(\prod_{j=1}^n r_{1j}, ... , \prod_{j=1}^n r_{mj})$ is coprime to $p_1, \mathellipsis, p_{l-1}$ and divisible by $p_l$ for uniformly and independently chosen $r_{ij}$'s from $\{1, 2, ..., N\}$. In particular, note that $g_N(l)$ is non-negative. We claim that we can move the limit to infinity past the summation sign:
	$$\lim_{N \to \infty} \sum_{s=1}^{\infty} g_N(s) = \sum_{s=1}^{\infty} \lim_{N \to \infty} g_N(s).$$
	In order to accomplish this, we invoke the Dominated Convergence Theorem (for series). To this end, we need to show that $g_N(l)$ is bounded above by $g(l) = \frac{n^m}{p_{l}^{km}}$ and $\sum_{l=1}^{\infty} g(l)$ converges.
	
	We start by showing that $g_N(l)$ is bounded. First of all, observe that
	$$g_N(l) \leq \text{Pr}\Big(p_l^k \Big| \gcd\Big(\prod_{j=1}^n r_{1j}, ... , \prod_{j=1}^n r_{mj}\Big)\Big).$$
	Computing the numerator to the probability on the last line, we find that
	\begin{align*} 
		\Big|\{(r_{ij}) : p_l^k \mid \prod_{j=1}^n r_{1j}\}\Big|^m &= \frac{(N^n - |\{(r_{1j}) : p_l^k \nmid \prod_{j=1}^n r_{1j}\}|)^m}{N^{mn}}\\
		&= \frac{(N^n - |\{r_{11} : p_l^k \nmid r_{11}\}|^n)^m}{N^{mn}}. \end{align*}
	Therefore, it follows that
	\begin{align*} 
		g_N(l) &\leq \frac{(N^n - |\{r_{11} : p_l^k \nmid r_{11}\}|^n)^m}{N^{mn}}\\
		&= \Big[1 - \Big(1 - \frac{1}{N} \Big\lfloor \frac{N}{p_l^k}\Big\rfloor\Big)^n\Big]^m\\
		&\leq \Big[1 - \Big(1 - \frac{1}{p_l^k}\Big)^n\Big]^m\\
        &\leq \frac{n^m}{p_l^{km}},\end{align*}
	where the last inequality directly follows from Bernoulli's inequality. Moreover $\sum_{l=1}^{\infty} g(l)$ converges, because we can bound this series from above with a convergent $p$-series (noting that $m \geq 2$):
	$$\sum_{l=1}^{\infty} g(l) = n^m \sum_{l=1}^{\infty} \frac{1}{p_l^{km}} < n^m \sum_{j=1}^{\infty} \frac{1}{j^{km}}.$$
	
	Having satisfied the hypotheses of the Dominated Convergence Theorem, we first observe that since $\sum_{s=1}^l g_N(s)$ is a telescoping sum, we obtain
	$$\sum_{s=1}^l g_N(s) = \sum_{s=1}^l (P(s-1, n) - P(s, n)) = 1 - P(l, N),$$ 
	and thus
	$$\lim_{N \to \infty} \sum_{s=1}^{\infty} g_N(s) = 1 - \prod_{p > B} \Big[1 - \Big(1 - \Big(1-\frac{1}{p_i}\Big)^n \Big(1 + \frac{{}_nH_1}{p} + ... + \frac{{}_nH_{k-1}}{p^{k-1}}\Big)\Big)^m\Big].$$
	
	\noindent Then since $\displaystyle \sum_{s=1}^{\infty} \lim_{N \to \infty} g_N(s)$ represents the complement of the probability we wanted to compute, the Dominated Convergence Theorem yields the desired assertion.
\end{proof}



\section{Background Material for the Algebraic Integer Case}
We now provide some key concepts involving algebraic integers that we use for the remainder of the proofs in this paper. Unsurprisingly, the derivations of these results have the similar flow and form to their algebraic integer counterparts, and by choosing particular values, will yield the rational integer cases. For further details, please consult a textbook on Algebraic Number Theory, such as \cite{Marcus}.

\begin{definition} A complex number is an \textbf{algebraic number} if it satisfies a polynomial with rational coefficients.
\end{definition}

\begin{definition}A number is an \textbf{algebraic integer} if it satisfies a monic polynomial with integer coefficients.
	
	The set of all algebraic integers forms a ring and is denoted by \(\mathbb{A}\).
\end{definition}

\begin{definition} The \textbf{algebraic number ring} (denoted by \(\mathcal{O}\)) is the ring of algebraic integers of the algebraic field \(K\) given by \(\mathcal{O}=K\cap \mathbb{A}\). 
\end{definition}

\begin{definition} 
	We say that an ideal \(\f{p}\) in a number ring $\mathcal{O}$ is \textbf{prime} if whenever \(\f{ab}\subseteq \f{p}\) for some ideals \(\f{a},\f{b}\) in $\mathcal{O}$ implies \(\f{a}\subseteq \f{p}\) or \(\f{b} \subseteq \f{p}\).  
\end{definition}
A key result is that a nonzero ideal in a Dedekind domain (such as a number ring) enjoys unique factorization into prime ideals. 

\begin{definition} \label{ideal-norm}
	Let $\mathfrak{a}$ be a nonzero ideal of a number ring $\mathcal{O}$. We
	define the \textbf{norm} of \(\f{a}\) as \(\f{N(a)}:=|\mathcal{O}/\f{a}|\).
\end{definition}

Not only is the norm on ideals finite, it also gives a completely multiplicative function. That is, for any ideals $\mathfrak{a}, \mathfrak{b}$ in $\mathcal{O}$, we have
$$\mathfrak{N}(\mathfrak{a} \mathfrak{b}) = \mathfrak{N}(\mathfrak{a}) \mathfrak{N}(\mathfrak{b}).$$
Having defined the norm, we can now give the number ring generalization of the Riemann zeta function. 
\begin{definition} Let \(\mathcal{O}\) be an algebraic number ring. The \textbf{Dedekind zeta function} of \(\mathcal{O}\) is given by 
$$\zeta_\mathcal{O}(s)=\sum_{\f{a\subseteq \mathcal{O}}}\frac{1}{\mathfrak{N}(\mathfrak{a})^s},$$
where the sum is over all non-zero ideals of \(\mathcal{O}\).
\end{definition}

Equivalently, the Dedekind zeta function can be defined by the Dirichlet series
$$\zeta_\mathcal{O}(s)=\sum_{n=1}^{\infty}\frac{c_n}{n^s},$$
where $c_n$ is the number of ideals with norm $n$. Moreover, the Dedekind Zeta function can be represented by the Euler Product indexed by all prime \textit{ideals} of $\mathcal{O}$:
$$\zeta_{\mathcal{O}}(s)=\prod_{\f{p \text{ prime}}}(1-\f{N(p)}^{-s})^{-1}.$$

\begin{remark}
	As with the Riemann Zeta Function, the Dedekind Zeta Function converges for all \(\text{Re} (s)>1\).
\end{remark}

\begin{definition} For an algebraic number ring $\mathcal{O}$, let the M\"obius function $\mu:\mathcal{O}\to \{0,\pm 1\}$ be given as
	\begin{equation*}
		\mu(\f{a})=\begin{cases}
			1 & \text{if \(\f{\mathfrak{N}(a)}=1\)}\\
			0 & \text{if \(\f{a\subseteq \f{p}}^2\) for some prime \(\f{p}\) }\\
			(-1)^r & \text{if \(\f{a}=\f{p}_1 \f{p}_2\mathellipsis\f{p}_r\) for distinct primes \( \f{p}_1,\f{p}_2,\mathellipsis, \f{p}_r\)}.
		\end{cases}
	\end{equation*}
\end{definition}

\begin{definition} Fix $r \in \mathbb{N}$. We say that nonzero ideals \(\f{a}_1,\f{a}_2,\mathellipsis,\f{a}_k \subseteq \mathcal{O}\)  are \textbf{relatively \(r\)-prime} if \(\f{a}_1,\f{a}_2,\mathellipsis,\f{a}_k\not\subseteq \f{b}^r\) for any nonzero and proper ideal \(\f{b}\).
\end{definition}

\begin{definition} \label{H}
	Let $H(n)$ denote the number of ideals in $\mathcal{O}$ with norm less than or equal to $n$.
\end{definition}

Note that $n$ need not be an integer. The following theorem provides an estimate that we repeatedly use in the derivations that follow. A proof of this can be found in \cite{Marcus}.

\begin{theorem} There exists $c > 0$ such that $H(n)=cn+O(n^{1-\epsilon})$ where $\epsilon = [K : \mathbb{Q}]^{-1}$. \label{H-estimate}
\end{theorem}

\noindent \textbf{Remark:} If $K=\mathbb{Q}$ so that $\mathcal{O} = \mathbb{Z}$, then $H$ reduces to the floor function. 

\section{Probability that the GCD of Products of Algebraic Integers is \texorpdfstring{$B$}{\mathbb{B}}-smooth}

\begin{definition} Let $B$ be a fixed positive integer and $\mathcal{O}$ be an algebraic number ring. A nonzero ideal $\mathfrak{a}$ in $\mathcal{O}$ is \textbf{$B$-smooth} (\textbf{$B$-friable}) if $\mathfrak{a}$ has no prime ideal factor whose norm is greater than $B$. 
\end{definition}

In the case where $\mathcal{O}$ is a PID, we can replace ideals with elements and check that the elements have no prime factors have absolute value of a norm greater than $B$.

\noindent \textbf{Example:} Let $\mathcal{O} = \mathbb{Z}[i]$, the ring of Gaussian integers, and take $B =  15$. Then $7 + 6i$ is not 15-smooth, because its prime factorization is $(2+i)(4+i)$, and $N(4+i) = 17 > 15$. On the other hand, $9-3i$ is 15-smooth, because its prime factorization is $3(3 - i)$, and the norm of each of its prime factors is less than 15.

For the remainder of this section, we are going to find the probability that the greatest common divisor (GCD) of $m$ products of $n$ nonzero ideals in an algebraic number ring $\mathcal{O}$ (all randomly chosen in a uniform and independent manner) is $B$-smooth. 

\vspace{.1 in}

To this end, we fix a positive integer $N$. Then, we randomly choose uniformly and independently ideals $\mathfrak{a}_{ij}$ whose norms at most $N$ for each integer $1 \leq i \leq m$ and $1 \leq j \leq n$ where $m \geq 2$ and $n \geq 1$. We want to find the probability that an ordered $m$-tuple $(\mathfrak{a}_{ij})$ such that 
$$\gcd\Big(\prod_{j=1}^n \mathfrak{a}_{1j}, ..., \prod_{j=1}^n \mathfrak{a}_{mj}\Big)$$
is coprime to the first $l$ prime ideals (arranged in non-decreasing order by norm) having norm greater than $B$. We start with the following error bound.

\begin{lemma}\label{H-estimate-frac} If $k$ is a positive integer, then there exists a constant $\epsilon > 0$ such that for all sufficiently large $n$:
	$$\frac{1}{H(n)} H\Big(\frac{n}{k}\Big) = \frac{1}{k} + O\Big(\frac{1}{n^{\epsilon}}\Big).$$
\end{lemma}

\begin{proof}
	Since we know from \cref{H-estimate} that $H(x) = cx + O(x^{1-\epsilon})$ for some constants $c, \epsilon > 0$, there exists a constant $A > 0$ such that for all sufficiently large $x$:
	$$|H(x) - cx| < Ax^{1-\epsilon}.$$
	By using this inequality, we deduce for all sufficiently large $n$ that
	$$\frac{1}{H(n)} H\Big(\frac{n}{k}\Big) \leq \frac{\frac{cn}{k} + A(\frac{n}{k})^{1-\epsilon}}{cn - An^{1-\epsilon}} = \frac{\frac{cn}{k} (1 + \frac{Ak^{\epsilon}}{c} n^{-\epsilon})}{cn(1 - \frac{A}{c}n^{-\epsilon})}.$$
	Applying the geometric series to the right side of the inequality above, we obtain for all sufficiently large $n$:
	$$\frac{1}{H(n)} H\Big(\frac{n}{k}\Big) = \frac{1}{k} \Big(1 + \frac{Ak^{\epsilon}}{c} n^{-\epsilon}\Big) \Big(1 + O\Big(\frac{1}{n^{\epsilon}}\Big)\Big) = \frac{1}{k} + O\Big(\frac{1}{n^{\epsilon}}\Big),$$
	and that is what we wanted to prove.
\end{proof}

\begin{theorem}
	Fix positive integers $B$ and $N$ and a number ring $\mathcal{O}$. Let $\mathfrak{p}_1, \mathfrak{p}_2, ...$ be the prime ideals in $\mathcal{O}$ with norm greater than $B$ arranged in non-decreasing order by norm. Then, the probability of the GCD of products of random ideals is not divisible by the first $l$ prime ideals greater than $B$ is given by 
	$$\prod_{i=1}^{l} \Big[1 - \Big(1 - \Big(1-\frac{1}{\mathfrak{N}(\mathfrak{p}_i)}\Big)^n\Big)^m\Big].$$
\end{theorem}

\begin{proof}
	For any prime ideal $\mathfrak{p}$, there are $\bigl[H(N)-H\bigl(\frac{N}{\f{N(p)}}\bigr)\bigr]^n$ products of ideals with norm at most $N$ that are not divisible by $\f{N(p)}$. Hence, there are  $H(N)^n-\bigl[H(N)-H\bigl(\frac{N}{\f{N(p)}}\bigr)\bigr]^n$ products of ideals with norm at most \(N\) that are divisible by $\f{N(p)}$. Therefore, there are 
	$$H(N)^{nm}-\Big[H(N)^n - \Big(H(N) - H\Big(\frac{N}{\f{N(p)}}\Big)\Big)^n\Big]^m$$
	ordered $m$-tuples of products of ideals, each of which has norm at most $N$ and is not divisible by $\f{N(p)}$. Then, the probability that an ordered $m$-tuple of products of ideals, each of which has norm at most $N$ and is not divisible by $\f{N(p)}$ is equal to
	$$\frac{H(N)^{nm}-\Big[H(N)^n - \Big(H(N) - H\Big(\frac{N}{\f{N(p)}}\Big)\Big)^n\Big]^m}{H(N)^{nm}} = 1 - \Big[1 - \Big(1 - \frac{1}{H(N)} H\Big(\frac{N}{\f{N(p)}}\Big)\Big)^n\Big]^m.$$
	Then since divisibility by finitely many primes yields independent events, we find that the probability that the GCD of $m$ products of random ideals with norm at most $N$ are not divisible by the first $l$ prime ideals:
	$$\prod_{i=1}^{l} \Big[1 - \Big(1 - \Big(1 - \frac{1}{H(N)} H\Big(\frac{N}{\f{N(p)}}\Big)\Big)^n \Big)^m \Big].$$
	
	We now examine what happens to our probability as $N \to \infty$. Using the estimate for $H(n)$ from \cref{H-estimate-frac} then gives us
	\begin{align*}
		&\prod_{i=1}^{l} \Big[1 - \Big(1 - \Big(1 - \frac{1}{\f{N}(\f{p}_i)} + O\Big(\frac{1}{N^{\epsilon}}\Big) \Big)^n \Big)^m\Big]\\
		&=\prod_{i=1}^{l} \Big[1 - \Big(1 - \Big(1 - \frac{1}{\f{N}(\f{p}_i)}\Big)^n \Big)^m \Big] + O\Big(\frac{1}{N^{\epsilon}}\Big).
	\end{align*}
	Finally, letting $N\to\infty$ gives us the desired result. 
\end{proof}

Setting \(\mathcal{O}=\mathbb{Z}\), prime ideals are substituted for the prime integers, and \(\epsilon=1\), yields the same result as for the integers case. 

As in the case for the integers, this only provides the probability for the first \(l\) prime ideals greater than \(B\). Using the Dominated Convergence Theorem, we can take the limit as \(l\) approaches infinity.

\begin{theorem}
	The probability that the GCD of products of random ideals of algebraic integers in a given number ring $\mathcal{O}$ is $B$-smooth converges to
	$$\prod_{\f{N}(\f{p})>B}\Big[1 - \Big(1 - \Big(1 - \frac{1}{\f{N}(\f{p})}\Big)^n \Big)^m \Big].$$
\end{theorem}

\begin{proof}
	Letting $P(l, N)$ denote the probability in the previous theorem (and for convenience setting $P(0, N) = 1$, we define $g_N(l) = P(l-1, n) - P(l, n)$. Then, we see that $g_N(l)$ is precisely the probability that $\gcd(\prod_{j=1}^n \mathfrak{a}_{1j}, ... , \prod_{j=1}^n \mathfrak{a}_{mj})$ is coprime to $\mathfrak{p}_1, ..., \mathfrak{p}_{l-1}$ and divisible by $\mathfrak{p}_l$ for uniformly and independently chosen $\mathfrak{a}_{ij}$'s from $\{1, 2, ..., N\}$. In particular, note that $g_N(l)$ is non-negative. We claim that we can move the limit to infinity past the summation sign:
	$$\lim_{N \to \infty} \sum_{k=1}^{\infty} g_N(k) = \sum_{k=1}^{\infty} \lim_{N \to \infty} g_N(k).$$
	In order to accomplish this, we invoke the Dominated Convergence Theorem (for series). To this end, we need to show that $g_N(l)$ is bounded above by $g(l) = \frac{n^m}{\mathfrak{N}(\mathfrak{p}_l)^m}$ and $\sum_{l=1}^{\infty} g(l)$ converges.
	
	We start by showing that $g_N(l)$ is bounded. First of all, observe that
	$$g_N(l) \leq \text{Pr}\Big(\mathfrak{p}_l \Big| \gcd\Big(\prod_{j=1}^n \mathfrak{a}_{1j}, ... , \prod_{j=1}^n \mathfrak{a}_{mj}\Big)\Big).$$
	Computing the numerator to the probability on the last line, we find that
	\begin{align*} 
		\Big|\{(\f{a}_{ij}) : \mathfrak{p}_l \mid \prod_{j=1}^n \mathfrak{a}_{1j}\}\Big|^m &= \frac{(H(N)^n - |\{(\mathfrak{a}_{1j}) : \mathfrak{p}_l \nmid \prod_{j=1}^n \mathfrak{a}_{1j}\}|)^m}{H(N)^{mn}}\\ &= \frac{(H(N)^n - |\{\f{a}_{11} : \mathfrak{p}_l \nmid \mathfrak{a}_{11}\}|^n)^m}{H(N)^{mn}}. \end{align*}
	Therefore, it follows for all sufficiently large $N$ that
	\begin{align*} 
		g_N(l) &\leq \frac{(H(N)^n - |\{\f{a}_{11} : \mathfrak{p}_l \nmid \mathfrak{a}_{11}\}|^n)^m}{H(N)^{mn}}\\ &\leq \Big(1 - \Big(1 - \frac{1}{H(N)} H\Big(\frac{N}{\f{N}(\f{p}_l)}\Big)\Big)^n\Big)^m\\
		&= \Big(1- \Big(1 - \Big(\frac{1}{\mathfrak{N}(\mathfrak{p}_l)} + \frac{A}{N^{\epsilon}}\Big)\Big)^n \Big)^m \text{ for some } A > 0\\ &\leq \Big(1- \Big(1 - \frac{2}{\mathfrak{N}(\mathfrak{p}_l)}\Big)^n \Big)^m, \end{align*}
	
	The latter expression is bounded above by $g(l) = \frac{(2n)^m}{\f{N(\f{p}_l)}^m}$ by Bernoulli's inequality in the form $1-(1-x)^t\leq xt$ for $t\geq 1$ and $0\leq x \leq 1$. Moreover $\sum_{l=1}^{\infty} g(l)$ converges, because we can bound this series from above with a constant multiple of the Dedekind zeta function (with $m \geq 2$):
	$$\sum_{l=1}^{\infty}g(l)\leq (2n)^m\sum_{l=1}^{\infty}\frac{1}{\f{N(\f{p}_l)}^m}\leq (2n)^m \zeta_{\mathcal{O}}(m) < \infty.$$
	
	Having satisfied the hypotheses of the Dominated Convergence Theorem, we first observe that since $\sum_{k=1}^l g_N(k)$ is a telescoping sum, we obtain
	$$\sum_{k=1}^l g_N(k) = \sum_{k=1}^l (P(k-1, n) - P(k, n)) = 1 - P(l,N),$$ 
	and thus
	$$\lim_{N \to \infty} \sum_{k=1}^{\infty} g_N(k) = 1 - \prod_{\mathfrak{N}(\mathfrak{p}) > B} \Big[1 - \Big(1 - \Big(1 - \frac{1}{\mathfrak{N}(\mathfrak{p})}  \Big)^n\Big)^m\Big].$$
	
	\noindent Since $\displaystyle \sum_{k=1}^{\infty} \lim_{N \to \infty} g_N(k)$ represents the complement of the probability we wanted to compute, the Dominated Convergence Theorem yields the desired assertion.
\end{proof}


\section{Probability that the \texorpdfstring{$k$}{\mathbb{k}}-GCD of Products of Algebraic Integers is \texorpdfstring{$B$}{\mathbb{B}}-smooth}

\begin{definition} Fix a positive integer $k$. The \bm{$k\text{-}\gcd$} of $n$ prime ideals $\f{a}_1$, $\f{a}_2$,$\mathellipsis$, $\f{a}_n$ in a number ring $\mathcal{O}$, written $\gcd_k(\f{a}_1,\f{a}_2, \mathellipsis,\f{a}_n)$, is the largest prime ideal whose $k^{\text{th}}$ power divides each of $\f{a}_1, \f{a}_2,\mathellipsis, \f{a}_n$.
\end{definition}	

\noindent Observe that when $k=1$, the $k$-GCD reduces to the standard GCD of ideals.

\begin{lemma}
	Fix positive integers $k$ and $N$. Let $Q_{\f{p}^k}(N)$ the number of ordered $n$-tuples of nonzero ideals in a number ring $\mathcal{O}$, in which each entry has norm at most $N$ such that $\f{p}^k$ does not divide the product of these entries. Then, we have
	$$Q_{\f{p}^k}(N) = \sum_{a_1+...+a_n < k} \Big[\prod_{j=1}^n \Big(H\bigg( \frac{N}{\f{N}(\f{p})^{a_j}}\bigg) - H\bigg(\frac{N}{\f{N}(\f{p})^{a_j + 1}}\bigg)\Big)\Big].$$
\end{lemma}

\begin{proof}
	Since the number of nonzero ideals with norm at most $N$ that are divisible by $\f{N}(\f{p})^j$ (for some nonzero prime ideal $\f{p}$ and positive integer $j$) equals $H\big(\frac{N}{\f{N}(\f{p})^j}\big)$, it follows that the number of nonzero prime ideals at most $N$ where $\f{N}(\f{p})^j$ \textit{exactly} divides the nonzero ideal is equal to
	$$H\bigg(\frac{N}{\f{N}(\f{p})^j}\bigg) - H\bigg(\frac{N}{\f{N}(\f{p})^{j+1}}\bigg).$$

   % \todo[inline, color=red]{NORM of $p^j$ Divides the ideal} Yes!
	
	\noindent Next, the number of ordered $n$-tuples $(m_1, ..., m_n)$ of nonzero ideals in which entry is at most $N$ satisfying the conditions $\f{p}^{a_j} \mid\mid m_j$ for each $j = 1, ..., n$ is given by the quantity
	$$\prod_{j=1}^n H\bigg(\frac{N}{\f{N}(\f{p})^j}\bigg) - H\bigg(\frac{N}{\f{N}(\f{p})^{j+1}}\bigg).$$
	
	\noindent Therefore, we conclude that $Q_{\f{p}^k}(N)$ is given as follows:
	$$Q_{\f{p}^k}(N) = \sum_{a_1+...+a_n < k} \bigg[\prod_{j=1}^n H\bigg(\frac{N}{\f{N}(\f{p})^j}\bigg) - H\bigg(\frac{N}{\f{N}(\f{p})^{j+1}}\bigg)\bigg].$$ 
\end{proof}

We now give a less cumbersome representation for $Q_{p^k}(N)$ by estimating the error from removing the ideal counting functions from its expression.

\begin{lemma}
	Using the same notation as in the previous lemma, we have the following estimate:
	$$Q_{\f{p}^k}(N) = H(N)^n\Big(1 - \frac{1}{\f{N}(\f{p})}\Big)^n \Big(1 + \frac{{}_nH_1}{\f{N}(\f{p})} + ... + \frac{{}_nH_{k-1}}{\f{N}(\f{p})^{k-1}}\Big) + O(1).$$
\end{lemma}

\begin{proof}
	Applying Marcus' estimate to the result from the previous lemma, for some $c,\epsilon>0$, we obtain
	\begin{align*} 
		Q_{\f{p}^k}(N) &= \sum_{a_1+...+a_n < k} H(N)^n\Big[\prod_{j=1}^n \frac{1}{\f{N}(\f{p})^j}-\frac{1}{\f{N}(\f{p})^{j+1}}+O\bigg(\frac{1}{N^\epsilon}\bigg)\Big]\\ 
		&= H(N)^n\Big(1 - \frac{1}{\f{N}(\f{p})}\Big)^n \sum_{a_1+...+a_n < k} \frac{1}{\f{N}(\f{p})^{a_1 + ... + a_n}} + O(1). 
	\end{align*} 
	\noindent Next for each non-negative integer less than $k$, there are \small${}_nH_j = \binom{n-1+j}{j}\ $ \normalsize  solutions in non-negative integers to the equation $a_1+...+a_n = j$. Hence we conclude that
	$$Q_{\f{p}^k}(N) = H(N)^n\Big(1 - \frac{1}{\f{N}(\f{p})}\Big)^n \Big(1 + \frac{{}_nH_1}{\f{N}(\f{p})} + ... + \frac{{}_nH_{k-1}}{\f{N}(\f{p})^{k-1}}\Big) + O(1),$$
	as required.
\end{proof}

Now, we are ready to state and prove the main theorem of this section.

\begin{theorem}
	Fix positive integers $B$ and $N$, and let $\f{p}_1, \f{p}_2, ...$ be the prime ideals in a number ring $\mathcal{O}$ with norm greater than $B$ in increasing order. Then, the probability of the $k$-GCD of products of random nonzero ideals in a number ring $\mathcal{O}$ is not divisible by the first $l$ prime ideals with norm greater than $B$ is given by 
	$$\prod_{i=1}^{l} \Big[1 - \Big(1 - \Big(1-\frac{1}{\mathfrak{N}(\f{p}_i)}\Big)^n \Big(1+\frac{{}_nH_1}{\f{N}(\f{p}_i)}+\mathellipsis+\frac{{}_nH_{k-1}}{\f{N}(\f{p}_i)^{k-1}}\Big)\Big)^m\Big].$$
\end{theorem}

\begin{proof}
	Letting $Q_{\f{p}^k}(N)$ denote the number of $n$ non zero ideals each of which have norm at most $N$ that are not divisible by $\f{p}^k$ (as in the previous two lemmas), there are $H(N)^n - Q_{\f{p}^k}(N)$ products of $n$ nonzero ideals each of which has norm at most $N$ that are divisible by $\f{p}^k$. Therefore, there are 
	$$H(N)^{mn} - (H(N)^n - Q_{\f{p}^k}(N))^m$$
	ordered $m$-tuples of products of $n$ nonzero ideals each of which is at most $N$ that are not divisible by $\f{p}^k$.
	
	Hence, the probability that an ordered $m$-tuples of products of $n$ nonzero ideals each of which has norm at most $N$ that are not divisible by $\f{p}^k$ is equal to
	$$\frac{H(N)^{mn} - (H(N)^n - Q_{\f{p}^k}(N))^m}{H(N)^{mn}} = 1 - \Big(1 - \frac{Q_{\f{p}^k}(N)}{H(N)^n}\Big)^m.$$
	
	Now, we use the result of the previous lemma to rewrite the latter probability as follows:
	$$1 - \Big[1 - \Big(1 - \frac{1}{\f{N}(\f{p})}\Big)^n \Big(1 + \frac{{}_nH_1}{\f{N}(\f{p})} + ... + \frac{{}_nH_{k-1}}{\f{N}(\f{p})^{k-1}}\Big)\Big]^m+O\bigg(\frac{1}{H(N)^{nm}}\bigg).$$
	Then, since divisibility by finitely many nonzero prime ideals give independent events, we deduce that the probability that an ordered $m$-tuples of products of $n$ nonzero ideals each of which is at most $N$ that are not divisible by $\f{p}_1^k,\mathellipsis,\f{p}_l^k$ is equal to
	\begin{align*} &\prod_{i=1}^l \Bigg[1 - \Big[1 - \Big(1 - \frac{1}{\f{N}(\f{p}_i)}\Big)^n \Big(1 + \frac{{}_nH_1}{\f{N}(\f{p}_i)} + ... + \frac{{}_nH_{k-1}}{\f{N}(\f{p}_i)^{k-1}}\Big)\Big]^m + O\bigg(\frac{1}{H(N)^{nm}}\bigg) \Bigg]\\
		&=\prod_{i=1}^l \Big[1 - \Big(1 -\Big(1-\frac{1}{\f{N}(\f{p}_i)}\Big)^n\Big(1+\frac{{}_nH_1}{\f{N}(\f{p}_i)}+\mathellipsis+\frac{{}_nH_{k-1}}{\f{N}(\f{p}_i)^{k-1}}\Big)\Big)^m\Big] + O\Big(\frac{1}{H(N)^{mn}}\Big). \end{align*}
	
	\noindent Letting $N \to \infty$ now gives us the desired result.
\end{proof}

We have so far found the probability that the $k$-GCD of products of random nonzero ideals in a number ring $\mathcal{O}$ is not divisible by the first \(l\) nonzero prime ideals greater than \(B\). We now use this result to find the probability that the $k$-GCD of products of random nonzero ideals is not divisible by \textit{all} non zero prime ideals with norm greater than $B$, thereby giving the probability that this $k$-GCD is $B$-smooth.

\begin{theorem}
Suppose that each $\f{a}_{ij}$ is randomly, independently, and uniformly chosen from $\mathcal{O}$. The probability that $\gcd_k(\prod_{j=1}^{n}\f{a}_{ij},\mathellipsis,\prod_{j=1}^{n}\f{a}_{mj})$ is $B$-smooth converges as $N\to\infty$ to $$\prod_{\f{N}(\f{p})>B} \Big[1 - \Big(1 -\Big(1-\frac{1}{\f{N}(\f{p})}\Big)^n\Big(1+\frac{{}_nH_1}{\f{N}(\f{p})}+\mathellipsis+\frac{{}_nH_{k-1}}{\f{N}(\f{p})^{k-1}}\Big)\Big)^m\Big].$$
\end{theorem}

\begin{proof}
Letting $P(l, N)$ denote the probability in the previous theorem (and for convenience setting $P(0, N) = 1$, we define $g_N(l) = P(l-1, n) - P(l, n)$. Then, we see that $g_N(l)$ is precisely the probability that $\gcd_k(\prod_{j=1}^n \f{a}_{1j}, ... , \prod_{j=1}^n \f{a}_{mj})$ is coprime to $\f{p}_1, \mathellipsis, \f{p}_{l-1}$ and divisible by $\f{p}_l$ for uniformly and independently chosen $\f{a}_{ij}$'s from $\{1, 2, ..., N\}$. In particular, note that $g_N(l)$ is non-negative. We claim that we can move the limit to infinity past the summation sign:
$$\lim_{N \to \infty} \sum_{s=1}^{\infty} g_N(s) = \sum_{s=1}^{\infty} \lim_{N \to \infty} g_N(s).$$
In order to accomplish this, we invoke the Dominated Convergence Theorem (for series). To this end, we need to show that $g_N(l)$ is bounded above by $g(l) = \frac{n^m}{\f{N}(\f{p}_{l})^{m}}$ and $\sum_{l=1}^{\infty} g(l)$ converges.

We start by showing that $g_N(l)$ is bounded. First of all, observe that
$$g_N(l) \leq \text{Pr}\Big(\f{p}_l^k \Big| \gcd\Big(\prod_{j=1}^n \f{a}_{1j}, \mathellipsis, \prod_{j=1}^n \f{a}_{mj}\Big)\Big).$$
Computing the numerator to the probability on the last line, we find that
\begin{align*} 
	\Big|\{(\f{a}_{ij}) : \f{p}_l^k \mid \prod_{j=1}^n \f{a}_{1j}\}\Big|^m &= \frac{(H(N)^n - |\{(\f{a}_{1j}) : \f{p}_l^k \nmid \prod_{j=1}^n \f{a}_{1j}\}|)^m}{H(N)^{mn}}\\
	&= \frac{(H(N)^n - |\{\f{a}_{11} : \f{p}_l^k \nmid \f{a}_{11}\}|^n)^m}{H(N)^{mn}}. \end{align*}
Therefore, it follows that
\begin{align*} 
	g_N(l) &\leq \frac{(H(N)^n - |\{\f{a}_{11} : \f{p}_l^k \nmid \f{a}_{11}\}|^n)^m}{H(N)^{mn}}\\
	&\leq \Big[1 - \Big(1 - \frac{1}{H(N)}  H\Big(\frac{N}{\f{N}(\f{p}_l)^k}\Big)\Big)^n\Big]^m\\
	&\leq \Big[1 - \Big(1 - \Big(\frac{1}{\f{N}(\f{p}_l)^k}+\frac{A}{N^{\varepsilon}}\Big)\Big)^n\Big]^m \text{ for some } A>0\\
	&\leq \Big(1 - \Big(1 - \frac{2}{\f{N}(\f{p}_l)^k}\Big)^n \Big)^m,
\end{align*}
the latter expression is bounded above by $g(l)=\frac{(2n)^m}{\f{N}(\f{p}_l)^{km}}$ from Bernoulli's inequality in the form $1-(1-x)^t\leq xt$ for $t\geq 1$ and $0\leq x \leq 1$.

 Moreover $\sum_{l=1}^{\infty} g(l)$ converges, because we can bound this series from above with a constant multiple of the Dedekind Zeta function (with $m\geq 2$) $$ \sum_{l=1}^{\infty}g(l)\leq(2n)^m\sum_{l=1}^{\infty}\frac{1}{\f{N}(\f{p}_l)^{km}}\leq (2n)^m\zeta_O(km)< \infty.$$

Having satisfied the hypotheses of the Dominated Convergence Theorem, we first observe that since $\sum_{s=1}^l g_N(s)$ is a telescoping sum, we obtain
$$\sum_{s=1}^l g_N(s) = \sum_{s=1}^l (P(s-1, n) - P(s, n)) = 1 - P(l, N),$$ 
and thus
$$\lim_{N \to \infty} \sum_{s=1}^{\infty} g_N(s) = 1 - \prod_{\f{N}(\f{p}) > B} \Big(1 - \Big[1 - \Big(1-\frac{1}{\f{N}(\f{p}_i)}\Big)^n \Big(1 + \frac{{}_nH_1}{\f{N}(\f{p})} + \mathellipsis + \frac{{}_nH_{k-1}}{\f{N}(\f{p})^{k-1}}\Big)\Big]^m\Big).$$

\noindent Then since $\displaystyle \sum_{s=1}^{\infty} \lim_{N \to \infty} g_N(s)$ represents the complement of the probability we wanted to compute, the Dominated Convergence Theorem yields the desired assertion.
\end{proof}

\newpage
\section{Appendix}
\subsection{Probability that the GCD of Products of Positive Integers is \texorpdfstring{$B$}{\mathbb{B}}-smooth Using the Inclusion-Exclusion Principle}
For purposes of comparison to the previous proof, we now show Cheong and Kim's original proof via the inclusion-exclusion principle in detail.
\begin{theorem}
	Fix positive integers $B$ and $N$, and let $p_1, p_2, ...$ be the primes greater than $B$ in increasing order. If $X_l = \{p_1, p_2, ..., p_l\}$, then
	$$T(l,N)=\sum_{P\subset X_l}{(-1)^{|P|}} \Big[\sum_{Q\subset P}(-1)^{|Q|} \Big(\sum_{R\subset Q}(-1)^{|R|}\Big\lfloor\frac{N}{\prod_{p\in R} p}\Big\rfloor\Big)^n\Big]^m.$$
\end{theorem}

\begin{proof}
By hypothesis, we want to find the value of
$$T(l,N) = \Big|\Big\{(r_{ij}) : \gcd\Big(\prod_{j=1}^nr_{1j}, ..., \prod_{j=1}^{n} r_{mj}\Big) \text{ is coprime to } p\in X_l \Big\}\Big|.$$
	
In order to accomplish this, we use the Inclusion-Exclusion principle to enumerate the number of ordered pairs that are not in a subset containing elements not coprime to some $p\in X_l$. Hence, 
	\begin{align*}
		T(l, N) &= \sum_{P\subset X_l}{(-1)^{|P|}}\Bigg|\Big\{(r_{ij}):\prod_{p\in P} p \, \Big| \gcd\Big(\prod_{j=1}^nr_{1j}, ... ,\prod_{j=1}^{n} r_{mj}\Big) \Big\} \Bigg|\\ &= \sum_{P\subset X_l}{(-1)^{|P|}} \Bigg|\Big\{(r_{ij}):\prod_{p\in P} p \, \Big| \prod_{j=1}^nr_{1j}\Big\}\Bigg|^m.
	\end{align*}
	
	\noindent Again, applying the Inclusion-Exclusion principle yields
	\begin{align*}
		\Bigg|\Big\{(r_{ij}):\prod_{p\in P} p \, \Big|\prod_{j=1}^nr_{1j}\Big\}\Bigg| &=\sum_{Q\subset P}(-1)^{|Q|}\Big|\Big\{(r_{1j}): p \nmid \prod_{j=1}^nr_{1j} \text{ for all } p\in Q \Big\}\Big|\\
		&=\sum_{Q\subset P}(-1)^{|Q|}\Biggl(\sum_{R\subset Q}(-1)^{|R|}\Biggl\lfloor\frac{N}{\prod_{p\in R} p}\Biggr\rfloor\Biggr)^n.
	\end{align*}
	
	\noindent Therefore, we conclude that
	$$T(l,N)=\sum_{P\subset X_l}{(-1)^{|P|}}\Bigg[\sum_{Q\subset P}(-1)^{|Q|}\Big(\sum_{R\subset Q}(-1)^{|R|}\Big\lfloor\frac{N}{\prod_{p\in R} p}\Big\rfloor\Big)^n\Bigg]^m.$$
\end{proof}

\subsection{Probability that the \texorpdfstring{$k$}{\mathbb{k}}-GCD of Products of Positive Integers is \texorpdfstring{$B$}{\mathbb{B}}-smooth using the Inclusion-Exclusion Principle}

\begin{theorem} Let $p$ be a prime number and $r_{ij}$ be nonzero integers for each $1 \leq i \leq m$ and $1 \leq j \leq n$. Then
	$p \mid \gcd_k(\prod_{j=1}^{n}r_{1j},\mathellipsis,\prod_{j=1}^n r_{mj})$ if and only if $p^k \mid \prod_{j}r_{ij}$ for some $i$.
\end{theorem}

\begin{proof}
	Suppose that $p \mid a$ where $a = \gcd_k(\prod_{j=1}^n r_{1j}, ...,\prod_{j=1}^n r_{mj})$. This is true if and only if $p \mid a$. Then this is equivalent to $p^k \mid a^k$ as well as $p^k \mid \prod_j r_{ij}$ for some $i$.
\end{proof}

\begin{theorem}
	Let \(p_1, p_2,\mathellipsis\) be the prime numbers larger than \(B\) in increasing order. Then, we have
	$$\lim_{N\to\infty} \frac{T_k(l,n)}{N^{mn}}=\prod_{i=1}^l\Big(1-\Big[1-\Big(1-\frac{1}{p_i}\Big)^n\Big(1+\frac{nH_1}{p_i}+\mathellipsis+\frac{nH_{k-1}}{p_i^{k-1}}\Big)\Big]\Big).$$
\end{theorem}

\begin{proof}
	Let \(X_l=\{p_1,\mathellipsis,p_l\}\) and \(1\leq r_{ij}\leq N\). Using the Inclusion-Exclusion principle, 
$$\frac{T_k(l,n)}{N^{nm}} = \sum_{P\subset X_l}(-1)^{|P|}\Big(\sum_{Q\subset P}\Pr\Big[p^k\nmid \prod_{j=1}^{n}r_{1j} \; \text{ for all } p\in Q \Big]\Big)^m.$$
	
	Let \(p^a \mid\mid x\) denote \(p^a\mid x\) and \(p^{a+1}\nmid x\) and let \(a_p,j\in Q\) and \(1\leq j \leq n\).
	
    \begin{remark}
		Suppose a tuple satisfies \(a_{p,1}+... +a_{p,n}<k\) with \(a_{p,1}+\mathellipsis+a_{p,n}+1=k\), then using the exactly divides symbol \(||\) guarantees \(p^k \nmid x\).
    \end{remark} 
	
Thus,
        \begin{align*}
		\Pr\Big[p^k\nmid \prod_{j=1}^{n}r_{1j},\:\forall p\in Q\Big] &= \sum_{a_{p,1}+\mathellipsis+a_{p,n}<l}\Pr\Big[p^{a_{p,j}} \mid\mid r_{1j},\: \text{ for all } p,j\Big]\\
		&=\sum_{a_{p,1}+\mathellipsis+a_{p,n}<k}\quad\prod_{j=1}^{n}\Pr\Big[p^{a_{p,j}} \mid\mid r_{1j} \text{ for all } p\in Q\Big],  
	\end{align*}
	
	\noindent where the last term utilizes the product rule for probability. This allows us to count the innermost quantity. 
	
	Using the inclusion-exclusion principle, 
	\begin{align*}
		&|\{(r_{1j}): p^{a_{{p,j}}}\mid\mid r_{1j} \: \text{ for all } p\in Q\}|\\ &= \Big\lfloor\frac{N}{\prod_{p\in Q}p^{a_{p,j}}}\Big\rfloor-\sum_{p\in Q}\Big\lfloor \frac{N}{\prod_{p\in Q}p^{a_{p,j}}}\Big\rfloor+\mathellipsis+(-1)^{|Q|}\Big\lfloor\frac{N}{\prod_{p\in Q}p^{a_{p,j}+1}}\Big\rfloor.
	\end{align*}
	
	Next, we form an estimate of the quantity by taking the difference of the last two terms. We find that
	\begin{align*}
		\Big\lfloor \frac{N}{\prod_{p\in Q} p^{a_{p,j}}} \Big\rfloor - \Big\lfloor \frac{N}{\prod_{p\in Q} p^{a_{p,j}+1}} \Big\rfloor &=
		\frac{N}{\prod_{p\in Q} p^{a_{p,j}}} -  \frac{N}{\prod_{p\in Q} p^{a_{p,j}+1}} + O(1)\\&=
		N\prod_ {p\in Q}\Big[ \frac{1}{ p^{a_{p,j}}}  -  \frac{1}{ p^{a_{p,j}+1}} \Big]+ O(1).
	\end{align*}
	
	Putting this all together, we obtain
	\begin{align*}
	\Pr\Big[p^k\nmid \prod_{j=1}^{n}r_{1j},\: \forall p\in Q\Big] &= \prod_{p\in Q} \Big(\sum_{a_{p,1}+\mathellipsis+a_{p,n}<k}\prod_{j=1}^{n}\frac{p-1}{p^{a_{p,j}+1}}\Big)+O\Big(\frac{1}{N}\Big)\\
		&=\prod_{p\in Q}\Big[\Big(1-\frac{1}{p}\Big)^n\sum_{a_{p,1}+\mathellipsis+a_{p,n}<k}\frac{1}{p^{a_{p,1}+\mathellipsis+a_{p,n}}}\Big]+O\Big(\frac{1}{N}\Big).
	\end{align*}

	Next, we count every composition $a_{p,1}+ ... +a_{p,n}<k$ for $p^{a_{p,j}}$ that divides the quantity. By using combinations with repetition, the number of $n$-tuples of positive integers which are a solution to \(a_{p,1}+\mathellipsis+a_{p,n}=i\) is given by $_nH_i = \binom{n+i-1}{i}$. Applying this result yields
	\begin{align*}
		&\prod_{p\in Q}\Big[\Big(1-\frac{1}{p}\Big)^n\Big[\sum_{a_{p,1}+\mathellipsis+a_{p,n}<k}\frac{1}{p^{a_{p,1}+\mathellipsis+a_{p,n}}}\Biggr]+O\Big(\frac{1}{N}\Big)\\
		&= \prod_{p\in Q}\Big[\Big(1-\frac{1}{p}\Big)^n\Big(1+\frac{_nH_1}{p}+\mathellipsis+\frac{_nH_{k-1}}{p^{k-1}}\Big)\Big]+O\Big(\frac{1}{N}\Big).
	\end{align*}
	
	Finally, substitute these results into the original expression and simplify. Letting $N\to \infty$, we see that the error approaches 0 as desired.
\end{proof}

% I prefer \gcd_k; it looks nicer!

\newpage

%\section{References}
\begin{thebibliography}{1}

\bibitem{Apostol} T. M. Apostol, Introduction to Analytic Number Theory, Springer, 1976.

\bibitem{Bai} S. Bai, Polynomial Selection for the Number Field Sieve. 2011. Australian National University, PhD dissertation. \url{https://maths-people.anu.edu.au/~brent/pd/Bai-thesis.pdf} %

\bibitem{Benkoski} S. J. Benkoski. The probability that $k$ positive integers are relatively $r$-prime, Journal of Number Theory, \textbf{8}: 218–223, 1976.

\bibitem{Cheon} J.H. Cheon and D. Kim, Probability that the $k$-GCD of products of positive integers is $B$-smooth, Journal of Number Theory, \textbf{168}: 72-80, 2016.

\bibitem{Dirichlet} P.G.L. Dirichlet, \"{U}ber die Bestimmung der mittleren Werthe in der Zahlentheorie, Abhandlungen der K\"{o}niglich Preussischen Akademie der Wissenchaften, \textbf{2}: 69-83, 1849.

\bibitem{Dixon} J.D. Dixon, Asymptotically Fast Factorization of Integers, Mathematics of Computation, \textbf{36}: 255-260, 1981.

\bibitem{Dummit} D. Dummit and R. Foote, Abstract Algebra, Prentice Hall, 1991.

\bibitem{Granville} A. Granville, Smooth numbers: Computational number theory and beyond, Algorithmic Number Theory, \textbf{44}: 267-323, 2008.

\bibitem{Lehmer} D. N. Lehmer, Asymptotic evaluation of certain totient sums, American Journal of Mathematics, \textbf{22}: 293–335, 1900.

\bibitem{Marcus} D.A. Marcus, Number Fields. 2nd Edition, Springer, 2018. %

\bibitem{Mollin} R.A. Mollin, A Brief History of Factoring and Primality Testing B.C. (Before Computers), Mathematics Magazine, \textbf{75}: 18-29, 2002.

\bibitem{Naccache} D. Naccache and I. Shparlinski, Divisibility, Smoothness and Cryptographic Applications,
\url{https://eprint.iacr.org/2008/437.pdf}. %

\bibitem{Nymann} J. E. Nymann, On the Probability that $k$ Positive Integers are Relatively Prime, Journal of Number Theory, \textbf{4}: 469-473, 1972.

\bibitem{Pomerance0} C.Pomerance, A Tale of Two Sieves, Notices of the American Mathematical Society, \textbf{36}: 1473-1485, 1996. %

\bibitem{Pomerance1} C. Pomerance, The Quadratic Sieve Factoring Algorithm, Advances in Cryptology, Proceedings of Eurocrypt’84, \textbf{44}: 169-182, 1985. %

\bibitem{Pomerance2} C. Pomerance, Smooth numbers and the quadratic sieve, Algorithmic Number Theory, \textbf{44}: 69-81, 2008. %

\bibitem{Shiu} P. Shiu, Fermat’s Method of Factorisation, Mathematical Gazette, \textbf{99}: 97–103, 2015. %

\bibitem{Sittinger} B. Sittinger, The probability that random algebraic integers are relatively $r$-prime, Journal of Number Theory, \textbf{130} (1): 164-171, 2010.

%Cite a proper combinatorics/discrete math textbook instead of wikipedia, for stars and bars and possibly inclusion exclusion; not completely necessary. Kenneth Rosen's Discrete Mathematics and its Applications has your information in it.

\end{thebibliography}

\end{document}