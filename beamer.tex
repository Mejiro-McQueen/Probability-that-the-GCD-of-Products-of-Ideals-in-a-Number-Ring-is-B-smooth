\documentclass[11pt,xcolor={dvipsnames}]{beamer}
\usepackage[utf8]{inputenc}
\usepackage[T1]{fontenc}
\usepackage{lmodern}
\usetheme{Copenhagen}
\usepackage{bm}
\newcommand{\f}[1]{\mathfrak{#1}}
\author{Adrian Vazquez}

%\setbeameroption{show notes}

\makeatother
\setbeamertemplate{footline}
{
  \leavevmode%
  \hbox{%
  \begin{beamercolorbox}[wd=.15\paperwidth,ht=2.25ex,dp=1ex,center]{author in head/foot}%
    \usebeamerfont{author in head/foot}\insertshortauthor
  \end{beamercolorbox}%
  \begin{beamercolorbox}[wd=.75\paperwidth,ht=2.25ex,dp=1ex,center]{title in head/foot}%
    \usebeamerfont{title in head/foot}\insertshorttitle
  \end{beamercolorbox}%
  \begin{beamercolorbox}[wd=.1\paperwidth,ht=2.25ex,dp=1ex,center]{date in head/foot}%
    \insertframenumber{} / \inserttotalframenumber\hspace*{1ex} 
  \end{beamercolorbox}}%
  \vskip0pt%
}
\makeatletter
%\setbeamertemplate{navigation symbols}{}


\begin{document}
	%\author{Adrian Vazquez}
	\title[Probability that the GCD of Products of Ideals is $B$-smooth]{The Probability that the GCD of Products of Ideals in a Number Ring is $B$-smooth}
	%\subtitle{}
	%\logo{}
	%\institute{}
	%\date{}
	%\subject{}
	%\setbeamercovered{transparent}
	%\setbeamertemplate{navigation symbols}{}
	\begin{frame}[plain]
		\maketitle
	\end{frame}

	\begin{frame}
		\frametitle{Introduction}
		\begin{definition}[Relatively $r$-prime]
			For a fixed integer $r>1$, we say that the integers $m_1, m_2, \mathellipsis, m_k$ are \textbf{relatively $r$-prime} if they have no common factor of the form $n^r$ for any integer $n > 1$. %When $r = 1$, this is the definition of being relatively prime.
		\end{definition}
		\begin{theorem}[Benkoski's Theorem]
			Fix $k$, $r\in \mathbb{N}$ not both equal to 1, and let $q(n)$ denote the number of ordered $k$-tuples of positive integers less than or equal to $n$ that are relatively $r$-prime. Then, $\lim\limits_{n\to\infty} \frac{q(n) }{n^k}=\frac{1}{\zeta(rk)}$.
		\end{theorem}
		\begin{enumerate}
                \item Sittinger provided a more concise proof and extension to algebraic integers.
                \item Apply techniques for more concise GCD proofs.
                \item Extend GCD proofs to algebraic integers.
		\end{enumerate}
	\end{frame}

	\begin{frame}
		\frametitle{B-Smooth (B-Friable)}
		\begin{definition} Let $B$ be a fixed positive integer. A positive integer $x$ is \textbf{\bm{$B$}-smooth} (\textbf{\bm{$B$}-friable}) if $x$ has no prime divisor greater than $B$. 
		\end{definition}
		\begin{example}
			Let $B = 20$. Then $57$ is $20$-smooth, because $57 = 2 \cdot 19$ and both 2 and 19 are less than 20.\newline
			
			However $46$ is not $20$-smooth, because $46= 2 \cdot 23$ and 23 is a prime number greater than 20.
		\end{example}
		\note{B-Friable is the newer name for the definition, B-Smooth slightly older, oldest cryptography papers don't mentioned either.}
	\end{frame}


	\begin{frame}
		\begin{theorem}[Divisibility by first $l$ primes]
			Fix positive integers $B$ and $N$, and let $p_1, p_2, ...$ be the primes greater than $B$ in increasing order. The probability of the GCD of products of random positive integers is not divisible by the first $l$ primes greater than $B$ is given by
			$$\prod_{i=1}^{l} \Big[1 - \Big(1 - \Big(1-\frac{1}{p_i}\Big)^n\Big)^m\Big].$$
		\end{theorem}
			\begin{proof}
		\begin{enumerate}
			\item Use counting arguments to construct:
			$\frac{N^{mn} - \Big(N^n - \Big(N - \Big\lfloor\frac{N}{p}\Big\rfloor\Big)^n\Big)^m}{N^{mn}}$
			\item Apply the estimate $\frac{1}{n} \Big\lfloor \frac{n}{k} \Big\rfloor = \frac{1}{k} + O\Big(\frac{1}{n}\Big)$.
			\item Selecting a prime from a finite set of primes is an independent event.
			\item Take limit $N \to \infty$. 
		\end{enumerate}
		\end{proof}
	\end{frame}

	\begin{frame}
		\begin{theorem}
			Suppose that each $r_{ij}$ is chosen uniformly and independently from $\{1, 2, ..., N\}$. Then, the probability that $\gcd(\prod_{j=1}^n r_{1j}, ... , \prod_{j=1}^n r_{mj})$ is $B$-smooth converges (as $N \to \infty$) to
			$$\prod_{p>B} \Big[1 - \Big(1 - \Big(1 - \frac{1}{p}\Big)^n \Big)^m \Big].$$
		\end{theorem}
		\begin{proof}
			\begin{enumerate}
				\item Bounded above by a convergent $p$-series.
                \item Use Lebesgue Dominated Convergence Theorem for Series.
				\item This yields $ 1 - \prod_{p > B} \Big[1 - \Big(1 - \Big(1 - \frac{1}{p_i}  \Big)^n\Big)^m\Big].$
			\end{enumerate}
		\end{proof}
	\end{frame}

	\begin{frame}
		\begin{definition}[$k$-GCD] Fix a positive integer $k$. The \textbf{$k$-GCD} of $n$ nonzero integers $x_1, ..., x_n$, written $k\text{-}\gcd(x_1,x_2, ...,x_n)$, is the largest integer whose $k^{\text{th}}$ power divides each of $x_1, x_2, ..., x_n$.
		\end{definition}	
		
		\begin{alertblock}{Remark}
			As $k$ varies, it is possible that the $k$-GCD also varies.\newline
                When $k=1$, the $k$-GCD reduces to the usual GCD. 
		\end{alertblock}
	\end{frame}

	\begin{frame}
		\begin{theorem}
			Fix positive integers $B$ and $N$, and let $p_1, p_2, ...$ be the primes greater than $B$ in increasing order. The probability of the $k$-GCD of products of random integers is not divisible by the first $l$ primes greater than $B$ is given by 
			$$\prod_{i=1}^{l} \Big[1 - \Big(1 - \Big(1-\frac{1}{p_i}\Big)^n \textcolor{PineGreen}{\Big(1 + \frac{{}_nH_1}{p} + ... + \frac{{}_nH_{k-1}}{p^{k-1}}\Big)}\Big)^m\Big].$$
		\end{theorem}
	
		\begin{proof}
			\begin{enumerate}
				\item Similar to previous case.
				\item Use
				$\Big\lfloor\frac{N}{p^j}\Big\rfloor - \Big\lfloor\frac{N}{p^{j+1}}\Big\rfloor$ to count integers divisible exactly by $p^j$.
				\item Applying stars and bars yields the \textcolor{PineGreen}{green product}.
			\end{enumerate}
		\end{proof}
	\end{frame}

	\begin{frame}
		\begin{theorem} Suppose that each $r_{ij}$ is chosen uniformly and independently from $\{1, 2, ..., N\}$. Then, the probability that $k\text{-}\gcd(\prod_{j=1}^n r_{1j}, ... , \prod_{j=1}^n r_{mj})$ is $B$-smooth converges (as $N \to \infty$) to
			$$\prod_{p>B} \Big[1 - \Big(1 - \Big(1-\frac{1}{p_i}\Big)^n \Big(1 + \frac{{}_nH_1}{p} + ... + \frac{{}_nH_{k-1}}{p^{k-1}}\Big)\Big)^m\Big].$$
		\end{theorem}
	
		\begin{proof}
			Nearly identical to GCD case.
		\end{proof}
	\end{frame}

	\begin{frame}
	\begin{theorem}
		Fix positive integers $B$ and $N$ and a number ring $\mathcal{O}$. Let $\mathfrak{p}_1, \mathfrak{p}_2, ...$ be the prime ideals in $\mathcal{O}$ with norm greater than $B$ arranged in non-decreasing order by norm. Then, the probability of the GCD of products of random ideals is not divisible by the first $l$ prime ideals greater than $B$ is given by 
		$$\prod_{i=1}^{l} \Big[1 - \Big(1 - \Big(1-\frac{1}{\mathfrak{N}(\mathfrak{p}_i)}\Big)^n\Big)^m\Big].$$
	\end{theorem}
	
	\begin{proof}
		\begin{enumerate}
			\item Use counting argument: $\frac{H(N)^{nm}-\Big[H(N)^n - \Big(H(N) - H\Big(\frac{N}{\f{N(p)}}\Big)\Big)^n\Big]^m}{H(N)^{nm}}$
			\item Apply estimate: $\frac{1}{H(n)} H\Big(\frac{n}{k}\Big) = \frac{1}{k} + O\Big(\frac{1}{n^{\epsilon}}\Big).$
			\item Selecting a prime from a finite set of primes is an independent event.
			\item Take limit $N \to \infty$.
		\end{enumerate}
	\end{proof}
\end{frame}

\begin{frame}
	\begin{theorem}
		The probability that the GCD of products of random ideals of algebraic integers in a given number ring $\mathcal{O}$ is $B$-smooth converges to
		$$\prod_{\f{N}(\f{p})>B}\Big[1 - \Big(1 - \Big(1 - \frac{1}{\f{N}(\f{p})}\Big)^n \Big)^m \Big].$$
	\end{theorem}
	\begin{proof}
		\begin{enumerate}
			\item Start with previous theorem.
                \item Bounded above by constant multiple of Dedekind zeta function.
			\item Apply Lebesgue Dominated Convergence Theorem for Series.
				\item This yields $1 - \prod_{\mathfrak{N}(\mathfrak{p}) > B} \Big[1 - \Big(1 - \Big(1 - \frac{1}{\mathfrak{N}(\mathfrak{p})}  \Big)^n\Big)^m\Big].$
		\end{enumerate}
	\end{proof}
\end{frame}

\begin{frame}
	\begin{theorem}
		Fix positive integers $B$ and $N$, and let $\f{p}_1, \f{p}_2, ...$ be the prime ideals in a number ring $\mathcal{O}$ with norm greater than $B$ in increasing order. Then, the probability of the $k$-gcd of products of random nonzero ideals in a number ring $\mathcal{O}$ is not divisible by the first $l$ prime ideals with norm greater than $B$ is given by 
		$$\prod_{i=1}^{l} \Big[1 - \Big(1 - \Big(1-\frac{1}{\mathfrak{N}(\f{p}_i)}\Big)^n \textcolor{PineGreen}{\Big(1+\frac{{}_nH_1}{\f{N}(\f{p}_i)}+\mathellipsis+\frac{{}_nH_{k-1}}{\f{N}(\f{p}_i)^{k-1}}\Big)}\Big)^m\Big].$$
	\end{theorem}
	
	\begin{proof}
		\begin{enumerate}
			\item Similar to 1-GCD case.
			\item Use $H\bigg(\frac{N}{\f{N}(\f{p})^j}\bigg) - H\bigg(\frac{N}{\f{N}(\f{p})^{j+1}}\bigg)$ to count the ideals where $\f{N}(\f{p}^j)$ divides the norm of the ideal exactly.
			\item Apply stars and bars to yield \textcolor{PineGreen}{green product.}
		\end{enumerate}
	\end{proof}
\end{frame}

\begin{frame}
	\begin{theorem}
		\frametitle{GCD of Random Algebraic Integers is B-smooth}
		Suppose that each $\f{a}_{ij}$ is randomly, independently, and uniformly chosen from $\mathcal{O}$. The probability that $\gcd_k(\prod_{j=1}^{n}\f{a}_{ij},\mathellipsis,\prod_{j=1}^{n}\f{a}_{mj})$ is $B$-smooth converges as $N\to\infty$ to $$\prod_{\f{N}(\f{p})>B} \Big[1 - \Big(1 -\Big(1-\frac{1}{\f{N}(\f{p})}\Big)^n\Big(1+\frac{{}_nH_1}{\f{N}(\f{p})}+\mathellipsis+\frac{{}_nH_{k-1}}{\f{N}(\f{p})^{k-1}}\Big)\Big)^m\Big].$$
	\end{theorem}
	
	\begin{proof}
		Similar to 1-GCD case. 
	\end{proof}
	
	\begin{alertblock}{Remarks}
		\begin{enumerate}
			\item Set $\mathcal{O}=\mathbb{Z}$ to yield case for integers.
			\item Set $k=1$ to yield 1-GCD case.
			\item Alternative proof using the Inclusion-Exclusion principle.
		\end{enumerate}
	\end{alertblock}
\end{frame}

\begin{frame}
	\frametitle{B-smooth Numbers in Integer Factorization}
	B-smooth numbers play a critical role in \textbf{integer factorization algorithms}.
	\begin{enumerate}
		\item Dixon's Random Squares (1981)
		\item Pomerance's Quadratic Number Sieve (1983)
		\item Pollard's General Number Sieve (1996)
	\end{enumerate}

	\begin{alertblock}{Remark}
		There is no known efficient non-quantum integer factoring algorithm.
	\end{alertblock}
	\note{Also, can be exploited to crack weak crypto, strong hashing, primality testing, ECF}
\end{frame}

\begin{frame}
	\frametitle{Encryption and the Integer Factorization Problem}
	How are \textbf{encryption} and integer factorization algorithms related?
	\begin{enumerate}
		\item Some encryption algorithms rely on the difficulty of solving the integer factorization problem (e.g. \textbf{RSA}).
		\item Integer factorization algorithms help determine if the difficulty of the problem has been reduced.
	\end{enumerate}
\end{frame}

\begin{frame}
	\frametitle{RSA Challenge}
	\begin{enumerate}
		\item A challenge issued by RSA Laboratories in 1991 to encourage research in computational number theory.
	
		\item The following 130 digit (430 bits) RSA number  has two prime factors (Solved using Quadratic Number Sieve).
	\end{enumerate}
	\begin{example}
		 \begin{align*}
			\text{RSA-}130\text{: } &18070820886874048059516561644059055662\\
			&781025167694013491701270214500566625402440
		\end{align*}
	\end{example}

	\note{15\% Faster than the previous record holder would have taken.}
	\note{MIPS = Million Instructions Per Second, MIPS-Year is standard way of measuring performance in Cryptography research.}
\end{frame}

	\begin{frame}
		\frametitle{How hard is encryption today?}
		\begin{enumerate}
			\item RSA-250 (829 bits, 250 digits) was factored in 2020 using the Quadratic Number Sieve.
			\item The recommended RSA key size is 3072 bits (925 digits), but 4096 bit (1233 digits) keys are also common.
			\item Quantum computers have not reached the required complexity to solve the RSA problem.
		\end{enumerate}
	\begin{alertblock}{Key Take Away}
		 Knowledge of B-Smooth numbers $\iff$ Efficient Factorization Algorithms $\iff$ Stronger encryption algorithms! 
	\end{alertblock}
	\end{frame}
\end{document}