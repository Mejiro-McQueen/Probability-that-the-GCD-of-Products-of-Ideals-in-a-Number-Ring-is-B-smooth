\documentclass[10pt,a4paper]{article}

\usepackage[utf8]{inputenc}
\usepackage[T1]{fontenc}
\usepackage{amsmath}
\usepackage{amsfonts}
\usepackage{amssymb}
\usepackage{graphicx}
\usepackage{amsthm}
\usepackage[pdfencoding=unicode, colorlinks=true, psdextra]{hyperref}
\usepackage[capitalize]{cleveref}
\usepackage{bookmark}
\usepackage{todonotes}
\usepackage{biblatex}
\usepackage{xcolor}
%\addbibresource{thesis.bib}

\newtheorem{theorem}{Theorem}[section]
\newtheorem{lemma}[theorem]{Lemma}
\newtheorem{corollary}{Corollary}[theorem]
\newtheorem{example}{Example}[theorem]
\theoremstyle{definition}
\newtheorem{definition}{Definition}[section]
\theoremstyle{remark}
\newtheorem{remark}{Remark}
\newtheorem{note}{Note}[section]

\newcommand{\f}[1]{\mathfrak{#1}} 

\begin{document}
	\tableofcontents
	\section{TODO}
	\listoftodos
	\section{Background definitions and Lemmas}
	In this section we state various definitions and lemmas that are common to most of the theorems discussed.
\begin{lemma}\label{alternating-sum-to-prod}
Let $Q = \mathcal{P}(x_1, ..., x_n)$ be the power set of $\{x_1, x_2, ..., x_n\}$. Then, we have
$$\prod_{i=1}^n (1-x_i)=\sum_{R\subset Q}(-1)^{|R|}\prod_{x\in R}x.$$  
\end{lemma}

\begin{proof}
Observe the pattern that emerges from applying the distributive property to the factors from $\prod_{i=1}^n (1-x_i)$ one at a time:
\begin{align*} 
&\prod_{i=1}^n (1-x_i)\\ 
&= (1-x_1)\prod_{i=2}^n(1-x_i)\\
&= (1-(x_1+x_2) + x_1x_2)\prod_{i=3}^n (1-x_1)\\
&=(1-(x_1+x_2+x_3)+(x_1x_2+x_1x_3+x_2x_3)- x_1x_2x_3)\prod_{i=4}^n (1-x_i). 
\end{align*}
Hence, we can express the product above as follows:
\begin{equation}\label{power-sum}
\prod_{i=1}^n (1-x_i) = 1 - \sum_{i}x_i+\sum_{i<j}x_ix_j - \sum_{i<j<k}x_ix_jx_k+\mathellipsis + (-1)^n x_1...x_n.
\end{equation}
Letting $Q$ denote the power set of $\{x_1, ..., x_n\}$, we see that we can express the sum in \cref{power-sum} as
$$\prod_{i=1}(1-x_i)=\sum_{R\subset Q}(-1)^{|R|}\prod_{x\in R}x.$$   
\end{proof}

\begin{definition} \label{riemann-zeta}
The \textbf{Riemann Zeta function} is defined by 
$$\zeta(s)=\sum_{j=1}^{\infty} \frac{1}{j^s}=\prod_{p \text{ prime}}(1-p^{-s})^{-1}.$$
\end{definition}	
	
\begin{definition} \label{mobius}
The \textbf{M\"obius function} $\mu: \mathbb{Z} \to \{0, \pm 1\}$ is defined by
$$\mu(n) =\begin{cases} 
\phantom{-}1      & \text{if }  n=\pm 1 \\
\phantom{-}0      & \text{if }  n \text{ is not square-free}\\
(-1)^r & \text{if }  n=p_1 p_2 ... p_r. 		\end{cases}$$		
\end{definition}

\begin{lemma}\label{mobious-riemann} 
For any $\operatorname{Re} \, s > 1$, we have
$$\sum_{j=1}^{\infty} \frac{\mu(j)}{j^s}=\prod_{p \textrm{ prime}} (1-p^{-s})=\frac{1}{\zeta(s)}.$$
\end{lemma}

\begin{proof}
Let $p_1, ..., p_k$ be the first $k$ prime numbers in increasing order. Then applying \cref{alternating-sum-to-prod}, we obtain
$$\prod_{p \leq p_k} (1-p^{-s})=\sum_{R\subset Q}(-1)^{|R|}\prod_{x\in R}x,$$ 
where the product is taken over all primes less than or equal to $p_k$ and $Q = \mathcal{P}(p_1,p_2,\mathellipsis,p_k)$.
		
		For any term, we have \(R=\{p_1,p_2,\mathellipsis, p_r\}\). Then \((-1)^{|R|} = 1\) whenever \(r\) is even, and \(-1\) whenever \(r\) is odd.
		Notice that these two cases coincide exactly with the last case of the M\"obius function.
					
		Thus, we can substitute the sign of the term with \(\mu(j)\) where \(j=\prod_{x\in R}x=p_1p_2\mathellipsis p_r\) and we can substitute the product of the term, which is given by \((p_1p_2\mathellipsis p_r)^{-s}\) with \(\frac{1}{j^{s}}\). 
	\end{proof}

\begin{definition} The \textbf{floor function} $f: \mathbb{R} \to \mathbb{Z}$ is given by $f(x) = \lfloor x \rfloor$, where $\lfloor x \rfloor = n$ for some fixed integer $n$ if and only if $n\leq x < n+1$.
\end{definition}

\begin{remark}
The definition for the floor function many equivalent forms, this form being the one we apply for ensuing derivations.
\end{remark}

\begin{lemma}(Floor Annihilators)\label{floor-annihilators}
Given a positive rational $\frac{a}{b}$, we have \(\lfloor \frac{a}{b}\rfloor = 0\) if and only if \(a<b\). 
\end{lemma}
\begin{proof}
Suppose \(\lfloor \frac{a}{b} \rfloor = 0\). Then \(n=0\) and from the definition of floor, \(0\leq \frac{a}{b} < 0+1\). Then solving for \(a\), \(0\leq a < b\).
	
Next, suppose \(a<b\), then working in reverse, \(\frac{a}{b}<1\)  and so \(0\leq \frac{a}{b} < 1\), which implies \(\lfloor \frac{a}{b} \rfloor = 0\). 
\end{proof}

\begin{definition} For a fixed integer \(r\geq 1\), we say that the integers \(m_1,m_2,\mathellipsis,m_k\) are \textbf{relatively \(r\)-prime} if and only if they have no common factor of the form \(n^r\) for \(n>1\). 
\end{definition}

% \begin{remark}
% 	An ordered \(k\)-tuple of integers is relatively \(r\)-prime if and only if there exists no prime \(p\) such that \(p^r\) divides each \(k\) integer.
% \end{remark}
% \begin{proof}
% 	Suppose \(M=(m_1,m_2,\mathellipsis,m_k)\) is a \(k\)-tuple whose elements are relatively \(r\)-prime. Then for a fixed \(r\), and any integer \(n\), we have \(n^r\nmid m_i\) for any \(m_i\in M\). Next, \(n\) is an integer and has a unique prime factorization, \(p_1^{\alpha_1}p_2^{\alpha_2}p_3^{\alpha_3}\mathellipsis p_n^{\alpha_n}\). 
	
% 	Thus, \(n^r=(p_1^{\alpha_1})^r(p_2^{\alpha_2})^r(p_3^{\alpha_3})^r\nmid m_i\) for all \(m_i\in M\). 
	
% 	Next, suppose there exists no prime \(p\) such that \(p^r\) divides each \(k\) integer in a \(k\)-tuple called \(M\). By prime factorization, we can construct any arbitrary integer \(n\). 
	
% 	Then, \((p_1^{\alpha_1})^r(p_2^{\alpha_2})^r(p_3^{\alpha_3})^r=(p_1^{\alpha_1}p_2^{\alpha_2}p_3^{\alpha_3})^r=n^r\nmid m_i\) for all \(m_i\in M\).
% \end{proof}

We apply the following identity when deriving Benkoski's Theorem below.

\begin{lemma} \label{difference-powers} For any real numbers $A$ and $B$,
$$A^k-B^k=(A-B)(A^{k-1}+A^{k-1}B+A^{k-1}B^2+ ... + B^{k-1}).$$
\end{lemma}

\begin{proof}
Applying the distributive property shows that all terms except the first and last terms cancel in pairs.
\end{proof}

\section{Probability that \texorpdfstring{\(k\)}{\mathbb{k}} random integers are relatively \texorpdfstring{\(r\)}{\mathbb{r}}-prime}
\subsection{Benkoski's Theorem}

\subsubsection{Background Arithmetic}
We list simple results below that we utilize to prove Benkoski's theorem. 
\begin{lemma}\label{ordered-k-tuples}
	 There are \(|S|^k\) ordered \(k\)-tuples that can be constructed from a set \(S\).
\end{lemma}
\begin{proof}
	Using the fundamental counting principle: We have \(|S|\) choices for each of the \(k\) tuple elements, for a total of \(|S|^k\) tuples. 
\end{proof}

\begin{lemma}
	There are \(n^k\) ordered \(k\)-tuples that can be constructed from a set \(S\) where each element of \(S\) is less than or equal to \(n\).
\end{lemma}
\begin{proof}
	Applying \cref{ordered-k-tuples} to the set $S=\{1,2,3,\mathellipsis,n\}$ implies that there are $n^k$ possible ordered $k$-tuples can be constructed.
\end{proof}

\begin{lemma}\label{less-than-n-divisible-by-m}
	Let \(m,n\) be positive integers. There are \(\lfloor\frac{n}{m}\rfloor\) integers less than \(n\) that are divisible by \(m\).
\end{lemma}

\begin{proof}
	Let \(n,m\) be positive integers.
	
	Case 1: \(n < m\). Then no integers are divisible by \(m\) and by \cref{floor-annihilators}, \(\lfloor\frac{n}{m}\rfloor = 0\) holds.
	
	Case 2: \(n=m\). Then only one integer \(m=n\) is divisible by \(m\) and \(\lfloor\frac{n}{m}\rfloor=\lfloor\frac{n}{n}\rfloor=1 \) holds.
	
	Case 3: \(n>m\). Let \(S=\{1m,2m,3m,\mathellipsis\}\), that is, all the multiples of \(m\). Then there must be some number \(am\) that is less than or equal to \(n\) with \(am+m>n\). 
	
	Thus we have, \(S=\{1m,2m,3m,\mathellipsis ,am\}\cup\{am+m,am+2m,\mathellipsis\}\). 
	
	By construction, \(|\{1m,2m,3m,\mathellipsis ,am\}|=a\), and all elements divisible by \(m\) and are less than or equal to \(n\).
	
	Next, \(am\leq n<am+m\), implying \(a\leq \frac{n}{m}<a+1\), satisfying the definition for floor. Thus \(\lfloor \frac{m}{n}\rfloor=a\) and the hypothesis holds. 
	
	Finally, we will now show that \(\lfloor \frac{m}{n}\rfloor\) can never yield an integer greater than \(a\). We have \(0\leq n < am+m\), which implies \(0\leq \frac{n}{m}<a+1\). Then to preserve the floor function, we must have \(a=0\) and \(\lfloor\frac{n}{m}\rfloor=0\), which is not the case.
\end{proof}

\subsubsection{Counting \texorpdfstring{\(q(n)\)}{\mathbb{q(n)}}}

Let $q(n)$ be the number of ordered \(k\)-tuples of positive integers less than or equal to $n$ that are relatively $r$-prime. We start by using the Inclusion-Exclusion principle to find a formula for the actual value of $q(n)$.

Fix $n\in \mathbb{N}$, and let $a_1=n^k$ be the cardinality of the set of all ordered $k$-tuples of positive integers at most than $n$. Next, let $a_2$ be the cardinality of the set of all $k$-tuples of elements at most $n$ that are not $r$-prime. 
	
	If a number is not \(r\)-prime, then it is divisible by \(p^r\) for at least one prime \(p\). Then, how many positive integers are less than \(n\) and divisible by \(p^r\)? By \cref{less-than-n-divisible-by-m}, \(\lfloor \frac{n}{p^r}\rfloor\). If an ordered \(k\)-tuple is not \(r\)-prime, then at least one of the tuple elements must be divisible by \(p^r\). Then by the fundamental counting principle, we can construct a non \(r\)-prime \(k\)-tuple as follows: For each \(k\) elements, choose \(\lfloor \frac{n}{p^r}\rfloor\) non \(r\)-prime elements. Thus we have \(\lfloor \frac{n}{p^r}\rfloor^k\) ordered \(k\)-tuples that are not \(r\)-prime. 
	
	At this point, we still do not know how to choose \(p\) that will allow us to count all of \(a_2\). However, we can index a sum over all primes \(p\). Thus we have  \(a_2=\sum_{p_{1}^r}\left\lfloor \frac{n}{p_{1}^r}\right\rfloor^k\).
	Notice that since $n$ is fixed, at some point \(n<p_i\), and the given sum trivially converges as all terms with $n > p_1^r$ equal zero.
	For the remaining terms we can continue the pattern required by the inclusion exclusion principle, for example, the next term, \(a_3\), would add the cardinality of the set of all ordered \(k\)-tuples with elements less than \(n\) that are not \(r\)-prime, this time indexing over the product of the primes from the previous terms as well.
	
	Thus, by the inclusion exclusion principle, we may write (where $p,q,r$ are primes)
	
		\begin{align}
		q(n) &= n^k-\sum_p \left\lfloor \frac{n}{p^r}\right\rfloor^k + \sum_{p < q}\left\lfloor \frac{n}{(pq)^r}\right\rfloor^k - \sum_{p < q < r}\left\lfloor \frac{n}{(pqr)^r}\right\rfloor^k +...\\ & =\sum_{i=1}(-1)^{i+1}\left(\sum_{j= \prod_{l=1}^{i-1}p_l}\left\lfloor\frac{n}{j^r}\right\rfloor^k\right).
	\end{align}
\begin{remark}
By convention, the null product equals the multiplicative identity, if it exists. In our case, \(\prod_{1}^{0}=1\).
\end{remark}

We can rewrite this more compactly using the M\"obius function.
\begin{lemma}[M\"obius form of \(q(n)\)]
	\begin{equation}
			q(n)=\sum_{i=1}(-1)^{i+1}\left(\sum_{j= \prod_{l=1}^{i-1}p_l}\left\lfloor\frac{n}{j^r}\right\rfloor^k\right)=\sum_{j=1}^{\infty}\mu(j)\left\lfloor\frac{n}{j^r}\right\rfloor^k
	\end{equation}
\end{lemma}
\begin{proof}
	The the value of \(i\) in the outer sum determines the value of \(j=\prod_{l=1}^{i-1}p_l\) in the inner sum. \newline
	Case 1: \(i\) is even implies \(m\) is even: Then the outer sum will be positive and \(j\) will be a product of an even number of factors.\newline
	Case 2: \(i\) is odd implies \(m\) is odd: Then the outer sum will be negative and \(j\) will be a product of an odd number of factors. \newline
	Case 3: \(j\) is always square free since its factors are unique primes. 
	
	These cases are satisfied by the M\"obius function.

	This allows us to substitute the outer sum for an infinite sum if we include the M\"obius function as a factor, yielding:
	
	\begin{equation}\label{infinite-mobius}
		q(n)=\sum_{j=1}^{\infty}\mu(j)\left\lfloor\frac{n}{j^r}\right\rfloor^k
	\end{equation}

	The M\"obius function provides the alternating sums for cases 1 and 2. 
	
	In regards to case 3, we no longer need to worry about the inner sum's product of primes. Suppose \(j\) is not a product of square free primes, then \(\mu(j)=0\) and the entire term is 0. Otherwise, if the prime factorization of \(j\) is square free, then \(\mu(j)=\pm1\) and the entire term is non zero.
\end{proof}

	A further simplification of \cref{infinite-mobius} can be found by applying \cref{floor-annihilators}, which removes the infinite sum by limiting the sum up to the floor annihilators.
	
\begin{corollary}
If $n, j, r \in \mathbb{N}$, then $\lfloor\frac{n}{j^r}\rfloor = 0$  whenever $j>\lfloor \sqrt[r]{n}\rfloor$.
\end{corollary}

\begin{proof}
Suppose that $j>\lfloor \sqrt[r]{n}\rfloor$. Then, we see that $n < j^r$ and the desired claim directly follows from \cref{floor-annihilators}.
\end{proof}
	
Applying this corollary, we have
\begin{equation}\label{finite-mobius}
		q(n)=\sum_{j=1}^{\sqrt[r]{n}}\mu(j)\left\lfloor\frac{n}{j^r}\right\rfloor^k.
\end{equation}
	
\subsection{Proof of Benkoski's Theorem}
\begin{theorem}[Benkoski's Theorem]
	Fix \(l,r\in \mathbb{N}\) not both equal to 1, and let \(q(n)\) denote the number of ordered \(k\)-tuples of positive integers less than or equal to \(n\) that are relatively \(r\)-prime. Then, \begin{equation}
		\lim\limits_{n\to\infty} \frac{q(n)}{n^k} = \frac{1}{\zeta(rk)}.
	\end{equation} 
\end{theorem}
\begin{proof}
	Given \begin{equation}
		q(n) = \sum_{j=1}^{\lfloor\sqrt[r]{n}\rfloor} \mu(j)\left\lfloor \frac{n}{j^r} \right\rfloor^k,
	\end{equation} we want to find an estimate that does not contain the floor function. 
	
	Using the definition of floor, we find the relationship \(0\leq x - \lfloor x \rfloor < 1\).
	This suggests that \(x\) and \(\lfloor x \rfloor\) are relatively close enough that we can perform a substitution if we account for a growth estimate. 
	
	Then using \cref{difference-powers}, 
	\begin{align}
		&\Bigl\lfloor \frac{n}{j^r}\Bigr\rfloor^k -\left(\frac{n}{j^r}\right)^k
		\\&=\left(\Bigl\lfloor \frac{n}{j^r}\Bigr\rfloor^k- \left(\frac{n}{j^r}\right)^k\right)\left(\Bigl\lfloor\frac{n}{j^r}\Bigr\rfloor^{k-1}+\Bigl\lfloor\frac{n}{j^r}\Bigr\rfloor^{k-2}\left(\frac{n}{j^r}\right)+\mathellipsis+\left(\frac{n}{j^r}\right)^{k-1}\right).
	\end{align}
	To find the growth estimate, using Big O notation:
		\begin{align}
		&O\left(\Bigl\lfloor \frac{n}{j^r}\Bigr\rfloor^k -\left(\frac{n}{j^r}\right)^k \right)
		\\&=O\left(\Bigl\lfloor \frac{n}{j^r}\Bigr\rfloor^k- \left(\frac{n}{j^r}\right)^k\right)O\left(\Bigl\lfloor\frac{n}{j^r}\Bigr\rfloor^{k-1}+\Bigl\lfloor\frac{n}{j^r}\Bigr\rfloor^{k-2}\left(\frac{n}{j^r}\right)+\mathellipsis+\left(\frac{n}{j^r}\right)^{k-1}\right).
	\end{align}
	Since the first factor \(\lfloor \frac{n}{j^r}\rfloor^k- (\frac{n}{j^r})^k\) is at most 1, it can be treated as a constant. For the right most factor, \(\lfloor x \rfloor^mx^n\leq x^{m+n}\), and so the fastest growing term is  \(\left(\frac{n}{j^r}\right)^{k-1}\). Thus, we obtain $\lfloor \frac{n}{j^r}\rfloor^k - (\frac{n}{j^r})^k = O((\frac{n}{j^r})^{k-1})$. 
	
	Then performing the substitution yields
	\begin{multline}
	 q(n)=\sum_{j=1}^{\lfloor\sqrt[r]{n}\rfloor}\mu(j)\left(\frac{n}{j^r}\right)^k+\sum_{j=1}^{\lfloor\sqrt[r]{n}\rfloor}\mu(j)O\left(\left(\frac{n}{j^r}\right)^{k-1}\right)\\
	 =\sum_{j=1}^{\lfloor\sqrt[r]{n}\rfloor}\mu(j)\left(\frac{n}{j^r}\right)^k+O\left(\sum_{j=1}^{\lfloor\sqrt[r]{n}\rfloor}|\mu(j)|\left(\frac{n}{j^r}\right)^{k-1}\right).  \label{two_estimates}
	\end{multline}
	
	We can apply some more Big O algebraic manipulation to remove constant terms.
	
	\begin{align}
	O\left(\sum_{j=1}^{\lfloor\sqrt[r]{n}\rfloor}|\mu(j)|\left(\frac{n}{j^r}\right)^{k-1}\right) 
	& = O\left(n^{k-1}\sum_{j=1}^{\lfloor\sqrt[r]{n}\rfloor}|\mu(j)|\left(\frac{1}{j^{r(k-1)}}\right)\right) \notag\\
	& = n^{k-1} \cdot O\left(\sum_{j=1}^{\lfloor\sqrt[r]{n}\rfloor} \frac{1}{j^{r(k-1)}}\right)\notag. \label{estimate_second}
	\end{align}

	Next, we need to find out how fast each term in the summation grows.
	By applying a Riemann integral:
	\begin{align}
		O\left(\sum_{j=1}^{\lfloor\sqrt[r]{n}\rfloor}\left(\frac{1}{j^{r(k-1)}}\right)\right)&\leq O\left(1+\int_{1}^{\sqrt[r]{n}}\frac{dx}{x^{r(k-1)}}\right) \notag \\
		&= \begin{cases}
			O(\ln n) \text{ if } r=1 \text{ and } k=2\\
			O(n^{\frac{1}{r}}) \text{ if } r\leq 2 \text{ and } k=1\\
			O(1) \text{ otherwise.}
		\end{cases}
	\end{align} 
	Then substituting this estimate into \cref{estimate_second}, we see that 
	
	\begin{align}
	O\left(n^{k-1}\right)O\left(\sum_{j=1}^{\lfloor\sqrt[r]{n}\rfloor}\left(\frac{1}{j^{r(k-1)}}\right)\right) &= \begin{cases}
	O(n\ln n) & \text{if } r=1 \text{ and } k=2\\
	O(n^{\frac{1}{r}}) & \text{if } r\leq 2 \text{ and } k=1\\
	O(n^{k-1}) & \text{otherwise.}
	\end{cases}
	\end{align}

	Now we want to find and estimate for the principal term (left term) of \cref{two_estimates}. By applying  \cref{mobious-riemann}, we see that
	\begin{align}\label{riemann-zeta-with-correction}
		\sum_{j=1}^{\lfloor\sqrt[r]{n}\rfloor}\mu(j)\left(\frac{n}{j^r}\right)^k &= \frac{n^k}{\zeta(rk)} - n^k\sum_{\lfloor \sqrt[r]{n} \rfloor + 1}^\infty \frac{\mu(j)}{j^{rk}}.
	\end{align}
	Then on \cref{riemann-zeta-with-correction} we can estimate the rightmost sum by applying another Riemann integral. 
	\begin{align}
		O\left(n^k\sum_{\sqrt[r]{n}+1}^\infty \frac{\mu(j)}{j^{rk}}\right) &= O(n^k)O\left(\sum_{\sqrt[r]{n}+1}^\infty \frac{\mu(j)}{j^{rk}}\right)\notag\\
		&= O(n^k)O\left(\int_{\sqrt[r]{n}}^{\infty}\frac{\mu(j)}{j^{rk}}\right)\notag\\
		&=O(n^k)O(n^{{1}/(r-k)})\notag\\
		&=O(n^{{1}/r}).
	\end{align} 

	Putting all the estimates together, 
	\begin{align}
		q(n) &= \frac{n^k}{\zeta(rk)}+O(n^{{1}/r})+ \begin{cases}
			O(n\ln n) \text{ if } r=1 \text{ and } k=2\\
			O(n^{\frac{1}{r}}) \text{ if } r\leq 2 \text{ and } k=1\\
			O(n^{k-1}) \text{ otherwise}\end{cases}\notag\\
		&= \frac{n^k}{\zeta(rk)}+ \begin{cases}
			O(n\ln n) \text{ if } r=1 \text{ and } k=2\\
			O(n^{\frac{1}{r}}) \text{ if } r\leq 2 \text{ and } k=1\\
			O(n^{k-1}) \text{ otherwise.}\end{cases}
	\end{align}
	Finally, since \(\lim\limits_{n\to\infty} n^k = \infty\) and \(\lim\limits_{n\to\infty} q(n)=\infty\), we can apply L'Hopital's rule:
	
	\begin{equation}
		\lim\limits_{n\to\infty} \frac{q(n)}{n^k} = \frac{1}{\zeta(rk)}.
	\end{equation}
\end{proof}

\section{Probability that the GCD of products of positive integers is  \texorpdfstring{\(B\)}{\mathbb{B}}-Friable}

\subsection{Main content}

\begin{definition} Let $B$ be a fixed positive integer. A positive integer $x$ is \textbf{$B$-friable ($B$-smooth)} if $x$ has no prime divisor greater than $B$. 
\end{definition}

\noindent \textbf{Example:} Let $B = 20$. Then $57$ is $20$-friable, because $57 = 2 \cdot 19$ and both 2 and 19 are less than 20. However, $46$ is not $20$-friable, because $46= 2 \cdot 23$ and 23 is a prime number greater than 20.

\vspace{.1 in}

For the remainder of this section, we are going to find the probability that the greatest common divisor (GCD) of $m$ products of $n$ positive integers (all randomly chosen in a uniform and independent manner) is $B$-friable.

\vspace{.1 in}

To this end, we fix a positive integer $N$. Then, we randomly choose uniformly and independently integers $r_{ij} \in [1, N]$ for each integer $1 \leq i \leq m$ and $1 \leq j \leq n$ where $m \geq 2$ and $n \geq 1$. Our goal is to compute the probability that an ordered $m$-tuple $(r_{ij})$satisfies the following condition:
$$\gcd\Big(\prod_{j=1}^nr_{1j}, ..., \prod_{j=1}^n r_{mj}\Big)$$
is relatively prime to the first $l$ primes greater than $B$. 

Before stating the main theorem for this section, we establish the following useful lemma that we use in the ensuing derivation below.

\begin{lemma} If $n$ and $k$ are positive integers, then
$$\frac{1}{n} \Big\lfloor \frac{n}{k} \Big\rfloor = \frac{1}{k} + O\Big(\frac{1}{n}\Big).$$
\end{lemma}

\begin{proof}
Since we know by definition of the floor function that $0 \leq x - \lfloor x \rfloor < 1$ for any real number $x$, we can write
$$0 \leq \frac{n}{k} - \Big\lfloor \frac{n}{k} \Big\rfloor < 1.$$
Then, rearranging this inequality and dividing all three parts by $n$ yields
$$0 \leq \frac{1}{k} - \frac{1}{n}\Big\lfloor \frac{n}{k}\Big\rfloor < \frac{1}{n}.$$
The claim now directly follows.
\end{proof}

Now, we are ready to state and prove the main theorem of this section.

\begin{theorem}
Fix positive integers $B$ and $N$, and let $p_1, p_2, ...$ be the primes greater than $B$ in increasing order. Then, the probability of the GCD of products of random positive integers is not divisible by the first $l$ primes greater than $B$ is given by \label{probability-random-first-l}
$$\prod_{i=1}^{l} \Big[1 - \Big(1 - \Big(1-\frac{1}{p_i}\Big)^n\Big)^m\Big].$$
\end{theorem}

\begin{proof}
We first find the probability that 
$$\gcd\Big(\prod_{j=1}^nr_{1j}, ..., \prod_{j=1}^n r_{mj}\Big)$$
is coprime to the first $l$ primes greater than $B$.  To this end, note that for any prime $p$, there are $(N - \lfloor\frac{N}{p}\rfloor)^n$ products of $n$ positive integers each of which is at most $N$ that are not divisible by $p$. Hence, there are $N^n - (N - \lfloor\frac{N}{p}\rfloor)^n$ products of $n$ positive integers each of which is at most $N$ that are divisible by $p$. Therefore, there are 
$$N^{mn} - \Big(N^n - \Big(N - \Big\lfloor\frac{N}{p}\Big\rfloor\Big)^n\Big)^m$$
$m$-tuples of products of $n$ positive integers each of which is at most $N$ that are not divisible by $p$.

Hence, the probability that an ordered $m$-tuple of products of $n$ positive integers each of which is at most $N$ are not divisible by $p$ is equal to
$$\frac{N^{mn} - \Big(N^n - \Big(N - \Big\lfloor\frac{N}{p}\Big\rfloor\Big)^n\Big)^m}{N^{mn}} = 1 - \Big(1 - \Big(1 - \frac{1}{N} \Big\lfloor\frac{N}{p}\Big\rfloor\Big)^n\Big)^m.$$

Then, since divisibility by finitely many primes yields independent events, the probability that the GCD of $m$ products of random positive integers at most $N$ are not divisible by the first $l$ primes greater than $B$ is equal to
$$\prod_{i=1}^l \Big[1 - \Big(1 - \Big(1 - \frac{1}{N}\Big\lfloor\frac{N}{p_i}\Big\rfloor\Big)^n\Big)^m\Big].$$

Since we ultimately want to let $N \to \infty$, we now use the above Lemma to estimate the error from removing the floor function from our probability statement. This gives us
\begin{align*}& \prod_{i=1}^l \Big[1 - \Big(1 - \Big(1 - \Big(\frac{1}{p_i} + O\Big(\frac{1}{N}\Big)\Big) \Big)^n\Big)^m\Big]\\
&= \prod_{i=1}^l \Big[1 - \Big(1 - \Big(1 - \frac{1}{p_i}  \Big)^n\Big)^m\Big] + O\Big(\frac{1}{N^{nm}}\Big).\end{align*}

\noindent Finally letting $N \to \infty$ now gives us the desired result.
\end{proof}

So far, we have found the probability that the GCD of products of random integers is not divisible by the first \(l\) primes greater than \(B\). We now use this result to find the probability that the GCD of products of random integers is not divisible by \textit{all} primes greater than $B$, thereby giving the probability that this GCD is $B$-friable.

\begin{theorem} Suppose that each $r_{ij}$ is chosen uniformly and independently from $\{1, 2, ..., N\}$. Then, the probability that $\gcd(\prod_{j=1}^n r_{1j}, ... , \prod_{j=1}^n r_{mj})$ is $B$-friable converges (as $N \to \infty$) to
$$\prod_{p>B} \Big[1 - \Big(1 - \Big(1 - \frac{1}{p}\Big)^n \Big)^m \Big].$$
\end{theorem}

\begin{proof}
Letting $P(l, N)$ denote the probability in the previous theorem (and for convenience setting $P(0, N) = 1$, we define $g_N(l) = P(l-1, n) - P(l, n)$. Then, we see that $g_N(l)$ is precisely the probability that $\gcd(\prod_{j=1}^n r_{1j}, ... , \prod_{j=1}^n r_{mj})$ is coprime to $p_1, ..., p_{l-1}$ and divisible by $p_l$ for uniformly and independently chosen $r_{ij}$'s from $\{1, 2, ..., N\}$. In particular, note that $g_N(l)$ is non-negative. We claim that we can move the limit to infinity past the summation sign:
$$\lim_{N \to \infty} \sum_{k=1}^{\infty} g_N(k) = \sum_{k=1}^{\infty} \lim_{N \to \infty} g_N(k).$$
In order to accomplish this, we invoke the Dominated Convergence Theorem (for series). To this end, we need to show that $g_N(l)$ is bounded above by $g(l) = \frac{n^m}{p_l^m}$ and $\sum_{l=1}^{\infty} g(l)$ converges.

We start by showing that $g_N(l)$ is bounded. First of all, observe that
$$g_N(l) \leq \text{Pr}\Big(p_l \Big| \gcd\Big(\prod_{j=1}^n r_{1j}, ... , \prod_{j=1}^n r_{mj}\Big)\Big).$$
Computing the numerator to the probability on the last line, we find that
\begin{align*} 
\Big|\{(r_{ij}) : p_l \mid \prod_{j=1}^n r_{1j}\Big|^m &= \frac{(N^n - |\{(r_{1j}) : p_l \nmid \prod_{j=1}^n r_{1j}\}|)^m}{N^{mn}}\\
&= \frac{(N^n - |\{r_{11} : p_l \nmid r_{11}\}|^n)^m}{N^{mn}}. \end{align*}
Therefore, it follows that
\begin{align*} 
g_N(l) &\leq \frac{(N^n - |\{r_{11} : p_l \nmid r_{11}\}|^n)^m}{N^{mn}}\\
&= \Big[1 - \Big(1 - \frac{1}{N} \Big\lfloor \frac{N}{p_l}\Big\rfloor\Big)^n\Big]^m\\
&\leq \Big[1 - \Big(1 - \frac{1}{p_l}\Big)^n\Big]^m\\
&\leq \frac{n^m}{p_l^m},\end{align*}
where the last inequality directly follows from Bernoulli's inequality. Moreover $\sum_{l=1}^{\infty} g(l)$ converges, because we can bound this series from above with a convergent $p$-series (noting that $m \geq 2$):
$$\sum_{l=1}^{\infty} g(l) = n^m \sum_{l=1}^{\infty} \frac{1}{p_l^m} < n^m \sum_{j=1}^{\infty} \frac{1}{j^m}.$$

Having satisfied the hypotheses of the Dominated Convergence Theorem, we first observe that since $\sum_{k=1}^l g_N(k)$ is a telescoping sum, we obtain
$$\sum_{k=1}^l g_N(k) = \sum_{k=1}^l (P(k-1, n) - P(k, n)) = 1 - P(l, N),$$ 
and thus
$$\lim_{N \to \infty} \sum_{k=1}^{\infty} g_N(k) = 1 - \prod_{p > B} \Big[1 - \Big(1 - \Big(1 - \frac{1}{p_i}  \Big)^n\Big)^m\Big].$$

\noindent Then since $\displaystyle \sum_{k=1}^{\infty} \lim_{N \to \infty} g_N(k)$ represents the complement of the probability we wanted to compute, the Dominated Convergence Theorem yields the desired assertion.
\end{proof}

Having derived the probability that the GCD of products of randomly chosen positive integers is $B$-friable, we now find more convenient bounds for this product representation.

\begin{theorem} The probability that $\gcd(\prod_{j=1}^n r_{1j}, ..., \prod_{j=1}^{n}r_{mj})$ is $B$-friable is bounded above by 
$$\frac{1}{\zeta(m)}\prod_{p\leq B}\Big(1-\frac{1}{p^m}\Big)^{-1},$$
and is bounded below by 
$$\prod_{B<p\leq\hat{n}} \Big[1 - \Big(1 - \Big(1 - \frac{1}{p}\Big)^n \Big)^m\Big] \cdot \prod_{\hat{n}<p\leq\hat{r}} \Big(1 - \Big(\frac{n}{p}\Big)^m\Big) \cdot \frac{1}{\zeta(s)},$$
where $\hat{n}=\max\{n,B\}$, $\hat{r}=\max\{\hat{n}, \lfloor n^{\frac{m}{m-1}}+1\rfloor\}$, and $s = m(1 - \log_{\hat{r}}{n}) > 1$.
\end{theorem} 

\noindent \textbf{Remark:} It is understood that for the lower bound, the first finite product is equal to 1 if $B = \hat{n}$, and the second finite product is equal to 1 if $\hat{n}=\hat{r}$.
 
\begin{proof}
We start by deriving the upper bound. Since $(1 - \frac{1}{p})^n$ decreases as a function of $n$ due to $1 - \frac{1}{p} \in (0, 1)$, we obtain
$$\prod_{B<p\leq\hat{n}} \Big[1 - \Big(1 - \Big(1 - \frac{1}{p}\Big)^n \Big)^m \Big] \leq \prod_{p>B} \Big(1 - \frac{1}{p^m}\Big) = \frac{1}{\zeta(m)} \prod_{p\leq B} \Big(1 - \frac{1}{p^m}\Big)^{-1},$$
where the rightmost equality uses the infinite product representation of the Riemann Zeta function.

\vspace{.1 in}

Next, we derive the lower bound for our probabilistic expression. We start by splitting the product at $\hat{n}$. Then we apply Bernoulli's inequality in the form $(1 - x)^n \geq 1 - x$ where $x \in [0, 1]$ and $n \geq 1$, where we take $x = \frac{1}{p}$ where $p > \hat{n}$ is a prime number. This yields 
$$\prod_{B<p\leq\hat{n}} \Big[1 - \Big(1 - \Big(1 - \frac{1}{p}\Big)^n \Big)^m \Big] \geq \prod_{B<p\leq \hat{n}}\Big[1 - \Big(1 - \Big(1 - \frac{1}{p}\Big)^n \Big)^m \Big] \cdot \prod_{p>\hat{n}}\Big(1 - \Big(\frac{n}{p}\Big)^m\Big).$$

\vspace{.1 in}

It remains to find a lower bound for $\prod_{p>\hat{n}} (1 - (\frac{n}{p})^m)$. From using the definition of $s$, we see that $p \geq \hat{r}$ is equivalent to $(\frac{n}{p})^m \leq \frac{1}{p^s}$. Using this fact in conjunction to the infinite product representation for the Riemann zeta function, we obtain 
\begin{align*} \prod_{p>\hat{n}}\Big(1 - \Big(\frac{n}{p}\Big)^m\Big) &= \prod_{\hat{n}< p \leq \hat{r}}\Big(1 - \Big(\frac{n}{p}\Big)^m\Big) \cdot \prod_{p>\hat{r}}\Big(1 - \Big(\frac{n}{p}\Big)^m\Big)\\ &\geq  \prod_{\hat{n}< p \leq \hat{r}}\Big(1 - \Big(\frac{n}{p}\Big)^m\Big) \cdot \prod_{p>\hat{r}}\Big(1 - \frac{1}{p^s}\Big)\\ &\geq  \prod_{\hat{n}< p \leq \hat{r}}\Big(1 - \Big(\frac{n}{p}\Big)^m\Big) \cdot \frac{1}{\zeta(s)}. \end{align*}
The claimed lower bound now directly follows. 

\vspace{.1 in}

It remains to show that $s > 1$. To this end, observe that since $r=\lfloor n^{\frac{m}{m-1}}+1\rfloor$ and $\hat{r} = \max\{n, B, r\}$, we have $\hat{r} \geq r > n^{\frac{m}{m-1}}$. Hence, we conclude that $s = m(1 - \log_{\hat{r}}{n}) > 1$ as required.
\end{proof}
	 
%Remark 2.4 from paper goes here, blocked by derivation of $s$
%\todo[inline]{Search your feelings, you know \(s=m(1-\log_{\hat{r}}n)\) to be true.}

\subsection{Alternative proof with Inclusion-Exclusion}

\begin{theorem}
Fix positive integers $B$ and $N$, and let $p_1, p_2, ...$ be the primes greater than $B$ in increasing order. If $X_l = \{p_1, p_2, ..., p_l\}$, then
$$T(l,N)=\sum_{P\subset X_l}{(-1)^{|P|}} \Bigg[\sum_{Q\subset P}(-1)^{|Q|} \Big(\sum_{R\subset Q}(-1)^{|R|}\Big\lfloor\frac{N}{\prod_{p\in R} p}\Big\rfloor\Big)^n\Bigg]^m.$$
\end{theorem}

\begin{proof}

%\begin{note}
%\href{https://en.wikipedia.org/wiki/Inclusion%E2%80%93exclusion_principle#A_generalization}{Reference for Inclusion Exclusion Principle Generalization}
%\end{note}

By hypothesis, we want to find the value of
$$T(l,N) = \Big|\Big\{(r_{ij}) : \gcd\Big(\prod_{j=1}^nr_{1j}, ..., \prod_{j=1}^{n} r_{mj}\Big) \text{ is coprime to } p\in X_l \Big\}\Big|.$$

In order to accomplish this, we use the Inclusion-Exclusion principle to enumerate the number of ordered pairs that are not in a subset containing elements not coprime to some $p\in X_l$. Hence, 
\begin{align}
T(l, N) &= \sum_{P\subset X_l}{(-1)^{|P|}}\Bigg|\Big\{(r_{ij}):\prod_{p\in P} p \, \Big| \gcd\Big(\prod_{j=1}^nr_{1j}, ... ,\prod_{j=1}^{n} r_{mj}\Big) \Big\} \Bigg|\\ &= \sum_{P\subset X_l}{(-1)^{|P|}} \Bigg|\Big\{(r_{ij}):\prod_{p\in P} p \, \Big| \prod_{j=1}^nr_{1j}\Big\}\Bigg|^m.
\end{align}
	
\noindent Again, applying the Inclusion-Exclusion principle yields
\begin{align}
\Bigg|\Big\{(r_{ij}):\prod_{p\in P} p \, \Big|\prod_{j=1}^nr_{1j}\Big\}\Bigg| &=\sum_{Q\subset P}(-1)^{|Q|}\Big|\Big\{(r_{1j}): p \nmid \prod_{j=1}^nr_{1j} \text{ for all } p\in Q \Big\}\Big|\\
&=\sum_{Q\subset P}(-1)^{|Q|}\Biggl(\sum_{R\subset Q}(-1)^{|R|}\Biggl\lfloor\frac{N}{\prod_{p\in R} p}\Biggr\rfloor\Biggr)^n.
\end{align}
	
\noindent Therefore, we conclude that
$$T(l,N)=\sum_{P\subset X_l}{(-1)^{|P|}}\Bigg[\sum_{Q\subset P}(-1)^{|Q|}\Big(\sum_{R\subset Q}(-1)^{|R|}\Big\lfloor\frac{N}{\prod_{p\in R} p}\Big\rfloor\Big)^n\Bigg]^m.$$
\end{proof}


\begin{theorem} The probability of the GCD of products of random integers is not divisible by the first $l$ primes greater than $B$ is given by 
$$\lim_{N\to\infty} \frac{T(l,N)}{N^{nm}} = \prod_{i=1}^{l} \Big(1 - \Big[1 - \Big(1-\frac{1}{p^i}\Big)^n\Big]^m\Big).$$
\end{theorem}

\begin{proof}
By the work done in the section so far, we see that $\displaystyle \frac{T(l,N)}{N^{nm}}$ is equal to 
 \begin{align}
&\frac{1}{N^{nm}}\sum_{P\subset X_l}{(-1)^{|P|}}\Big[\sum_{Q\subset P}(-1)^{|Q|}\Big(\sum_{R\subset Q}(-1)^{|R|}\Big\lfloor\frac{N}{\prod_{p\in R} p} \Big\rfloor\Big)^n\Big]^m\\
&= \sum_{P\subset X_l}{(-1)^{|P|}} \Big[\sum_{Q\subset P}(-1)^{|Q|} \Big(\sum_{R\subset Q}(-1)^{|R|} \frac{1}{\prod_{p\in R} p}\Big)^n\Big]^m+O\Big(\frac{1}{N^{nm}}\Big).
\end{align}

\noindent Next, multiple applications of \cref{alternating-sum-to-prod} yields
\begin{align}
&\sum_{P\subset X_l}{(-1)^{|P|}} \Big[\sum_{Q\subset P}(-1)^{|Q|} \Big(\sum_{R\subset Q}(-1)^{|R|}\frac{1}{\prod_{p\in R} p}\Big)^n \Big]^m + O\Big(\frac{1}{N^{nm}}\Big)\\ &= \prod_{i=1}^{l} \Big(1 - \Big[1 - \Big(1 - \frac{1}{p_i}\Big)^n\Big]^m \Big) + O\Big(\frac{1}{N^{mn}}\Big). 
\end{align}
	
\noindent Finally, taking the limit as $N \to \infty$ yields the desired conclusion.
\end{proof}

%%%%%%%%%%%%%%%%%%%%%%
\section{Probability that the \texorpdfstring{\(k\)}{\mathbb{k}}-GCD of products of positive integers is  \texorpdfstring{\(B\)}{\mathbb{B}}-Friable}

\subsection{Main content}

\begin{definition} Fix a positive integer $k$. The \textbf{$k$-gcd} of $n$ nonzero integers $x_1, ..., x_n$, written $\gcd_k(x_1,x_2, ...,x_n)$, is the largest integer whose $k^{\text{th}}$ power divides each of $x_1, x_2, ..., x_n$.
\end{definition}	

\noindent Observe that when $k=1$, the $k$-gcd reduces to the classic gcd.

\begin{definition} When $\gcd_k(x_1,x_2,\mathellipsis,x_n)=1$ we say that  $x_1,x_2,\mathellipsis,x_n$ are \textbf{relatively $k$-prime}.
\end{definition}

\noindent \textbf{Example:}
As $k$ varies, it is possible that the $k$-gcd of a set of nonzero integers can change. 

\noindent For instance, although $\gcd(2^7, 2^6 \cdot 3, 2^8 \cdot 5) = 64$, we see that $\gcd_2(2^7, 2^6 \cdot 3, 2^8 \cdot 5)=8$, while $\gcd_3(2^7, 2^6 \cdot 3, 2^8 \cdot 5)=4$ and $\gcd_6(2^7, 2^6 \cdot 3, 2^8 \cdot 5) = 2$. 

\noindent However since $\gcd_8(2^7, 2^6 \cdot 3, 2^8 \cdot 5) = 1$, we find that $2^7, 2^6 \cdot 3, 2^8 \cdot 5$ are relatively $8$-prime.

As in the previous sections, we fix a positive integer $N$. Then, we randomly choose uniformly and independently integers $r_{ij} \in [1, N]$ for each integer $1 \leq i \leq m$ and $1 \leq j \leq n$ where $m \geq 2$ and $n \geq 1$. We want to find the probability that an ordered $m$-tuple $(r_{ij})$ such that 
$$\gcd{}_k\Big(\prod_{j=1}^nr_{1j}, ..., \prod_{j=1}^n r_{mj}\Big)$$
is coprime to the first $l$ primes greater than $B$. 

Before stating the main theorem for this section, we establish the following lemmas that we use in the ensuing derivation below. We first need the following terminology from elementary number theory.

\begin{definition}
Let $a, n$ be integers and $p$ be a prime number. We say that $p$ \textbf{exactly divides} $n$, written as $p^a \mid \mid n$, if $p^a \mid n$ but $p^{a+1} \nmid n$. 
\end{definition}

\begin{lemma}
Fix positive integers $k$ and $N$. Let $Q_{p^k}(N)$ the number of ordered $n$-tuples of positive integers in which each entry is at most $N$ such that $p^k$ does not divide the product of these entries. Then, we have
$$Q_{p^k}(N) = \sum_{a_1+...+a_n < k} \Big[\prod_{j=1}^n \Big(\Big\lfloor \frac{N}{p^{a_j}}\Big\rfloor - \Big\lfloor \frac{N}{p^{a_j + 1}}\Big\rfloor\Big)\Big].$$
\end{lemma}

\begin{proof}
Since the number of positive integers at most $N$ that are divisible by $p^j$ (for some prime $p$ and positive integer $j$) equals $\lfloor\frac{N}{p^j}\rfloor$, it follows that the number of positive integers at most $N$ where $p^j$ \textit{exactly} divides the positive integer is equal to
$$\Big\lfloor\frac{N}{p^j}\Big\rfloor - \Big\lfloor\frac{N}{p^{j+1}}\Big\rfloor.$$

\noindent Next, the number of ordered $n$-tuples $(m_1, ..., m_n)$ of positive integers in which entry is at most $N$ satisfying the conditions $p^{a_j} \mid\mid m_j$ for each $j = 1, ..., n$ is given by the quantity
$$\prod_{j=1}^n \Big(\Big\lfloor\frac{N}{p^{a_j}}\Big\rfloor - \Big\lfloor\frac{N}{p^{a_j+1}}\Big\rfloor\Big).$$

\noindent Therefore, we conclude that $Q_{p^k}(N)$ is given as follows:
$$Q_{p^k}(N) = \sum_{a_1+...+a_n < k} \Big[\prod_{j=1}^n \Big(\Big\lfloor \frac{N}{p^{a_j}}\Big\rfloor - \Big\lfloor \frac{N}{p^{a_j + 1}}\Big\rfloor\Big)\Big].$$
\end{proof}

We now give a less cumbersome representation for $Q_{p^k}(N)$ by estimating the error from removing the floor functions from its expression.

\begin{lemma}
Using the same notation as in the previous lemma, we have the following estimate:
$$Q_{p^k}(N) = \Big(N - \frac{N}{p}\Big)^n \Big(1 + \frac{{}_nH_1}{p} + ... + \frac{{}_nH_{k-1}}{p^{k-1}}\Big) + O(1).$$
\end{lemma}

\begin{proof}
Applying the basic estimate $\lfloor x \rfloor  = x + O(1)$ to the result from the previous lemma, we obtain
\begin{align*} Q_{p^k}(N) &= \sum_{a_1+...+a_n < k} \Big[\prod_{j=1}^n \Big( \frac{N}{p^{a_j}} - \frac{N}{p^{a_j + 1}} + O(1)\Big)\Big]\\ &= \Big(N - \frac{N}{p}\Big)^n \sum_{a_1+...+a_n < k} \frac{1}{p^{a_1 + ... + a_n}} + O(1). \end{align*} 
\noindent Next for each non-negative integer less than $k$, there are ${}_nH_j = \binom{n-1+j}{j}$ solutions in non-negative integers to the equation $a_1+...+a_n = j$. Hence we conclude that
$$Q_{p^k}(N) = \Big(N - \frac{N}{p}\Big)^n \Big(1 + \frac{{}_nH_1}{p} + ... + \frac{{}_nH_{k-1}}{p^{k-1}}\Big) + O(1),$$
as required.
\end{proof}

Now, we are ready to state and prove the main theorem of this section.

\begin{theorem}
Fix positive integers $B$ and $N$, and let $p_1, p_2, ...$ be the primes greater than $B$ in increasing order. Then, the probability of the $k$-GCD of products of random integers is not divisible by the first $l$ primes greater than $B$ is given by 
$$\prod_{i=1}^{l} \Big[1 - \Big(1 - \Big(1-\frac{1}{p_i}\Big)^n \Big(1 + \frac{{}_nH_1}{p} + ... + \frac{{}_nH_{k-1}}{p^{k-1}}\Big)\Big)^m\Big].$$
\end{theorem}

\begin{proof}
Letting $Q_{p^k}(N)$ denote the number of $n$ positive integers each of which is at most $N$ that are not divisible by $p^k$ (as in the previous two lemmas), there are $N^n - Q_{p^k}(N)$ products of $n$ positive integers each of which is at most $N$ that are divisible by $p^k$. Therefore, there are 
$$N^{mn} - (N^n - Q_{p^k}(N))^m$$
ordered $m$-tuples of products of $n$ positive integers each of which is at most $N$ that are not divisible by $p^k$.
Hence, the probability that an ordered $m$-tuples of products of $n$ positive integers each of which is at most $N$ that are not divisible by $p^k$ is equal to
$$\frac{N^{mn} - (N^n - Q_{p^k}(N))^m}{N^{mn}} = 1 - \Big(1 - \frac{Q_{p^k}(N)}{N^n}\Big)^m.$$
Now, we use the result of the previous lemma to rewrite the latter probability as follows:
$$1 - \Big[1 - \Big(1 - \frac{1}{p}\Big)^n \Big(1 + \frac{{}_nH_1}{p} + ... + \frac{{}_nH_{k-1}}{p^{k-1}}\Big)\Big]^m + O\Big(\frac{1}{N^{mn}}\Big).$$
Then, since divisibility by finitely many primes give independent events, we deduce that the probability that an ordered $m$-tuples of products of $n$ positive integers each of which is at most $N$ that are not divisible by $p_1^k$, ..., $p_l^k$ is equal to
\begin{align*} &\prod_{i=1}^l \Big[1 - \Big(1 -\Big(1-\frac{1}{p_i}\Big)^n\Big(1+\frac{{}_nH_1}{p_i}+\mathellipsis+\frac{{}_nH_{k-1}}{p_i^{k-1}}\Big)\Big)^m +O\Big(\frac{1}{N^{mn}}\Big) \Big]\\
&=\prod_{i=1}^l \Big[1 - \Big(1 -\Big(1-\frac{1}{p_i}\Big)^n\Big(1+\frac{{}_nH_1}{p_i}+\mathellipsis+\frac{{}_nH_{k-1}}{p_i^{k-1}}\Big)\Big)^m\Big] + O\Big(\frac{1}{N^{mn}}\Big). \end{align*}

\noindent Letting $N \to \infty$ now gives us the desired result.
\end{proof}

So far, we have found the probability that the $k$-GCD of products of random integers is not divisible by the first \(l\) primes greater than \(B\). We now use this result to find the probability that the $k$-GCD of products of random integers is not divisible by \textit{all} primes greater than $B$, thereby giving the probability that this $k$-GCD is $B$-friable.

\begin{theorem} Suppose that each $r_{ij}$ is chosen uniformly and independently from $\{1, 2, ..., N\}$. Then, the probability that $\gcd_k(\prod_{j=1}^n r_{1j}, ... , \prod_{j=1}^n r_{mj})$ is $B$-friable converges (as $N \to \infty$) to
$$\prod_{p>B} \Big[1 - \Big(1 - \Big(1-\frac{1}{p_i}\Big)^n\Big)^m\Big].$$
\end{theorem}

\todo[inline]{REVIEW ME! Prove via the Dominated Convergence Theorem stuff here!}
\begin{proof}
	Letting $P(l, N)$ denote the probability in the previous theorem (and for convenience setting $P(0, N) = 1$, we define $g_N(l) = P(l-1, n) - P(l, n)$. Then, we see that $g_N(l)$ is precisely the probability that $k\text{-}\gcd(\prod_{j=1}^n r_{1j}, ... , \prod_{j=1}^n r_{mj})$ is coprime to $p^k_1, \mathellipsis, p^k_{l-1}$ and divisible by $p^k_l$ for uniformly and independently chosen $r_{ij}$'s from $\{1, 2, ..., N\}$. In particular, note that $g_N(l)$ is non-negative. We claim that we can move the limit to infinity past the summation sign:
	$$\lim_{N \to \infty} \sum_{s=1}^{\infty} g_N(s) = \sum_{s=1}^{\infty} \lim_{N \to \infty} g_N(s).$$
	In order to accomplish this, we invoke the Dominated Convergence Theorem (for series). To this end, we need to show that $g_N(l)$ is bounded above by $g(l) = \frac{n^m}{p_{l}^{m}}$ and $\sum_{l=1}^{\infty} g(l)$ converges.
	
	We start by showing that $g_N(l)$ is bounded. First of all, observe that
	$$g_N(l) \leq \text{Pr}\Big(p_l^k \Big| \gcd\Big(\prod_{j=1}^n r_{1j}, ... , \prod_{j=1}^n r_{mj}\Big)\Big).$$
	Computing the numerator to the probability on the last line, we find that
	\begin{align*} 
		\Big|\{(r_{ij}) : p_l^k \mid \prod_{j=1}^n r_{1j}\}\Big|^m &= \frac{(N^n - |\{(r_{1j}) : p_l^k \nmid \prod_{j=1}^n r_{1j}\}|)^m}{N^{mn}}\\
		&= \frac{(N^n - |\{r_{11} : p_l^k \nmid r_{11}\}|^n)^m}{N^{mn}}. \end{align*}
	Therefore, it follows that
	\begin{align*} 
		g_N(l) &\leq \frac{(N^n - |\{r_{11} : p_l^k \nmid r_{11}\}|^n)^m}{N^{mn}}\\
		&= \Big[1 - \Big(1 - \frac{1}{N} \Big\lfloor \frac{N}{p_l^k}\Big\rfloor\Big)^n\Big]^m\\
		&\leq \Big[1 - \Big(1 - \frac{1}{p_l^k}\Big)^n\Big]^m\\
		&\leq \frac{n^m}{p_l^{m}},\end{align*}
	where the last inequality directly follows from Bernoulli's inequality. Moreover $\sum_{l=1}^{\infty} g(l)$ converges, because we can bound this series from above with a convergent $p$-series (noting that $m \geq 2$):
	$$\sum_{l=1}^{\infty} g(l) = n^m \sum_{l=1}^{\infty} \frac{1}{p_l^m} < n^m \sum_{j=1}^{\infty} \frac{1}{j^m}.$$
	
	Having satisfied the hypotheses of the Dominated Convergence Theorem, we first observe that since $\sum_{s=1}^l g_N(s)$ is a telescoping sum, we obtain
	$$\sum_{s=1}^l g_N(s) = \sum_{s=1}^l (P(s-1, n) - P(s, n)) = 1 - P(l, N),$$ 
	and thus
	$$\lim_{N \to \infty} \sum_{s=1}^{\infty} g_N(s) = 1 - \prod_{p > B} \Big[1 - \Big(1 - \Big(1 - \frac{1}{p^k_i}  \Big)^n\Big)^m\Big].$$
	
	\noindent Then since $\displaystyle \sum_{s=1}^{\infty} \lim_{N \to \infty} g_N(s)$ represents the complement of the probability we wanted to compute, the Dominated Convergence Theorem yields the desired assertion.
\end{proof}

%%%%%%%%%%%%%%%%%%%
%%%%%%%%%%%%%%%%%%%

%\begin{lemma}
%Fix positive integers $k$ and $N$ as well as a prime number $p$. Then, the probability that a positive integer that is no bigger than $N$ that \textit{exactly} divides $p^k$ is given by
%$$\frac{p-1}{p^{k+1}} + O\Big(\frac{1}{N}\Big).$$
%\end{lemma}

%\begin{proof}
%Since the number of positive integers at most $N$ that are divisible by %$p^k$ equals $\lfloor\frac{N}{p^k}\rfloor$, it follows that the number of %positive integers at most $N$ where $p^k$ \textit{exactly} divides the %positive integer is given by
%$\lfloor\frac{N}{p^k}\rfloor - \lfloor\frac{N}{p^{k+1}}\rfloor$. Hence, the probability that $p^k$ exactly divides a positive integer that is no bigger than $N$ is given by
%$$\frac{1}{N} \Big(\Big\lfloor\frac{N}{p^k}\Big\rfloor - \Big\lfloor\frac{N}{p^{k+1}}\Big\rfloor\Big).$$
%In order to remove the floor functions from the expression above, applying Lemma 4.1 implies that the probability in question can be rewritten as
%$$\frac{1}{p^k} - \frac{1}{p^{k+1}} + O\Big(\frac{1}{N}\Big) = \frac{p-1}{p^{k+1}} + O\Big(\frac{1}{N}\Big),$$
%as required. 
%\end{proof}

%Next, we fix $i \in \{1, ..., m\}$. Then given an ordered $n$-tuple $(r_{i1}, ..., r_{in})$, the probability that $p^{a_j} \mid\mid r_{ij}$ for each $j = 1, ..., n$ is given by the quantity
%    \begin{align*}
%    \prod_{j=1}^n \frac{1}{N} \Big(\Big\lfloor\frac{N}{p^{a_j}}\Big\rfloor - \Big\lfloor\frac{N}{p^{a_j+1}}\Big\rfloor\Big)
%    &= \prod_{j=1}^n \Big(\frac{p-1}{p^{k+1}} + O\Big(\frac{1}{N}\Big)\Big)\\
%    &= \Big(\frac{p-1}{p}\Big)^n\prod_{j=1}^{n}\Big(\frac{1}{p^{a_j}}+O\Big(\frac{1}{N}\Big)\Big)\\
%   &= \Big(1-\frac{1}{p}\Big)^n\frac{1}{p^{a_1+\mathellipsis+a_n}}+O\Big(\frac{1}{N}\Big).
%    \end{align*}

%\noindent Therefore, the probability that $p^k$ does not divide %$\prod_{j=1}^n r_{ij}$ is given by
%    \begin{align*}
    %\Pr(p^k \nmid r_{ij} \text{ for all \(j\)}) &=  \sum_{a_1+\mathellipsis+a_n < k}\Pr(p^{a_j} \mid\mid r_{ij} \text{ for all \(j\)\,})\\
%    &\sum_{a_1+\mathellipsis+a_n < k}  \Big[\Big(1-\frac{1}{p}\Big)^n \frac{1}{p^{a_1+\mathellipsis+a_n}}+O\Big(\frac{1}{N}\Big)\Big]\\
%    &= \bigg(1-\frac{1}{p}\bigg)^n\bigg(1+\frac{{}_nH_1}{p}+\mathellipsis+\frac{{}_nH_{k-1}}{p^{k-1}}\bigg)+O\Big(\frac{1}{N}\Big).
%    \end{align*}

\subsection{Alternate k-gcd stuff}

\begin{theorem} Let $p$ be a prime number and $r_{ij}$ be nonzero integers for each $1 \leq i \leq m$ and $1 \leq j \leq n$. Then
$p \mid \gcd_k(\prod_{j=1}^{n}r_{1j},\mathellipsis,\prod_{j=1}^n r_{mj})$ if and only if $p^k \mid \prod_{j}r_{ij}$ for some $i$.
\end{theorem}

\begin{proof}
Suppose that $p \mid a$ where $a = \gcd_k(\prod_{j=1}^n r_{1j}, ...,\prod_{j=1}^n r_{mj})$. This is true if and only if $p \mid a$. Then, this is equivalent to $p^k \mid a^k$ as well as $p^k \mid \prod_j r_{ij}$ for some $i$.
\end{proof}
 
	\begin{theorem}
		Let \(p_1, p_2,\mathellipsis\) be the prime numbers larger than \(B\) in increasing order.
		\begin{equation}
			\lim\limits_{N\to\infty} \frac{T_k(l,n)}{N^{mn}}=\prod_{i=1}^l\Biggl(1-\Biggl[1-\Biggl(1-\frac{1}{p_i}\Biggr)^n\Biggl(1+\frac{nH_1}{p_i}+\mathellipsis+\frac{nH_{k-1}}{p_i^{k-1}}\Biggr)\Biggr]\Biggr)
		\end{equation}
	\end{theorem}

	\begin{proof}
		  Let \(X_l=\{p_1,\mathellipsis,p_l\}\) and \(1\leq r_{ij}\leq N\). Using the inclusion exclusion theorem, 
		  \begin{equation}
		  	\frac{T_k(l,n)}{N^{nm}}=\sum_{P\subset X_l}(-1)^{|P|}\Biggl(\sum_{Q\subset P}\Pr\Biggl[p^k\nmid \prod_{j=1}^{n}r_{1j} \; \forall p\in Q \Biggr]\Biggr)^m
		  \end{equation}
		  
		  Let \(p^a \mid\mid x\) denote \(p^a\mid x\) and \(p^{a+1}\nmid x\) and let \(a_p,j\in Q\) and \(1\leq j \leq n\).
		  
		  \begin{remark}
		  	Regarding \(||\), suppose a tuple satisfies \(a_{p,1}+\mathellipsis+a_{p,n}<k\) with \(a_{p,1}+\mathellipsis+a_{p,n}+1=k\), then \(||\) guarantees \(p^k \nmid x\).
		  \end{remark} 
		  
		  
		  The number of \(n\)-tuples of positive integers which satisfy \(a_{p,1}+\mathellipsis+a_{p,n}=i\) is given by \(_nH_i = (\binom{n+i-1}{i})\)
		  \begin{remark}
		  	The previous statement is a result from combinatorics \cite{noauthor_composition_2022}\cite{noauthor_stars_2022}.
		  \end{remark}
		  
		  Next, we count every composition \(a_{p,1}+\mathellipsis+a_{p,n}<k\) for \(p^{a_{p,j}}\) that divides the quantity.
		  
		  Thus \begin{align}
		  	&\Pr\Biggl[p^k\nmid \prod_{j=1}^{n}r_{1j},\:\forall p\in Q\Biggr] = \sum_{a_{p,1}+\mathellipsis+a_{p,n}<l}\Pr\Bigl[p^{a_{p,j}} \mid\mid r_{1j},\: \forall p,j\Bigr]\\
		  	 &=\sum_{a_{p,1}+\mathellipsis+a_{p,n}<k}\quad\prod_{j=1}^{n}\Pr\Bigl[p^{a_{p,j}} \mid\mid r_{1j} \text{, for all } p\in Q\Bigr],  
		  \end{align}
	  
	  	where the last term utilizes the product rule for probability. This allows us to count the innermost quantity. 
		  
		  Using the inclusion exclusion principle, 
		  \begin{align}
		  	|\{(r_{1j}): p^{a_{{p,j}}}\mid\mid r_{1j} \: \forall p\in Q\}| = \\
		  	\Biggl\lfloor\frac{N}{\prod_{p\in Q}p^{a_{p,j}}}\Biggr\rfloor-\sum_{p\in Q}\Biggl\lfloor \frac{N}{\prod_{p\in Q}p^{a_{p,j}}}\Biggr\rfloor+\mathellipsis+(-1)^{|Q|}\Biggl\lfloor\frac{N}{\prod_{p\in Q}p^{a_{p,j}+1}}\Biggr\rfloor
		  \end{align}
		  
		  Next, we form an estimate of the quantity by taking the difference of the last two terms. 
		  
		  \begin{align}
		  	\Biggl\lfloor \frac{N}{\prod {p\in Q} p^{a_{p,j}}} \Biggr\rfloor - \Biggl\lfloor \frac{N}{\prod {p\in Q} p^{a_{p,j}+1}} \Biggr\rfloor =\\
		  	  \frac{N}{\prod {p\in Q} p^{a_{p,j}}} + O(1) -  \frac{N}{\prod {p\in Q} p^{a_{p,j}+1}} + O(1) =\\
		  	   N\prod_ {p\in Q}\Biggl[ \frac{1}{ p^{a_{p,j}}}  -  \frac{1}{ p^{a_{p,j}+1}} \Biggr]+ \frac{2}{N}O(1)=\\
		  	   N\prod_ {p\in Q}\Biggl[ \frac{1}{ p^{a_{p,j}}}  -  \frac{1}{ p^{a_{p,j}+1}} \Biggr]+ O(1)
		  \end{align}
		  
		  Thus we have,
		  \begin{align}
		  	\Pr\Biggl[p^k\nmid \prod_{j=1}^{n}r_{1j},\: \forall p\in Q\Biggr] =\\
		  	 \prod_{p\in Q} \Biggl(\sum_{a_{p,1}+\mathellipsis+a_{p,n}<k}\prod_{j=1}^{n}\frac{p-1}{p^{a_{p,j}+1}}\Biggr)+O\Biggl(\frac{1}{N}\Biggr) = \\
		  	 \prod_{p\in Q}\Biggl[\Biggl(1-\frac{1}{p}\Biggr)^n\sum_{a_{p,1}+\mathellipsis+a_{p,n}<k}\frac{1}{p^{a_{p,1}+\mathellipsis+a_{p,n}}}\Biggr]+O\Biggl(\frac{1}{N}\Biggr)
		  \end{align}
		  
		  Finally, invoking an earlier result yields
		  \begin{align}
		  	\prod_{p\in Q}\Biggl[\Biggl(1-\frac{1}{p}\Biggr)^n\Biggl[\sum_{a_{p,1}+\mathellipsis+a_{p,n}<k}\frac{1}{p^{a_{p,1}+\mathellipsis+a_{p,n}}}\Biggr]+O\Biggl(\frac{1}{N}\Biggr)=\\
		  	 \prod_{p\in Q}\Biggl[\Biggl(1-\frac{1}{p}\Biggr)^n\Biggl(1+\frac{_nH_1}{p}+\mathellipsis+\frac{_nH_{k-1}}{p^{k-1}}\Biggr)\Biggr]+O\Biggl(\frac{1}{N}\Biggr)
		  \end{align}
		  
		  Finally, substitute these results into the original expression and repeatedly apply (\cref{alternating-sum-to-prod}): to simplify as in an earlier result, and let \(N\to \infty\), and the error approaches 0. 
	\end{proof}

\section{Benkoski's Theorem on Algebraic Integers}

\subsection{Background material}

%Cite where we can find more background and details as needed here!
 
	\begin{definition} A complex number is an \textbf{algebraic number} if it satisfies a polynomial with rational coefficients.
	\end{definition}

	\begin{definition}A number is an \textbf{algebraic integer} if it satisfies a monic polynomial with integer coefficients.
		
		The set of all algebraic integers forms a ring and is denoted by \(\mathbb{A}\).
	\end{definition}

	\begin{definition} The \textbf{algebraic number ring} (denoted by \(\mathcal{O}\)) is the ring of algebraic integers of the algebraic field \(K\) given by \(\mathcal{O}=K\cap \mathbb{A}\). 
	\end{definition}

	\begin{definition} 
		We say that an ideal \(\f{p}\) in a number ring $\mathcal{O}$ is \textbf{prime} if whenever \(\f{ab}\subseteq \f{p}\) for some ideals \(\f{a},\f{b}\) in $\mathcal{O}$ implies \(\f{a}\subseteq \f{p}\) or \(\f{b} \subseteq \f{p}\).  
	\end{definition}
A key result is that a nonzero ideal in a Dedekind domain (such as a number ring) enjoys unique factorization into prime ideals. 

\begin{definition} \label{ideal-norm}
		Let $\mathfrak{a}$ be a nonzero ideal of a number ring $\mathcal{O}$. We
		define the \textbf{norm} of \(\f{a}\) as \(\f{N(a)}:=|\mathcal{O}/\f{a}|\).
	\end{definition}

The norm on ideals can be shown to be finite. \cite{marcus_number_2012} 
	
\begin{definition} Let \(\mathcal{O}\) be an algebraic number ring. The \textbf{Dedekind Zeta Function} of \(\mathcal{O}\) is given by \begin{equation}
			\zeta_\mathcal{O}(s)=\sum_{\f{a\subseteq \mathcal{O}}}\frac{1}{\f{N(a)}}
		\end{equation} where the sum is over all non-zero ideals of \(\mathcal{O}\).
	\end{definition}

Equivalently, the Dedekind zeta function can be equivalently defined by the Dirichlet series
$$\zeta_\mathcal{O}(s)=\sum_{n=1}^{\infty}\frac{c_n}{n^s},$$
where \(c_n\) is the number of ideals with norm \(n\).

Moreover, the Dedekind Zeta function can be represented by the Euler Product
$$\zeta_{\mathcal{O}}(s)=\prod_{\f{p \text{ prime}}}(1-\f{N(p)}^{-s})^{-1}.$$

\begin{remark}
As with the Riemann Zeta Function, the Dedekind Zeta Function converges for all \(\text{Re} \, s>1\).
\end{remark}

	\begin{definition} Define the M\"obius Function for Algebraic Number Rings by: 
		\begin{equation}
			\mu:\mathcal{O}\to \{0,\pm 1\} \notag
		\end{equation}
		 \begin{equation}
			\mu(\f{a})=\begin{cases}
				1 & \text{if \(\f{\mathfrak{N}(a)}=1\)}\\
				0 & \text{if \(\f{a\subseteq \f{p}}^2\) for some prime \(\f{p}\) }\\
				(-1)^r & \text{if \(\f{a}=\f{p}_1 \f{p}_2\mathellipsis\f{p}_r\) for distinct primes \( \f{p}_1,\f{p}_2,\mathellipsis, \f{p}_r\)}
			\end{cases}
		\end{equation}
	\end{definition}

	\begin{theorem}
		\label{sum-to-dedekind}
		\begin{equation} 
			\sum_{\f{a\subseteq\mathcal{O}}}\frac{\mu(\f{a})}{\f{\mathfrak{N}(a)}^s} = \prod_{\f{p \text{ prime}}}(1-\f{\mathfrak{N}(p)}^{-s})=\frac{1}{\zeta_{\mathcal{O}}(s)}.		\end{equation}
	\end{theorem}
	\begin{proof}
		We can apply the same techniques as in \cref{mobious-riemann}.
	\end{proof}

	\begin{definition} Fix $r \in \mathbb{N}$. We say that nonzero ideals \(f{a}_1,\f{a}_2,\mathellipsis,\f{a}_k \subseteq \mathcal{O}\)  are \textbf{relatively \(r\)-prime} if \(\f{a}_1,\f{a}_2,\mathellipsis,\f{a}_k\not\subseteq \f{b}^r\) for any nonzero and proper ideal \(\f{b}\).
	\end{definition}

	\begin{definition} \label{H}
		Let \(H(n):\mathbb{R} \to \mathbb{Z}\), denote the number of ideals in \(\mathcal{O}\) with norm less than or equal to \(n\).
	\end{definition}

	\begin{remark}
		Note that \(n\) need not be an integer.
	\end{remark}
	
	\begin{theorem} There is a positive constant \(c\) such that \(H(n)=cn+O(n^{1-\epsilon})\) where \(\epsilon = [K : \mathbb{Q}]^{-1}\) \label{H-estimate}. If \(K=\mathbb{Q}\), as in the case for the integers, then \(H\) reduces to the floor function.
	\end{theorem}
	\begin{proof}
	    Refer to \cite{marcus_number_2012}.
	\end{proof}

\subsection{Benkoski's Theorem for a number ring}

	The following theorem describes the probability that random algebraic integers are relatively \(r\)-prime.
	\begin{theorem}[Benkoski's Theorem for an Algebraic Number Ring]
		Let \(Q(n)\) denote the number of ordered \(k\)-tuples of ideals in \(\mathcal{O}\) with norm less than or equal \(n\) that are relatively prime.
		Fix \(k,r\in\mathbb{N}\) not both equal to \(1\). Then, \[\lim\limits_{n\to\infty}\frac{Q(n)}{H(n)^k}=\frac{1}{\zeta_\mathcal{O}(rk)}\]
	\end{theorem}

	\begin{proof}
		 An ordered \(k\)-tuple of ideals \((\f{a}_1, \f{a}_2, \mathellipsis, \f{a}_k)\) is relatively \(r\)-prime if and only if there exists no prime ideal \(\f{p}\) such that \(\f{a}_1, \f{a}_2,\mathellipsis,\f{a}_k\subset \f{p}^r\).
		
		Then, using the Inclusion-Exclusion principle, as in an earlier case,
		\begin{equation}
			Q(n) = H(n)^k - \sum_{\f{p}_1}H\left(\frac{n}{\f{N}(\f{p}_1^r)}\right)^k+\sum_{\f{p}_1,\f{p}_2}H\left(\frac{n}{\f{N}((\f{p}_1\f{p}_2)^r)}\right)^k-\mathellipsis
		\end{equation}
	Where \(\f{p}_1,\f{p}_2,\mathellipsis\) are distinct prime ideals not counting repetition. 
	
	This can be written more compactly as  
	\begin{equation}
		Q(n)=\sum_{i=1}(-1)^{i+1}\left(\sum_{\f{a}= \prod_{j=1}^{i-1}\f{p}_j}H\left(\frac{n}{\f{N}(\f{a}^r)}\right)^k\right)
	\end{equation}
	\end{proof}
	\begin{remark}
		By convention, the null product equals the multiplicative identity, if it exists. 
	\end{remark}

	Next, we can use the M\"obius function to further simplify,
	\begin{equation}
		Q(n)=\sum_{\f{a}}\mu(\f{a})H\left(\frac{n}{\f{N}(\f{a}^r)}\right)^k
	\end{equation}

	The sum ranges over all nonzero ideals \(\f{a}\) such that \(\f{N}(\f{a})\leq \lfloor \sqrt[r]{n} \rfloor \).
	Recall that since the norm function is totally multiplicative, 
	$$\f{N}(\f{a}) \leq \lfloor \sqrt[r]{n} \rfloor \text{ iff }
		n-1 < \f{N}(\f{a}^r) \leq n.$$
	Note that \(H(n)=0\) whenever \(n<1\), since no ideal can satisfy having a norm less than 1. Thus, whenever \(\f{N}(\f{a}) \leq \left\lfloor\sqrt[r]{n}\right\rfloor\), we must have \(\f{N}(\f{a}^r)> n \), and \(H(n)\) is nonzero. Thus, \(H\) annihilates any summand where the index satisfies \( \f{N}(\f{a})> \lfloor\sqrt[r]{n}\rfloor \)

	Yielding the simplification,
	\begin{equation}
		Q(n)=\sum_{\substack{a\subseteq \mathcal{O}\\ 
				\f{N}(\f{a}) \leq \lfloor \sqrt[r]{n} \rfloor}} \mu(\f{a})H\left(\frac{n}{\f{N}(\f{a}^r)}\right)^k
	\end{equation}
	
	Applying this estimate,
	\begin{equation}
	Q(n)=\sum_{\substack{a\subseteq \mathcal{O}\\ 
			\f{N}(\f{a}) \leq \lfloor \sqrt[r]{n}) \rfloor}} \mu(\f{a})\Biggl[\left(\frac{cn}{\f{N}(\f{a}^r)}\right)^k+\mathcal{O}\left(\frac{n}{\f{N}(\f{a}^r)}\right)^{1-\epsilon}\Biggl]
	\end{equation}

	Next, applying the Binomial Theorem
	\begin{equation}
		Q(n)=(cn)^k\sum_{\substack{a\subseteq \mathcal{O}\\ 
				\f{N}(\f{a}) \leq \lfloor \sqrt[r]{n} \rfloor}}\frac{\mu(\f{a})}{\f{N}(\f{a})^{rk}}+\sum_{\substack{a\subseteq \mathcal{O}\\ 
				\f{N}(\f{a}) \leq \lfloor \sqrt[r]{n} \rfloor}}\mu(\f{a})\left(\frac{cn}{\f{N}(\f{a})^{r}}\right)^{k-1}\mathcal{O}\left(\frac{n}{\f{N}(\f{a}^r)}\right)^{1-\epsilon}
	\end{equation}

	Next, using \cref{sum-to-dedekind}, 
	\begin{equation}
		\sum_{\substack{a\subseteq \mathcal{O}\\ 
				\f{N}(\f{a}) \leq \lfloor\sqrt[r]{n}\rfloor}} \frac{\mu(\f{a})}{\f{N}(\f{a})^{rk}}=\frac{1}{\zeta_{\mathcal{O}}(rk)} - \sum_{\substack{a\subseteq \mathcal{O}\\  				\f{N}(\f{a}) >\lfloor\sqrt[r]{n}\rfloor}}\frac{\mu(\f{a})}{\f{N}(\f{a})^{rk}}
	\end{equation}
	
	Next, the number of ideals with norm exactly \(n\) is given by \(H(n)-H(n-1)=O(n^{1-\epsilon})\).Then by applying a Riemann integral, 
	
	\begin{equation}
		\sum_{\substack{a\subseteq O\\ 
				\f{N}(\f{a}) > \lfloor\sqrt[r]{n}\rfloor}} \frac{\mu(\f{a})}{\f{N}(\f{a})^{rk}} \leq \int_{\sqrt[r]{n}}^{\infty}\frac{cx^{1-\epsilon}}{x^{rk}}\, dx = O(n^{(2-\epsilon-rk)/r})
	\end{equation}
	Using Big O rules to simplify,
	\begin{align}
	 O(n^{(2-\epsilon-rk)/r}) &= O(n^{(2-\epsilon)/r})O(n^{k})\\
	 & =  O(n^{(2-\epsilon)/r})
	\end{align}
	
	Thus we have so far, 
	\begin{equation}
		Q(n)=\frac{(cn)^k}{\zeta_{\mathcal{O}}(rk)}+O(n^{(2-\epsilon)/r}) +\sum_{\f{a}}\mu(\f{a})\left(\frac{cn}{\f{N}(\f{a})^r}\right)^{k-1}O\left(\frac{n}{\f{N}(\f{a})^r}\right)^{1-\epsilon}
	\end{equation}

	Next, we estimate the second sum. Applying Big O rules to simplify, 
	
	\begin{align}	
	&\sum_{\f{a}}\mu(\f{a})\left(\frac{cn}{\f{N}(\f{a})^r}\right)^{k-1}O\left(\frac{n}{\f{N}(\f{a})^r}\right)^{1-\epsilon}\\
	&=\sum_{\f{a}}O\left(\left|\mu(\f{a})\right|\left|\left(\frac{cn}{\f{N}(\f{a})^r}\right)^{k-1}\right|\frac{n}{\f{N}(\f{a})^r}\right)^{1-\epsilon}\notag\\
	&=\sum_{\f{a}}O\left(\left|\mu\left(\f{a}\right)\right|\right)O\left(|c|\right)O\left(\frac{n}{\f{N}(\f{a})^r}\right)^{k-\epsilon}\notag\\
	&=\sum_{\f{a}}O\left(\frac{n^{k-\epsilon}}{\f{N}(\f{a})^{r(k-\epsilon)}}\right)\notag\\
	&= n^{k-\epsilon} \sum_{\f{a}}O\left(\frac{1}{\f{N}(\f{a})^{r(k-\epsilon)}}\right)
	\end{align}

	Applying our estimate for the norm and a Riemann integral, 
	
	\begin{align}
		\sum_{\f{a}}O\left(\frac{1}{\f{N}(\f{a})^{r(k-\epsilon)}}\right) &= \sum_{j=1}^{\lfloor \sqrt[r]{n} \rfloor} O\left(\frac{cj^{1-\epsilon}}{j^{r(k-\epsilon)}}\right)\\
		&= O\left(1+\int_{1}^{\sqrt[r]{n}} \frac{dx}{x^{r(k-1)}}\right)\\
		&= \begin{cases}
			O(\ln n) & \text{if } r=1, k=2 \\
			O(n^\frac{1}{r}) & \text{if } r\leq2, k=1 \\
			O(1) & \text{otherwise}
		\end{cases}
	\end{align}

	Thus we have
	\begin{align}
		Q(n)&=\frac{(cn)^k}{\zeta_{O}(rk)}+O(n^{(2-\epsilon)/r}) + n^{k-\epsilon} \cdot \begin{cases}
			O(\ln n) & \text{if } r=1, k=2 \\
			O(n^\frac{1}{r}) & \text{if } r\leq2, k=1 \\
			O(n^{k-1}) & \text{otherwise}
		\end{cases}\\
		&=\frac{(cn)^k}{\zeta_{O}(rk)}+O(n^{(2-\epsilon)/r}) + \begin{cases}
			O(n^{k-\epsilon}\ln n) & \text{if  } r=1, k=2 \\
			O(n^{k-\epsilon+\frac{1}{r}}) & \text{if } r\leq2, k=1 \\
			O(n^{2k-\epsilon-1}) & \text{otherwise}
		\end{cases}\\ 
	&=\frac{(cn)^k}{\zeta_{O}(rk)} + \begin{cases}
		O(n^{k-\epsilon}) & \text{if  } k>2, \text{ or } r \geq 2\\
		O(n^{2-\epsilon} \ln n) & \text{if  } k=2, r=1 \\
		O(n^{1-\epsilon} \ln n) & \text{if } k=1, \epsilon=\frac{r-2}{r-1} \\
		O(n^{1-\epsilon}) & \text{if } k=1, \epsilon< \frac{r-2}{r-1}\\
		O(n^{(2-\epsilon)/r} & \text{if } k=1, \epsilon > \frac{r-2}{r-1}
	\end{cases}
	\end{align}

	Finally, applying these estimates, 
	\begin{equation}
		\lim\limits_{n\to\infty} \frac{Q(n)}{H(n)^k} = \lim\limits_{n\to\infty} \frac{Q(n)/n^k}{(H(n)/n)^k}=\frac{c^k\zeta_\mathcal{O}(rk)^{-1}+0}{c^k}=\frac{1}{\zeta_\mathcal{O}(rk)}
	\end{equation}

\section{B-Friable Algebraic Integers}
\subsection{Main content}

\begin{definition} Let $B$ be a fixed positive integer and an algebraic number ring $\mathcal{O}$. A nonzero ideal $\mathfrak{a}$ in $\mathcal{O}$ is \textbf{$B$-friable ($B$-smooth)} if $\mathfrak{a}$ has no prime divisor greater than $B$. 
\end{definition}
%Example
For the remainder of this section, we are going to find the probability that the greatest common divisor (GCD) of $m$ products of $n$ nonzero ideals in an algebraic number ring $\mathcal{O}$ (all randomly chosen in a uniform and independent manner) is $B$-friable. 

\vspace{.1 in}

To this end, we fix a positive integer $N$. Then, we randomly choose uniformly and independently ideals $\mathfrak{a}_{ij}$ whose norms at most $N$ for each integer $1 \leq i \leq m$ and $1 \leq j \leq n$ where $m \geq 2$ and $n \geq 1$. We want to find the probability that an ordered $m$-tuple $(\mathfrak{a}_{ij})$ such that 
$$\gcd\Big(\prod_{j=1}^n \mathfrak{a}_{1j}, ..., \prod_{j=1}^n \mathfrak{a}_{mj}\Big)$$
is coprime to the first $l$ prime ideals (arranged in non-decreasing order by norm) having norm greater than $B$. We start with the following error bound.

\begin{lemma} If $k$ is a positive integer, then we have for all sufficiently large $n$
$$\frac{1}{H(n)} H\Big(\frac{n}{k}\Big) = \frac{1}{k} + O\Big(\frac{1}{n^{\epsilon}}\Big).$$
%Where \epsilon is as in the Marcus estimate; cite!
\end{lemma}

\begin{proof}
Since we know from \cite{marcus_number_2012} that $H(x) = cx + O(x^{1-\epsilon})$ for some constants $c, \epsilon > 0$, there exists a constant $A > 0$ such that for all sufficiently large $x$:
$$|H(x) - cx| < Ax^{1-\epsilon}.$$
By using this inequality, we deduce for all sufficiently large $n$ that
$$\frac{1}{H(n)} H\Big(\frac{n}{k}\Big) \leq \frac{\frac{cn}{k} + A(\frac{n}{k})^{1-\epsilon}}{cn - An^{1-\epsilon}} = \frac{\frac{cn}{k} (1 + \frac{Ak^{\epsilon}}{c} n^{-\epsilon})}{cn(1 - \frac{A}{c}n^{-\epsilon})}.$$
Applying the geometric series to the right side of the inequality above, we obtain for all sufficiently large $n$:
$$\frac{1}{H(n)} H\Big(\frac{n}{k}\Big) = \frac{1}{k} \Big(1 + \frac{Ak^{\epsilon}}{c} n^{-\epsilon}\Big) \Big(1 + O\Big(\frac{1}{n^{\epsilon}}\Big)\Big) = \frac{1}{k} + O\Big(\frac{1}{n^{\epsilon}}\Big),$$
and that is what we wanted to prove.
\end{proof}

\begin{theorem}
Fix positive integers $B$ and $N$ and a number ring $\mathcal{O}$. Let $\mathfrak{p}_1, \mathfrak{p}_2, ...$ be the prime ideals in $\mathcal{O}$ with norm greater than $B$ arranged in non-decreasing order by norm. Then, the probability of the GCD of products of random ideals is not divisible by the first $l$ prime ideals greater than $B$ is given by 
$$\prod_{i=1}^{l} \Big[1 - \Big(1 - \Big(1-\frac{1}{\mathfrak{N}(\mathfrak{p}_i)}\Big)^n\Big)^m\Big].$$
\end{theorem}

\begin{proof}
For any prime ideal $\mathfrak{p}$, there are $\bigl[H(N)-H\bigl(\frac{N}{\f{N(p)}}\bigr)\bigr]^n$ products of ideals with norm at most $N$ that are not divisible by $\f{N(p)}$. Hence, there are  $H(N)^n-\bigl[H(N)-H\bigl(\frac{N}{\f{N(p)}}\bigr)\bigr]^n$ products of ideals with norm at most \(N\) that are divisible by $\f{N(p)}$. Therefore, there are 
$$H(N)^{nm}-\Big[H(N)^n - \Big(H(N) - H\Big(\frac{N}{\f{N(p)}}\Big)\Big)^n\Big]^m$$
ordered $m$-tuples of products of ideals, each of which has norm at most $N$ and is not divisible by $\f{N(p)}$. Then, the probability that an ordered $m$-tuple of products of ideals, each of which has norm at most $N$ and is not divisible by $\f{N(p)}$ is equal to
$$\frac{H(N)^{nm}-\Big[H(N)^n - \Big(H(N) - H\Big(\frac{N}{\f{N(p)}}\Big)\Big)^n\Big]^m}{H(N)^{nm}} = 1 - \Big[1 - \Big(1 - \frac{1}{H(N)} H\Big(\frac{N}{\f{N(p)}}\Big)\Big)^n\Big]^m.$$
Then since divisibility by finitely many primes yields independent events, we find that the probability that the GCD of $m$ products of random ideals with norm at most $N$ are not divisible by the first $l$ prime ideals:
$$\prod_{i=1}^{l} \Big[1 - \Big(1 - \Big(1 - \frac{1}{H(N)} H\Big(\frac{N}{\f{N(p)}}\Big)\Big)^n \Big)^m \Big].$$

We now examine what happens to our probability as $N \to \infty$. Using the estimate for $H(n)$ from \cite{marcus_number_2012} then gives us
\begin{align*}
&\prod_{i=1}^{l} \Big[1 - \Big(1 - \Big(1 - \frac{1}{\f{N}(\f{p}_i)} + O\Big(\frac{1}{N^{\epsilon}}\Big) \Big)^n \Big)^m\Big]\\
&=\prod_{i=1}^{l} \Big[1 - \Big(1 - \Big(1 - \frac{1}{\f{N}(\f{p}_i)}\Big)^n \Big)^m \Big] + O\Big(\frac{1}{N^{\epsilon}}\Big).
\end{align*}
Finally, letting $N\to\infty$ gives us the desired result. 
\end{proof}

Setting \(\mathcal{O}=\mathbb{Z}\), prime ideals are substituted for the prime integers, and \(c=1\), yields the same result as for the integers case. As in the case for the integers, this only provides the probability for the first \(l\) prime ideals greater than \(B\). Using the Dominated Convergence Theorem, we can take the limit as \(l\) approaches infinity.

\begin{theorem}
The probability that the GCD of products of random ideals of algebraic integers in a given number ring $\mathcal{O}$ is $B$-Friable converges to
$$\prod_{\f{N}(\f{p})>B}\Big[1 - \Big(1 - \Big(1 - \frac{1}{\f{N}(\f{p})}\Big)^n \Big)^m \Big].$$
\end{theorem}

\begin{proof}
Letting $P(l, N)$ denote the probability in the previous theorem (and for convenience setting $P(0, N) = 1$, we define $g_N(l) = P(l-1, n) - P(l, n)$. Then, we see that $g_N(l)$ is precisely the probability that $\gcd(\prod_{j=1}^n \mathfrak{a}_{1j}, ... , \prod_{j=1}^n \mathfrak{a}_{mj})$ is coprime to $\mathfrak{p}_1, ..., \mathfrak{p}_{l-1}$ and divisible by $\mathfrak{p}_l$ for uniformly and independently chosen $\mathfrak{a}_{ij}$'s from $\{1, 2, ..., N\}$. In particular, note that $g_N(l)$ is non-negative. We claim that we can move the limit to infinity past the summation sign:

$$\lim_{N \to \infty} \sum_{k=1}^{\infty} g_N(k) = \sum_{k=1}^{\infty} \lim_{N \to \infty} g_N(k).$$
In order to accomplish this, we invoke the Dominated Convergence Theorem (for series). To this end, we need to show that $g_N(l)$ is bounded above by $g(l) = \frac{n^m}{\mathfrak{N}(\mathfrak{p}_l)^m}$ and $\sum_{l=1}^{\infty} g(l)$ converges.

We start by showing that $g_N(l)$ is bounded. First of all, observe that
$$g_N(l) \leq \text{Pr}\Big(\mathfrak{p}_l \Big| \gcd\Big(\prod_{j=1}^n \mathfrak{a}_{1j}, ... , \prod_{j=1}^n \mathfrak{a}_{mj}\Big)\Big).$$
Computing the numerator to the probability on the last line, we find that
\begin{align*} 
\Big|\{(r_{ij}) : \mathfrak{p}_l \mid \prod_{j=1}^n \mathfrak{a}_{1j}\}\Big|^m &= \frac{(H(N)^n - |\{(\mathfrak{a}_{1j}) : \mathfrak{p}_l \nmid \prod_{j=1}^n \mathfrak{a}_{1j}\}|)^m}{H(N)^{mn}}\\ &= \frac{(H(N)^n - |\{r_{11} : \mathfrak{p}_l \nmid \mathfrak{a}_{11}\}|^n)^m}{H(N)^{mn}}. \end{align*}
Therefore, it follows for all sufficiently large $N$ that
\begin{align*} 
g_N(l) &\leq \frac{(H(N)^n - |\{r_{11} : \mathfrak{p}_l \nmid \mathfrak{a}_{11}\}|^n)^m}{H(N)^{mn}}\\ &\leq \Big(1 - \Big(1 - \frac{1}{H(N)} H\Big(\frac{N}{\f{N}(\f{p}_l)}\Big)\Big)^n\Big)^m\\
&= \Big(1- \Big(1 - \Big(\frac{1}{\mathfrak{N}(\mathfrak{p}_l)} + \frac{A}{N^{\epsilon}}\Big)\Big)^n \Big)^m \text{ for some } A > 0\\ &\leq \Big(1- \Big(1 - \frac{2}{\mathfrak{N}(\mathfrak{p}_l)}\Big)^n \Big)^m, \end{align*}

The latter expression is bounded above by $g(l) = \frac{(2n)^m}{\f{N(\f{p}_l)}^m}$ by Bernoulli's inequality in the form $1-(1-x)^t\leq xt$ for $t\geq 1$ and $0\leq x \leq 1$. Moreover $\sum_{l=1}^{\infty} g(l)$ converges, because we can bound this series from above with a constant multiple of the Dedekind zeta function (with $m \geq 2$):
$$\sum_{l=1}^{\infty}g(l)\leq (2n)^m\sum_{l=1}^{\infty}\frac{1}{\f{N(\f{p}_l)}^m}\leq (2n)^m \zeta_{\mathcal{O}}(m) < \infty.$$

Having satisfied the hypotheses of the Dominated Convergence Theorem, we first observe that since $\sum_{k=1}^l g_N(k)$ is a telescoping sum, we obtain
$$\sum_{k=1}^l g_N(k) = \sum_{k=1}^l (P(k-1, n) - P(k, n)) = 1 - P(l,N),$$ 
and thus
$$\lim_{N \to \infty} \sum_{k=1}^{\infty} g_N(k) = 1 - \prod_{\mathfrak{N}(\mathfrak{p}) > B} \Big[1 - \Big(1 - \Big(1 - \frac{1}{\mathfrak{N}(\mathfrak{p})}  \Big)^n\Big)^m\Big].$$

\noindent Since $\displaystyle \sum_{k=1}^{\infty} \lim_{N \to \infty} g_N(k)$ represents the complement of the probability we wanted to compute, the Dominated Convergence Theorem yields the desired assertion.
\end{proof}

%%%%%%%%%%%%%%%%%%%%%%%%%%%%%%%%%%%%%%
%%%%%%%%%%%%%%%%%%%%%%%%%%%%%%%%%%%%%%

\subsection{Alternative proof with Inclusion-Exclusion}

\begin{theorem} Fix positive integers $B$ and $N$, and let $\mathfrak{p}_1, \mathfrak{p}_2, ...$ be the prime ideals in $\mathcal{O}$ with norm greater than $B$ arranged in non-decreasing order by norm. If $X_l = \{\mathfrak{p}_1, \mathfrak{p}_2, ..., \mathfrak{p}_l\}$, then
$$T(l,N) = \sum_{P\subset X_l}(-1)^{|P|} \Big(\sum_{Q\subset P} (-1)^{|Q|} \Big[\sum_{R\subset Q}(-1)^{|R|} H\Big(\frac{N}{\f{N}\Big(\prod_{\f{p}\in R}\f{p}\Big)} \Big) \Big]^n \Big)^m.$$
\end{theorem}

\begin{proof}
By the inclusion exclusion principle, 
		\begin{align}
			T(l,N)&=\left|\left\{(\f{a}_{ij}): \gcd\left(\prod_{j=1}^{n}\f{a}_{1j},\mathellipsis,\prod_{j=1}^{n}\f{a}_{mj}\right)\text{ is coprime to } \f{p}_1,\f{p}_2,\mathellipsis,\f{p}_l\right\}\right|\\
			&= \sum_{P\subset X_l}(-1)^{|P|}\left|\left\{(\f{a}_{ij}): \prod_{\f{p}\in P} \f{p} \mid \gcd\left(\prod_{j=1}^{n}\f{a}_{1j},\mathellipsis, \prod_{j=1}^{n}\f{a}_{mj}\right)\right\}\right|\\
			&=\sum_{P\subset X_l}(-1)^{|P|}\left|\left\{(\f{a}_{ij}): \prod_{\f{p}\in P} \f{p} \mid \prod_{j=1}^{n}\f{a}_{1j}\right\}\right|^m
		\end{align}
		 Next, using \cref{H}, \cref{ideal-norm}, and applying the inclusion exclusion principle again,
		\begin{align}
			\left|\Biggl\{(\f{a}_{ij}): \prod_{\f{p}\in P} \f{p} \mid \prod_{j=1}^{n} \f{a}_{1j}\Biggr\}\right| &=\sum_{q\subset P}(-1)^{|Q|}\left\{(\f{a}_{1j}): \f{p}\nmid \prod_{j=1}^n\f{a}_{1j}, \forall p\in Q\right\} \\ 
			&=\sum_{Q\subset P} (-1)^{|Q|}\left[\sum_{R\subset Q}(-1)^{|R|}H\left(\frac{N}{\f{N}\left(\prod_{\f{p}\in R}\f{p}\right)}\right)\right]^n
		\end{align}
	Thus we have, 
	\begin{equation}
		T(l,N) = \sum_{P\subset X_l}(-1)^{|P|} \left(\sum_{Q\subset P} (-1)^{|Q|} \left[\sum_{R\subset Q}(-1)^{|R|}H\left(\frac{N}{\f{N}\left(\prod_{\f{p}\in R}\f{p}\right)}\right)\right]^n\right)^m
	\end{equation}
	\end{proof}




\begin{theorem} The probability of the GCD of products of random ideals in a number ring $\mathcal{O}$ is not divisible by the first $l$ prime ideals greater than $B$ is given by
$$\lim\limits_{N\to\infty}\frac{T(l,N)}{H(N)^{mn}} =\prod_{i=1}^{l}\Big[1-\Big(1 - \Big(1 - \frac{1}{\f{N}(\f{p}_i)}\Big)^n\Big)^m\Big].$$
\end{theorem}

	\begin{proof}
	As in the case for the integers, we begin by multiplying \(\frac{1}{N^{mn}}\) into the innermost sum.
	\begin{align}
		&\frac{T(l,N)}{H(N)^{mn}} \\
		&= \frac{1}{H(N)^{mn}}\sum_{P\subset X_l}(-1)^{|P|} \left\{\sum_{Q\subset P} (-1)^{|Q|} \left[\sum_{R\subset Q}(-1)^{|R|}H\left(\frac{N}{\f{N}\left(\prod_{\f{p}\in R}\f{p}\right)}\right)\right]^n\right\}^m\\
		&=\sum_{P\subset X_l}(-1)^{|P|} \left\{\sum_{Q\subset P} (-1)^{|Q|} \left[\sum_{R\subset Q}(-1)^{|R|}\frac{1}{N}H\left(\frac{N}{\f{N}\left(\prod_{\f{p}\in R}\f{p}\right)}\right)\right]^n\right\}^m
	\end{align}

	Next, we apply Marcus' estimate,
	\begin{align}
		\frac{T(l,N)}{H(N)^{mn}}=&\sum_{P\subset X_l}(-1)^{|P|} \left\{\sum_{Q\subset P} (-1)^{|Q|} \left[\sum_{R\subset Q}(-1)^{|R|}\left(\frac{1}{\f{N}\left(\prod_{\f{p}\in R}\f{p}\right)}\right)\right]^n\right\}^m \\
		&+ O\Biggl(\Biggl[\frac{c}{\f{N}\bigl(\prod_{\f{p}\in R}\f{p}\bigr)}\Biggl]\Biggr[\frac{\f{N}\bigl(\prod_{\f{p}\in R}\f{p}\bigr)}{cN}\Biggr]^{\varepsilon}\Biggr).
	\end{align}

	Next,
	\begin{equation}
	\begin{split}
		\frac{T(l,N)}{H(N)^{mn}}=&\sum_{P\subset X_l}(-1)^{|P|} \Big\{\sum_{Q\subset P} (-1)^{|Q|} \Big[\sum_{R\subset Q}(-1)^{|R|}\frac{1}{\f{N}\Big(\prod_{\f{p}\in R}\f{p}\Big)} \Big]^n\Big\}^m + O\Big(\frac{1}{N^{\epsilon}}\Big).
	\end{split}
	\end{equation}

	Since the ideal norm is multiplicative, we can pull out the big product,
	\begin{equation}
	\begin{split}
		\frac{T(l,N)}{H(N)^{mn}}=&\sum_{P\subset X_l}(-1)^{|P|} \Biggl\{\sum_{Q\subset P} (-1)^{|Q|} \Biggl[\sum_{R\subset Q}(-1)^{|R|}\frac{1}{\prod_{\f{p}\in R}\f{N}\bigl(\f{p}\bigr)}\Biggr]^n\Biggr\}^m + O\Big(\frac{1}{N^{\epsilon}}\Big).
	\end{split}
        \end{equation}

	Next, since \(P\subset X_l \subset \mathcal{P}(\f{p}_1,\f{p}_2,\mathellipsis,\f{p}_l)\), we use multiple applications of \cref{alternating-sum-to-prod},
		$$\frac{T(l,N)}{H(N)^{mn}}=\prod_{i=1}^{l}\Biggr[1-\Biggr\{1-\Biggl(1-\frac{1}{\f{N}(\f{p}_i)}\Biggr)^n\Biggr\}^m\Biggr] + O\Big(\frac{1}{N^{\epsilon}}\Big).$$
		
\noindent Finally letting $N \to \infty$, we conclude that
$$\lim_{N\to\infty} \frac{T(l,N)}{H(N)^{mn}} = \prod_{i=1}^{l}\Big[1-\Big\{1-\Big(1-\frac{1}{\f{N}(\f{p}_i)}\Big)^n\Big\}^m\Big].$$

\end{proof}

\section{k-gcd number ring prototype}

\begin{definition} Fix a positive integer $k$. The \textbf{$k$-gcd} of $n$ prime ideals $\f{a}_1$, $\f{a}_2$,$\mathellipsis$, $\f{a}_n$ in a number ring $\mathcal{O}$, written $\gcd_k(\f{a}_1,\f{a}_2, \mathellipsis,\f{a}_n)$, is the largest prime ideal whose $k^{\text{th}}$ power divides each of $\f{a}_1, \f{a}_2,\mathellipsis, \f{a}_n$.
\end{definition}	

\noindent Observe that when $k=1$, the $k$-gcd reduces to the standard gcd of ideals.

\begin{lemma}
	Fix positive integers $k$ and $N$. Let $Q_{\f{p}^k}(N)$ the number of ordered $n$-tuples of nonzero ideals in a number ring $\mathcal{O}$, in which each entry has norm at most $N$ such that $\f{p}^k$ does not divide the product of these entries. Then, we have
	$$Q_{\f{p}^k}(N) = \sum_{a_1+...+a_n < k} \Big[\prod_{j=1}^n \Big(H\bigg( \frac{N}{\f{N}(\f{p})^{a_j}}\bigg) - H\bigg(\frac{N}{\f{N}(\f{p})^{a_j + 1}}\bigg)\Big)\Big].$$
\end{lemma}

\begin{proof}
	Since the number of nonzero ideals with norm at most $N$ that are divisible by $\f{p}^j$ (for some nonzero prime ideal $\f{p}$ and positive integer $j$) equals $H\big(\frac{N}{\f{N}(\f{p})^j}\big)$, it follows that the number of nonzero prime ideals at most $N$ where $\f{p}^j$ \textit{exactly} divides the positive integer is equal to
	$$H\bigg(\frac{N}{\f{N}(\f{p})^j}\bigg) - H\bigg(\frac{N}{\f{N}(\f{p})^{j+1}}\bigg).$$
	
	\noindent Next, the number of ordered $n$-tuples $(m_1, ..., m_n)$ of nonzero ideals in which entry is at most $N$ satisfying the conditions $\f{p}^{a_j} \mid\mid m_j$ for each $j = 1, ..., n$ is given by the quantity
	$$\prod_{j=1}^n H\bigg(\frac{N}{\f{N}(\f{p})^j}\bigg) - H\bigg(\frac{N}{\f{N}(\f{p})^{j+1}}\bigg).$$
	
	\noindent Therefore, we conclude that $Q_{\f{p}^k}(N)$ is given as follows:
	$$Q_{\f{p}^k}(N) = \sum_{a_1+...+a_n < k} \bigg[\prod_{j=1}^n H\bigg(\frac{N}{\f{N}(\f{p})^j}\bigg) - H\bigg(\frac{N}{\f{N}(\f{p})^{j+1}}\bigg)\bigg].$$ 
\end{proof}

We now give a less cumbersome representation for $Q_{p^k}(N)$ by estimating the error from removing the ideal counting functions from its expression.

\begin{lemma}
	Using the same notation as in the previous lemma, we have the following estimate:
	$$Q_{\f{p}^k}(N) = H(N)^n\Big(1 - \frac{1}{\f{N}(\f{p})}\Big)^n \Big(1 + \frac{{}_nH_1}{\f{N}(\f{p})} + ... + \frac{{}_nH_{k-1}}{\f{N}(\f{p})^{k-1}}\Big) + O(1).$$
\end{lemma}

\begin{proof}
	Applying Marcus' estimate to the result from the previous lemma, for some $c,\epsilon>0$, we obtain
	\begin{align*} 
		Q_{p^k}(N) &= \sum_{a_1+...+a_n < k} H(N)^n\Big[\prod_{j=1}^n \frac{1}{\f{N}(\f{p})^j}-\frac{1}{\f{N}(\f{p})^{j+1}}+O\bigg(\frac{1}{N^\epsilon}\bigg)\Big]\\ 
		&= H(N)^n\Big(1 - \frac{1}{\f{N}(\f{p})}\Big)^n \sum_{a_1+...+a_n < k} \frac{1}{\f{N}(\f{p})^{a_1 + ... + a_n}} + O(1). 
	\end{align*} 
	\noindent Next for each non-negative integer less than $k$, there are \small${}_nH_j = \binom{n-1+j}{j}\ $ \normalsize  solutions in non-negative integers to the equation $a_1+...+a_n = j$. Hence we conclude that
	$$Q_{\f{p}^k}(N) = H(N)^n\Big(1 - \frac{1}{\f{N}(\f{p})}\Big)^n \Big(1 + \frac{{}_nH_1}{\f{N}(\f{p})} + ... + \frac{{}_nH_{k-1}}{\f{N}(\f{p})^{k-1}}\Big) + O(1),$$
	as required.
\end{proof}

Now, we are ready to state and prove the main theorem of this section.

\begin{theorem}
	Fix positive integers $B$ and $N$, and let $\f{p}_1, \f{p}_2, ...$ be the prime ideals in a number ring $\mathcal{O}$ with norm greater than $B$ in increasing order. Then, the probability of the $k$-GCD of products of random nonzero ideals in a number ring $\mathcal{O}$ is not divisible by the first $l$ prime ideals with norm greater than $B$ is given by 
	$$\prod_{i=1}^{l} \Big[1 - \Big(1 - \Big(1-\frac{1}{\mathfrak{N}(\f{p}_i)}\Big)^n \Big(1+\frac{{}_nH_1}{\f{N}(\f{p}_i)}+\mathellipsis+\frac{{}_nH_{k-1}}{\f{N}(\f{p}_i)^{k-1}}\Big)\Big)^m\Big].$$
\end{theorem}

\begin{proof}
Letting $Q_{\f{p}^k}(N)$ denote the number of $n$ non zero ideals each of which have norm at most $N$ that are not divisible by $\f{p}^k$ (as in the previous two lemmas), there are $H(N)^n - Q_{\f{p}^k}(N)$ products of $n$ nonzero ideals each of which has norm at most $N$ that are divisible by $\f{p}^k$. Therefore, there are 
$$H(N)^{mn} - (H(N)^n - Q_{\f{p}^k}(N))^m$$
ordered $m$-tuples of products of $n$ nonzero ideals each of which is at most $N$ that are not divisible by $\f{p}^k$.

Hence, the probability that an ordered $m$-tuples of products of $n$ nonzero ideals each of which has norm at most $N$ that are not divisible by $\f{p}^k$ is equal to
$$\frac{H(N)^{mn} - (H(N)^n - Q_{\f{p}^k}(N))^m}{H(N)^{mn}} = 1 - \Big(1 - \frac{Q_{\f{p}^k}(N)}{H(N)^n}\Big)^m.$$

Now, we use the result of the previous lemma to rewrite the latter probability as follows:
$$1 - \Big[1 - \Big(1 - \frac{1}{\f{N}(\f{p})}\Big)^n \Big(1 + \frac{{}_nH_1}{\f{N}(\f{p})} + ... + \frac{{}_nH_{k-1}}{\f{N}(\f{p})^{k-1}}\Big)\Big]^m+O\bigg(\frac{1}{H(N)^{nm}}\bigg).$$
Then, since divisibility by finitely many nonzero prime ideals give independent events, we deduce that the probability that an ordered $m$-tuples of products of $n$ nonzero ideals each of which is at most $N$ that are not divisible by $\f{p}_1^k,\mathellipsis,\f{p}_l^k$ is equal to
\begin{align*} &\prod_{i=1}^l \Bigg[1 - \Big[1 - \Big(1 - \frac{1}{\f{N}(\f{p}_i)}\Big)^n \Big(1 + \frac{{}_nH_1}{\f{N}(\f{p}_i)} + ... + \frac{{}_nH_{k-1}}{\f{N}(\f{p}_i)^{k-1}}\Big)\Big]^m + O\bigg(\frac{1}{H(N)^{nm}}\bigg) \Bigg]\\
&=\prod_{i=1}^l \Big[1 - \Big(1 -\Big(1-\frac{1}{\f{N}(\f{p}_i)}\Big)^n\Big(1+\frac{{}_nH_1}{\f{N}(\f{p}_i)}+\mathellipsis+\frac{{}_nH_{k-1}}{\f{N}(\f{p}_i)^{k-1}}\Big)\Big)^m\Big] + O\Big(\frac{1}{H(N)^{mn}}\Big). \end{align*}
	
\noindent Letting $N \to \infty$ now gives us the desired result.
\end{proof}

So far, we have found the probability that the $k$-GCD of products of random nonzero ideals in a number ring $\mathcal{O}$ is not divisible by the first \(l\) nonzero prime ideals greater than \(B\). We now use this result to find the probability that the $k$-GCD of products of random nonzero ideals is not divisible by \textit{all} non zero prime ideals with norm greater than $B$, thereby giving the probability that this $k$-GCD is $B$-friable.


%\printbibliography

\end{document}